%% ================================================================================
%% This LaTeX file was created by AbiWord.                                         
%% AbiWord is a free, Open Source word processor.                                  
%% More information about AbiWord is available at http://www.abisource.com/        
%% ================================================================================

\documentclass[a4paper,portrait,12pt]{article}
\usepackage[latin1]{inputenc}
\usepackage{calc}
\usepackage{setspace}
\usepackage{fixltx2e}
\usepackage{graphicx}
\usepackage{multicol}
\usepackage[normalem]{ulem}
\usepackage{color}
\usepackage{hyperref}
 
\begin{document}


\begin{flushleft}
CL 452 - Design Project
\end{flushleft}


\begin{flushleft}
Spring 2019
\end{flushleft}


\begin{flushleft}
Group 9
\end{flushleft}





\begin{flushleft}
Nitric Acid
\end{flushleft}


\begin{flushleft}
Final Project Report
\end{flushleft}


\begin{flushleft}
Members
\end{flushleft}


\begin{flushleft}
Vishal Jajodia : 15B030014
\end{flushleft}


\begin{flushleft}
Abhishek Pawan Sharma : 150020010
\end{flushleft}


\begin{flushleft}
Shubhangi Chuhadia : 150020072
\end{flushleft}


\begin{flushleft}
Hrishav Ranjan : 150020074
\end{flushleft}


\begin{flushleft}
Kshitij Chandrachoor : 150020058
\end{flushleft}





\begin{flushleft}
Department of Chemical Engineering
\end{flushleft}


\begin{flushleft}
Indian Institute of Technology Bombay
\end{flushleft}





\begin{flushleft}
\newpage
Executive Summary
\end{flushleft}


\begin{flushleft}
This report speaks in detail about the manufacturing of Nitric Acid. Chapter 1 is about
\end{flushleft}


\begin{flushleft}
the molecule, its properties and its uses in the industry. Information about different
\end{flushleft}


\begin{flushleft}
grades of Nitric Acid is also mentioned in this chapter. We then move on to a market
\end{flushleft}


\begin{flushleft}
survey in Chapter 2, which provides an insight into the patterns observed over the past
\end{flushleft}


\begin{flushleft}
decade in worldwide nitric acid production and consumption. Top market producers
\end{flushleft}


\begin{flushleft}
in both the domestic and global markets were identified and their technology providers
\end{flushleft}


\begin{flushleft}
were also identified. We then finalise our new plant location to be Hazira, Gujarat,
\end{flushleft}


\begin{flushleft}
based on factors mentioned and discussed in Chapter 2.
\end{flushleft}





\begin{flushleft}
In Chapter 3, multiple production routes for Nitric Acid are discussed, varying from
\end{flushleft}


\begin{flushleft}
different laboratory methods and the commercially used Ostwald Process. Since the
\end{flushleft}


\begin{flushleft}
laboratory methods are not feasible, Ostwald Process is discussed in further detail, explaining the stages and the operating parameters for each stage. Variations are available
\end{flushleft}


\begin{flushleft}
in Ostwald Process, which are useful for plants of different capacities. Processes are differentiated on the basis of the operating pressures in multiple stages of the process.
\end{flushleft}


\begin{flushleft}
Single Pressure and Dual Pressure processes are explained and compared with each
\end{flushleft}


\begin{flushleft}
other. Based on our requirement and multiple factors discussed in this section, we finally select the Uhde Dual Pressure Nitric Acid Process to be implemented in our new
\end{flushleft}


\begin{flushleft}
plant. We choose Thyssenkrupp Industrial Solutions (Uhde GmbH) as our technology
\end{flushleft}


\begin{flushleft}
model.
\end{flushleft}





\begin{flushleft}
To get a better insight into the Uhde Dual Pressure Nitric Acid Process, we visited
\end{flushleft}


\begin{flushleft}
Deepak Fertilisers and Petrochemicals Corp. Ltd.'s Taloja facility. Chapter 4 discusses
\end{flushleft}


\begin{flushleft}
this visit in detail. The visit helped us indentify the actual operating parameters for the
\end{flushleft}


\begin{flushleft}
Uhde Dual Pressure Process and also helped us gather important information, relevant
\end{flushleft}


\begin{flushleft}
to equipment sizing and costing. Based on the information gathered from our visit,
\end{flushleft}


\begin{flushleft}
we created our Process Flow Diagram. Chapter 5 discusses the Uhde Dual Pressure
\end{flushleft}


\begin{flushleft}
Process, as implemented in the PFD. We also identify the intensive Heat Integration
\end{flushleft}


\begin{flushleft}
already available in this well defined process.
\end{flushleft}





\begin{flushleft}
ii
\end{flushleft}





\begin{flushleft}
\newpage
Executive Summary
\end{flushleft}





\begin{flushleft}
iii
\end{flushleft}





\begin{flushleft}
In Chapter 6, we carry out an analytical Mass Balance, providing sample calculations
\end{flushleft}


\begin{flushleft}
for each major equipment. Chapter 7 provides a look into the flowsheet simulation of
\end{flushleft}


\begin{flushleft}
the entire process. The simulation was performed on DWSIM. Comparisons between
\end{flushleft}


\begin{flushleft}
the analytical mass balance and flowsheet results is also done in this section. Chapter
\end{flushleft}


\begin{flushleft}
8 shows the sensitivity analysis done on the flowsheet for the reactor, heat exchangers
\end{flushleft}


\begin{flushleft}
and the absorption column to optimise the process operation.
\end{flushleft}





\begin{flushleft}
Equipment Sizing and Costing is done in Chapter 9. Sample calculations are done
\end{flushleft}


\begin{flushleft}
for each major equipment, and subsequent costing calculations are also tabulated, using costing relations and nomograms for this exercise. The total purchased equipment
\end{flushleft}


\begin{flushleft}
delivered cost is used to calculate the total capital investment required for our plant,
\end{flushleft}


\begin{flushleft}
which comes out to be INR 894 Crores. Payback Period calculations are also shown in
\end{flushleft}


\begin{flushleft}
this section, after total product costs and revenue calculations are carried out. Plant
\end{flushleft}


\begin{flushleft}
economics for our process was compared with a recently setup Uhde Plant in Donaldsonville, Louisiana, USA.
\end{flushleft}





\begin{flushleft}
Finally, Environmental Impact Analysis is shown in Chapter 12, detailing the emission
\end{flushleft}


\begin{flushleft}
limits in India, process required to bring our emissions within these statutory limits and
\end{flushleft}


\begin{flushleft}
the economic impact these additions have on our plant.
\end{flushleft}





\begin{flushleft}
\newpage
Acknowledgement
\end{flushleft}


\begin{flushleft}
We would like to take this opportunity to thank our panel members for providing us
\end{flushleft}


\begin{flushleft}
constant support and constructive feedbacks, throughout the Design Project exercise.
\end{flushleft}





\begin{flushleft}
We would also like to thank Prof. Sanjay Mahajani, Department of Chemical Engineering, IIT Bombay for setting up a visit to the Deepak Fertilisers and Petrochemicals
\end{flushleft}


\begin{flushleft}
Corp. Ltd. facility in Taloja, Maharashtra.
\end{flushleft}





\begin{flushleft}
At many stages of our project, we acquired guidance from Mr. Amrish Dholakia, Engineering Manager, GM - Processes, Thyssenkrupp Industrial Solutions, India. Many
\end{flushleft}


\begin{flushleft}
tasks such as Equipment Sizing, Costing, Plant Economics, etc. would not have been
\end{flushleft}


\begin{flushleft}
possible without his inputs.
\end{flushleft}


\begin{flushleft}
Finally, we would like to express our gratitude to the Overall Supervisors, Nitric Acid
\end{flushleft}


\begin{flushleft}
Process Head and the Staff of Deepak Fertilisers and Petrochemicals Corp. Ltd., Taloja,
\end{flushleft}


\begin{flushleft}
Maharashtra for helping us understand the intricacies of the Nitric Acid Process during
\end{flushleft}


\begin{flushleft}
our visit to their facility.
\end{flushleft}





\begin{flushleft}
iv
\end{flushleft}





\begin{flushleft}
\newpage
Contents
\end{flushleft}


\begin{flushleft}
Executive Summary
\end{flushleft}





\begin{flushleft}
ii
\end{flushleft}





\begin{flushleft}
Acknowledgement
\end{flushleft}





\begin{flushleft}
iv
\end{flushleft}





\begin{flushleft}
List of Figures
\end{flushleft}





\begin{flushleft}
viii
\end{flushleft}





\begin{flushleft}
List of Tables
\end{flushleft}





\begin{flushleft}
ix
\end{flushleft}





\begin{flushleft}
1 Nitric Acid
\end{flushleft}


\begin{flushleft}
1.1 Properties . . . . . . . . . . . . . . . .
\end{flushleft}


\begin{flushleft}
1.2 Uses of Nitric Acid . . . . . . . . . . .
\end{flushleft}


\begin{flushleft}
1.3 Uses of Intermediates and Byproducts
\end{flushleft}


\begin{flushleft}
1.4 Grades of Nitric Acid . . . . . . . . .
\end{flushleft}


\begin{flushleft}
2 Market Survey
\end{flushleft}


\begin{flushleft}
2.1 World Consumption Patterns . .
\end{flushleft}


\begin{flushleft}
2.2 World Production Patterns . . .
\end{flushleft}


\begin{flushleft}
2.3 Top Market Producers . . . . . .
\end{flushleft}


\begin{flushleft}
2.4 Technology Providers . . . . . . .
\end{flushleft}


\begin{flushleft}
2.5 Location Selection for New Plant
\end{flushleft}





.


.


.


.


.





\begin{flushleft}
3 Production Process
\end{flushleft}


\begin{flushleft}
3.1 Production Routes . . . . . . . . .
\end{flushleft}


\begin{flushleft}
3.2 Ostwald Process . . . . . . . . . .
\end{flushleft}


\begin{flushleft}
3.2.1 Ammonia Oxidation . . . .
\end{flushleft}


\begin{flushleft}
3.2.2 Nitric Oxide Oxidation . .
\end{flushleft}


\begin{flushleft}
3.2.3 Absorption . . . . . . . . .
\end{flushleft}


\begin{flushleft}
3.3 Deviations in Ostwald Process . . .
\end{flushleft}


\begin{flushleft}
3.3.1 Single Pressure Processes .
\end{flushleft}


\begin{flushleft}
3.3.1.1 Medium Pressure
\end{flushleft}


\begin{flushleft}
3.3.1.2 High Pressure . .
\end{flushleft}


\begin{flushleft}
3.3.2 Dual Pressure Process . . .
\end{flushleft}


\begin{flushleft}
3.4 Selection of Final Process . . . . .
\end{flushleft}





.


.


.


.


.





.


.


.


.


.


.


.


.


.


.


.





.


.


.


.


.





.


.


.


.


.


.


.


.


.


.


.





.


.


.


.





.


.


.


.


.





.


.


.


.


.


.


.


.


.


.


.





.


.


.


.





.


.


.


.


.





.


.


.


.


.


.


.


.


.


.


.





.


.


.


.





.


.


.


.


.





.


.


.


.


.


.


.


.


.


.


.





.


.


.


.





.


.


.


.


.





.


.


.


.


.


.


.


.


.


.


.





.


.


.


.





.


.


.


.


.





.


.


.


.


.


.


.


.


.


.


.





.


.


.


.





.


.


.


.


.





.


.


.


.


.


.


.


.


.


.


.





.


.


.


.





.


.


.


.


.





.


.


.


.


.


.


.


.


.


.


.





.


.


.


.





.


.


.


.


.





.


.


.


.


.


.


.


.


.


.


.





.


.


.


.





.


.


.


.


.





.


.


.


.


.


.


.


.


.


.


.





.


.


.


.





.


.


.


.


.





.


.


.


.


.


.


.


.


.


.


.





.


.


.


.





.


.


.


.


.





.


.


.


.


.


.


.


.


.


.


.





.


.


.


.





.


.


.


.


.





.


.


.


.


.


.


.


.


.


.


.





.


.


.


.





.


.


.


.


.





.


.


.


.


.


.


.


.


.


.


.





.


.


.


.





.


.


.


.


.





.


.


.


.


.


.


.


.


.


.


.





.


.


.


.





.


.


.


.


.





.


.


.


.


.


.


.


.


.


.


.





.


.


.


.





.


.


.


.


.





.


.


.


.


.


.


.


.


.


.


.





.


.


.


.





.


.


.


.


.





.


.


.


.


.


.


.


.


.


.


.





.


.


.


.





.


.


.


.


.





.


.


.


.


.


.


.


.


.


.


.





.


.


.


.





.


.


.


.


.





.


.


.


.


.


.


.


.


.


.


.





.


.


.


.





1


1


2


3


4





.


.


.


.


.





5


5


6


8


9


9





.


.


.


.


.


.


.


.


.


.


.





11


11


12


12


12


13


14


14


14


14


15


15





\begin{flushleft}
4 Industrial Visit
\end{flushleft}


18


\begin{flushleft}
4.1 Process Highlights . . . . . . . . . . . . . . . . . . . . . . . . . . . . . . . 19
\end{flushleft}


\begin{flushleft}
4.1.1 Ammonia . . . . . . . . . . . . . . . . . . . . . . . . . . . . . . . . 19
\end{flushleft}


\begin{flushleft}
4.1.2 Ammonia Filters . . . . . . . . . . . . . . . . . . . . . . . . . . . . 19
\end{flushleft}


\begin{flushleft}
v
\end{flushleft}





\newpage
4.2





\begin{flushleft}
4.1.3 Air Filters . . . . . . .
\end{flushleft}


\begin{flushleft}
4.1.4 Ammonia - Air Mixer
\end{flushleft}


\begin{flushleft}
4.1.5 Turbine Shaft . . . . .
\end{flushleft}


\begin{flushleft}
4.1.6 Reactor . . . . . . . .
\end{flushleft}


\begin{flushleft}
4.1.7 Boiler . . . . . . . . .
\end{flushleft}


\begin{flushleft}
4.1.8 Cooler Condenser . . .
\end{flushleft}


\begin{flushleft}
4.1.9 Absorption Column .
\end{flushleft}


\begin{flushleft}
Conclusion . . . . . . . . . .
\end{flushleft}





.


.


.


.


.


.


.


.





.


.


.


.


.


.


.


.





.


.


.


.


.


.


.


.





.


.


.


.


.


.


.


.





.


.


.


.


.


.


.


.





.


.


.


.


.


.


.


.





.


.


.


.


.


.


.


.





.


.


.


.


.


.


.


.





.


.


.


.


.


.


.


.





.


.


.


.


.


.


.


.





.


.


.


.


.


.


.


.





.


.


.


.


.


.


.


.





.


.


.


.


.


.


.


.





.


.


.


.


.


.


.


.





.


.


.


.


.


.


.


.





.


.


.


.


.


.


.


.





.


.


.


.


.


.


.


.





.


.


.


.


.


.


.


.





.


.


.


.


.


.


.


.





.


.


.


.


.


.


.


.





.


.


.


.


.


.


.


.





.


.


.


.


.


.


.


.





.


.


.


.


.


.


.


.





.


.


.


.


.


.


.


.





.


.


.


.


.


.


.


.





19


19


20


20


20


21


21


21





\begin{flushleft}
5 Process Flow Diagram
\end{flushleft}


22


\begin{flushleft}
5.1 Process Flow Diagram . . . . . . . . . . . . . . . . . . . . . . . . . . . . . 22
\end{flushleft}


\begin{flushleft}
5.2 Process Explanation . . . . . . . . . . . . . . . . . . . . . . . . . . . . . . 22
\end{flushleft}


\begin{flushleft}
5.3 Process Heat Integration . . . . . . . . . . . . . . . . . . . . . . . . . . . . 27
\end{flushleft}


\begin{flushleft}
6 Mass Balance
\end{flushleft}


\begin{flushleft}
6.1 Sample Calculations . . . . . . . . . . . . . . . . .
\end{flushleft}


\begin{flushleft}
6.1.1 Reactor - R101 . . . . . . . . . . . . . . . .
\end{flushleft}


\begin{flushleft}
6.1.2 Heat Exchangers and Cooler Condenser . .
\end{flushleft}


\begin{flushleft}
6.1.3 Absorption Column and Cooler Condenser
\end{flushleft}


\begin{flushleft}
6.2 Final Mass Balance Table . . . . . . . . . . . . . .
\end{flushleft}


\begin{flushleft}
7 DWSIM Flowsheet Simulation
\end{flushleft}


\begin{flushleft}
7.1 Kinetics and Specifications used in DWSIM
\end{flushleft}


\begin{flushleft}
7.1.1 Ammonia Oxidation . . . . . . . . .
\end{flushleft}


\begin{flushleft}
7.1.2 Nitric Oxide Oxidation . . . . . . .
\end{flushleft}


\begin{flushleft}
7.1.3 Absorption Column . . . . . . . . .
\end{flushleft}


\begin{flushleft}
7.2 Comparisons . . . . . . . . . . . . . . . . .
\end{flushleft}


\begin{flushleft}
7.2.1 Ammonia Reactor . . . . . . . . . .
\end{flushleft}


\begin{flushleft}
7.2.2 Heat Exchangers . . . . . . . . . . .
\end{flushleft}


\begin{flushleft}
7.2.3 Absorption Column . . . . . . . . .
\end{flushleft}





.


.


.


.


.


.


.


.





.


.


.


.


.


.


.


.





.


.


.


.


.


.


.


.





.


.


.


.


.


.


.


.





.


.


.


.


.





.


.


.


.


.


.


.


.





.


.


.


.


.





.


.


.


.


.


.


.


.





.


.


.


.


.





.


.


.


.


.


.


.


.





.


.


.


.


.





.


.


.


.


.


.


.


.





.


.


.


.


.





.


.


.


.


.


.


.


.





.


.


.


.


.





.


.


.


.


.


.


.


.





.


.


.


.


.





.


.


.


.


.


.


.


.





.


.


.


.


.





.


.


.


.


.


.


.


.





.


.


.


.


.





.


.


.


.


.


.


.


.





.


.


.


.


.





.


.


.


.


.


.


.


.





.


.


.


.


.





.


.


.


.


.


.


.


.





.


.


.


.


.





.


.


.


.


.


.


.


.





.


.


.


.


.





30


30


30


32


33


33





.


.


.


.


.


.


.


.





35


35


36


36


37


38


39


39


40





\begin{flushleft}
8 Sensitivity Analysis
\end{flushleft}


41


\begin{flushleft}
8.1 Reactor Analysis . . . . . . . . . . . . . . . . . . . . . . . . . . . . . . . . 41
\end{flushleft}


\begin{flushleft}
8.2 Heat Exchanger simulated as PFR . . . . . . . . . . . . . . . . . . . . . . 42
\end{flushleft}


\begin{flushleft}
8.3 Absorption Column Analysis . . . . . . . . . . . . . . . . . . . . . . . . . 43
\end{flushleft}


\begin{flushleft}
9 Equipment Sizing and Costing
\end{flushleft}


\begin{flushleft}
9.1 Ammonia Oxidation Reactor . .
\end{flushleft}


\begin{flushleft}
9.1.1 Diameter Calculation . .
\end{flushleft}


\begin{flushleft}
9.1.2 Catalyst Costs . . . . . .
\end{flushleft}


\begin{flushleft}
9.1.3 Burner Head Calculation
\end{flushleft}


\begin{flushleft}
9.2 Storage Tanks . . . . . . . . . . .
\end{flushleft}


\begin{flushleft}
9.3 Compressors . . . . . . . . . . . .
\end{flushleft}


\begin{flushleft}
9.4 Pumps . . . . . . . . . . . . . . .
\end{flushleft}


\begin{flushleft}
9.5 Heat Exchangers . . . . . . . . .
\end{flushleft}


\begin{flushleft}
9.6 Columns . . . . . . . . . . . . . .
\end{flushleft}


\begin{flushleft}
9.6.1 Shell Calculations . . . .
\end{flushleft}


\begin{flushleft}
9.6.2 Tray Calculations . . . . .
\end{flushleft}





.


.


.


.


.


.


.


.


.


.


.


\begin{flushleft}
vi
\end{flushleft}





.


.


.


.


.


.


.


.


.


.


.





.


.


.


.


.


.


.


.


.


.


.





.


.


.


.


.


.


.


.


.


.


.





.


.


.


.


.


.


.


.


.


.


.





.


.


.


.


.


.


.


.


.


.


.





.


.


.


.


.


.


.


.


.


.


.





.


.


.


.


.


.


.


.


.


.


.





.


.


.


.


.


.


.


.


.


.


.





.


.


.


.


.


.


.


.


.


.


.





.


.


.


.


.


.


.


.


.


.


.





.


.


.


.


.


.


.


.


.


.


.





.


.


.


.


.


.


.


.


.


.


.





.


.


.


.


.


.


.


.


.


.


.





.


.


.


.


.


.


.


.


.


.


.





.


.


.


.


.


.


.


.


.


.


.





.


.


.


.


.


.


.


.


.


.


.





.


.


.


.


.


.


.


.


.


.


.





.


.


.


.


.


.


.


.


.


.


.





.


.


.


.


.


.


.


.


.


.


.





.


.


.


.


.


.


.


.


.


.


.





.


.


.


.


.


.


.


.


.


.


.





.


.


.


.


.


.


.


.


.


.


.





45


45


46


46


47


48


49


50


52


54


54


56





\begin{flushleft}
\newpage
10 Plant Economics
\end{flushleft}


\begin{flushleft}
10.1 Major Equipment Costing . . .
\end{flushleft}


\begin{flushleft}
10.2 Capital Investment Estimation
\end{flushleft}


\begin{flushleft}
10.3 Raw Material Cost . . . . . . .
\end{flushleft}


\begin{flushleft}
10.4 Revenue Calculation . . . . . .
\end{flushleft}


\begin{flushleft}
10.5 Total Product Cost . . . . . . .
\end{flushleft}


\begin{flushleft}
10.6 Payback Period . . . . . . . . .
\end{flushleft}





.


.


.


.


.


.





.


.


.


.


.


.





.


.


.


.


.


.





.


.


.


.


.


.





.


.


.


.


.


.





.


.


.


.


.


.





.


.


.


.


.


.





.


.


.


.


.


.





.


.


.


.


.


.





.


.


.


.


.


.





.


.


.


.


.


.





.


.


.


.


.


.





.


.


.


.


.


.





.


.


.


.


.


.





.


.


.


.


.


.





.


.


.


.


.


.





.


.


.


.


.


.





.


.


.


.


.


.





.


.


.


.


.


.





.


.


.


.


.


.





.


.


.


.


.


.





57


57


58


59


59


60


60





\begin{flushleft}
11 Detailed Design - Absorption Column
\end{flushleft}


\begin{flushleft}
11.1 Mathematical Model . . . . . . . . . .
\end{flushleft}


\begin{flushleft}
11.2 Tray Selection and Specifications . . .
\end{flushleft}


\begin{flushleft}
11.2.1 Weeping Check . . . . . . . . .
\end{flushleft}


\begin{flushleft}
11.2.2 Plate Pressure Drop . . . . . .
\end{flushleft}


\begin{flushleft}
11.2.3 Perforated Area and Hole Pitch
\end{flushleft}


\begin{flushleft}
11.2.4 Number of Holes . . . . . . . .
\end{flushleft}


\begin{flushleft}
11.3 Head Selection . . . . . . . . . . . . .
\end{flushleft}


\begin{flushleft}
11.4 Tower Thickness . . . . . . . . . . . .
\end{flushleft}





.


.


.


.


.


.


.


.





.


.


.


.


.


.


.


.





.


.


.


.


.


.


.


.





.


.


.


.


.


.


.


.





.


.


.


.


.


.


.


.





.


.


.


.


.


.


.


.





.


.


.


.


.


.


.


.





.


.


.


.


.


.


.


.





.


.


.


.


.


.


.


.





.


.


.


.


.


.


.


.





.


.


.


.


.


.


.


.





.


.


.


.


.


.


.


.





.


.


.


.


.


.


.


.





.


.


.


.


.


.


.


.





.


.


.


.


.


.


.


.





.


.


.


.


.


.


.


.





.


.


.


.


.


.


.


.





.


.


.


.


.


.


.


.





.


.


.


.


.


.


.


.





.


.


.


.


.


.


.


.





63


64


67


68


68


68


69


71


71





.


.


.


.


.


.


.


.


.





73


73


74


75


76


76


76


77


77


78





.


.


.


.


.


.





.


.


.


.


.


.





.


.


.


.


.


.





\begin{flushleft}
12 Environmental Impact
\end{flushleft}


\begin{flushleft}
12.1 Tail Gas Composition and Emission Limits
\end{flushleft}


\begin{flushleft}
12.2 Abatement Technology . . . . . . . . . . . .
\end{flushleft}


\begin{flushleft}
12.3 Tail Gas Post Abatement . . . . . . . . . .
\end{flushleft}


\begin{flushleft}
12.4 Revised Plant Economics . . . . . . . . . .
\end{flushleft}


\begin{flushleft}
12.4.1 Reactor Unit Cost . . . . . . . . . .
\end{flushleft}


\begin{flushleft}
12.4.2 Additional Raw Material Costs . . .
\end{flushleft}


\begin{flushleft}
12.4.3 Additional Revenue Calculation . . .
\end{flushleft}


\begin{flushleft}
12.4.3.1 No CER Revenue . . . . .
\end{flushleft}


\begin{flushleft}
12.4.3.2 With CER Revenue . . . .
\end{flushleft}





.


.


.


.


.


.


.


.


.





.


.


.


.


.


.


.


.


.





.


.


.


.


.


.


.


.


.





.


.


.


.


.


.


.


.


.





.


.


.


.


.


.


.


.


.





.


.


.


.


.


.


.


.


.





.


.


.


.


.


.


.


.


.





.


.


.


.


.


.


.


.


.





.


.


.


.


.


.


.


.


.





.


.


.


.


.


.


.


.


.





.


.


.


.


.


.


.


.


.





.


.


.


.


.


.


.


.


.





.


.


.


.


.


.


.


.


.





.


.


.


.


.


.


.


.


.





.


.


.


.


.


.


.


.


.





.


.


.


.


.


.


.


.


.





\begin{flushleft}
A Final Process Flow Diagram
\end{flushleft}





79





\begin{flushleft}
B MATLAB Code for the Mathematical Model for NOX Absorption
\end{flushleft}





81





\begin{flushleft}
C Detailed Drawing of Absorption Column
\end{flushleft}





84





\begin{flushleft}
D DWSIM Flowsheet
\end{flushleft}





86





\begin{flushleft}
References
\end{flushleft}





88





\begin{flushleft}
vii
\end{flushleft}





\begin{flushleft}
\newpage
List of Figures
\end{flushleft}


1.1





\begin{flushleft}
Uses of Nitric Acid . . . . . . . . . . . . . . . . . . . . . . . . . . . . . . .
\end{flushleft}





2





2.1


2.2


2.3





\begin{flushleft}
World Nitric Acid Consumption . . . . . . . . . . . . . . . . . . . . . . . .
\end{flushleft}


\begin{flushleft}
World Nitric Acid Production . . . . . . . . . . . . . . . . . . . . . . . . .
\end{flushleft}


\begin{flushleft}
Nitric Acid Top Market Producers . . . . . . . . . . . . . . . . . . . . . .
\end{flushleft}





6


7


8





3.1





\begin{flushleft}
Block Diagram of Ostwald Process . . . . . . . . . . . . . . . . . . . . . . 13
\end{flushleft}





4.1


4.2


4.3





\begin{flushleft}
Ammonia Flow through Ceramic Filters . . . . . . . . . . . . . . . . . . . 19
\end{flushleft}


\begin{flushleft}
DFPCL Turbine Shaft Structure . . . . . . . . . . . . . . . . . . . . . . . 20
\end{flushleft}


\begin{flushleft}
Nitric Acid Cooler Condenser . . . . . . . . . . . . . . . . . . . . . . . . . 21
\end{flushleft}





5.1


5.2





\begin{flushleft}
Process Flow Diagram for Dual Pressure Nitric Acid Process . . . . . . . 26
\end{flushleft}


\begin{flushleft}
Heat Exchanger Network . . . . . . . . . . . . . . . . . . . . . . . . . . . 29
\end{flushleft}





7.1


7.2


7.3





\begin{flushleft}
DWSIM Flowsheet - Ammonia Oxidation . . . . . . . . . . . . . . . . . . 36
\end{flushleft}


\begin{flushleft}
DWSIM Flowsheet - NO Oxidation and Heat Exchanger Network . . . . . 37
\end{flushleft}


\begin{flushleft}
DWSIM Flowsheet - Reactive Absorption Column Modelling . . . . . . . 38
\end{flushleft}





8.1


8.2


8.3


8.4





\begin{flushleft}
Sensitivity
\end{flushleft}


\begin{flushleft}
Sensitivity
\end{flushleft}


\begin{flushleft}
Sensitivity
\end{flushleft}


\begin{flushleft}
Sensitivity
\end{flushleft}





\begin{flushleft}
Analysis
\end{flushleft}


\begin{flushleft}
Analysis
\end{flushleft}


\begin{flushleft}
Analysis
\end{flushleft}


\begin{flushleft}
Analysis
\end{flushleft}





\begin{flushleft}
for
\end{flushleft}


\begin{flushleft}
for
\end{flushleft}


\begin{flushleft}
for
\end{flushleft}


\begin{flushleft}
for
\end{flushleft}





\begin{flushleft}
Reactor . . . . . . . . .
\end{flushleft}


\begin{flushleft}
Heat Exchanger . . . .
\end{flushleft}


\begin{flushleft}
Absorption Column - 1
\end{flushleft}


\begin{flushleft}
Absorption Column - 2
\end{flushleft}





.


.


.


.





.


.


.


.





.


.


.


.





.


.


.


.





.


.


.


.





.


.


.


.





.


.


.


.





.


.


.


.





.


.


.


.





.


.


.


.





.


.


.


.





.


.


.


.





.


.


.


.





.


.


.


.





.


.


.


.





41


42


43


43





\begin{flushleft}
10.1 Estimated Cash flow from Plant Operation . . . . . . . . . . . . . . . . . 62
\end{flushleft}


\begin{flushleft}
11.1 Sieve Tray Perforated Area with Downcomer . . . . . . . . . . . . . . . . 70
\end{flushleft}


\begin{flushleft}
11.2 Sieve Tray Cooling Coils . . . . . . . . . . . . . . . . . . . . . . . . . . . . 70
\end{flushleft}


\begin{flushleft}
12.1 EnviNOX R Process Variant - 2 . . . . . . . . . . . . . . . . . . . . . . . . 75
\end{flushleft}





\begin{flushleft}
viii
\end{flushleft}





\begin{flushleft}
\newpage
List of Tables
\end{flushleft}


1.1





\begin{flushleft}
Nitric Acid Properties . . . . . . . . . . . . . . . . . . . . . . . . . . . . .
\end{flushleft}





1





2.1


2.2


2.3





\begin{flushleft}
World Nitric Acid Production . . . . . . . . . . . . . . . . . . . . . . . . .
\end{flushleft}


\begin{flushleft}
Major Producers in Global Market . . . . . . . . . . . . . . . . . . . . . .
\end{flushleft}


\begin{flushleft}
Major Producers in Domestic Market . . . . . . . . . . . . . . . . . . . . .
\end{flushleft}





6


8


8





3.1





\begin{flushleft}
Temperature and Pressure Dependence of Ostwald Process . . . . . . . . 14
\end{flushleft}





5.1





\begin{flushleft}
Utility Data after Heat Integration . . . . . . . . . . . . . . . . . . . . . . 28
\end{flushleft}





6.1


6.2


6.3


6.4





\begin{flushleft}
Reactor Balance . . . . . . . . . . . . . .
\end{flushleft}


\begin{flushleft}
Heat Exchanger Balance . . . . . . . . . .
\end{flushleft}


\begin{flushleft}
Absorption Column and Cooler Condenser
\end{flushleft}


\begin{flushleft}
Final Mass Balance . . . . . . . . . . . . .
\end{flushleft}





7.1


7.2


7.3





\begin{flushleft}
Mass Balance Comparison for Reactor . . . . . . . . . . . . . . . . . . . . 39
\end{flushleft}


\begin{flushleft}
Mass Balance Comparison for Heat Exchanger network . . . . . . . . . . . 39
\end{flushleft}


\begin{flushleft}
Mass Balance Comparison for Absorption Column . . . . . . . . . . . . . 40
\end{flushleft}





9.1


9.2


9.3


9.4


9.5


9.6


9.7


9.8


9.9


9.10


9.11


9.12


9.13





\begin{flushleft}
Catalyst Costs . . . . . . . . . . . . . . . . . . . . . . .
\end{flushleft}


\begin{flushleft}
Burner Head Design Parameters . . . . . . . . . . . . .
\end{flushleft}


\begin{flushleft}
Burner Head Costing Parameters . . . . . . . . . . . . .
\end{flushleft}


\begin{flushleft}
Nitric Acid Storage Tank Costing . . . . . . . . . . . . .
\end{flushleft}


\begin{flushleft}
Compressor Power Calculation . . . . . . . . . . . . . .
\end{flushleft}


\begin{flushleft}
Costing of Multi Stage Compressors with Turbines . . .
\end{flushleft}


\begin{flushleft}
Pump Costing Factors . . . . . . . . . . . . . . . . . . .
\end{flushleft}


\begin{flushleft}
Pump Factors . . . . . . . . . . . . . . . . . . . . . . . .
\end{flushleft}


\begin{flushleft}
Costing of Centrifugal Pumps . . . . . . . . . . . . . . .
\end{flushleft}


\begin{flushleft}
Operating Parameters and Costing of Heat Exchangers .
\end{flushleft}


\begin{flushleft}
Column Parameters and Costing . . . . . . . . . . . . .
\end{flushleft}


\begin{flushleft}
Sieve Tray Costing . . . . . . . . . . . . . . . . . . . . .
\end{flushleft}


\begin{flushleft}
Total Purchased Costs for Columns . . . . . . . . . . . .
\end{flushleft}





.


.


.


.


.


.


.


.


.


.


.


.


.





.


.


.


.


.


.


.


.


.


.


.


.


.





.


.


.


.


.


.


.


.


.


.


.


.


.





.


.


.


.


.


.


.


.


.


.


.


.


.





.


.


.


.


.


.


.


.


.


.


.


.


.





.


.


.


.


.


.


.


.


.


.


.


.


.





.


.


.


.


.


.


.


.


.


.


.


.


.





.


.


.


.


.


.


.


.


.


.


.


.


.





.


.


.


.


.


.


.


.


.


.


.


.


.





.


.


.


.


.


.


.


.


.


.


.


.


.





46


47


47


48


49


50


51


51


52


53


55


56


56





10.1


10.2


10.3


10.4


10.5


10.6


10.7





\begin{flushleft}
Total Purchased Equipment Cost . . . . . . . . .
\end{flushleft}


\begin{flushleft}
Total Capital Investment Estimation . . . . . . .
\end{flushleft}


\begin{flushleft}
Total Raw Material Cost Calculation . . . . . . .
\end{flushleft}


\begin{flushleft}
Total Revenue Calculation . . . . . . . . . . . . .
\end{flushleft}


\begin{flushleft}
Total Product Cost Estimation . . . . . . . . . .
\end{flushleft}


\begin{flushleft}
Cash Flow from Plant production . . . . . . . . .
\end{flushleft}


\begin{flushleft}
Plant Economics Comparison with recently setup
\end{flushleft}





.


.


.


.


.


.


.





.


.


.


.


.


.


.





.


.


.


.


.


.


.





.


.


.


.


.


.


.





.


.


.


.


.


.


.





.


.


.


.


.


.


.





.


.


.


.


.


.


.





.


.


.


.


.


.


.





.


.


.


.


.


.


.





.


.


.


.


.


.


.





57


58


59


59


60


61


62





\begin{flushleft}
ix
\end{flushleft}





. . . . .


. . . . .


\begin{flushleft}
Balance
\end{flushleft}


. . . . .





.


.


.


.





.


.


.


.





.


.


.


.





. . . .


. . . .


. . . .


. . . .


. . . .


. . . .


\begin{flushleft}
plant
\end{flushleft}





.


.


.


.





.


.


.


.





.


.


.


.





.


.


.


.





.


.


.


.





.


.


.


.





.


.


.


.





.


.


.


.





.


.


.


.





.


.


.


.





31


32


33


34





\newpage
11.1


11.2


11.3


11.4





\begin{flushleft}
Mathematical Model Results
\end{flushleft}


\begin{flushleft}
Tray Selection Parameters . .
\end{flushleft}


\begin{flushleft}
Final Tray Specifications . . .
\end{flushleft}


\begin{flushleft}
Final Column Specifications .
\end{flushleft}





.


.


.


.





.


.


.


.





.


.


.


.





.


.


.


.





.


.


.


.





.


.


.


.





.


.


.


.





.


.


.


.





.


.


.


.





.


.


.


.





.


.


.


.





.


.


.


.





.


.


.


.





.


.


.


.





.


.


.


.





.


.


.


.





.


.


.


.





.


.


.


.





.


.


.


.





.


.


.


.





.


.


.


.





.


.


.


.





.


.


.


.





.


.


.


.





.


.


.


.





65


67


69


72





\begin{flushleft}
12.1 Composition of Tail Gas leaving Absorption Column . . . . . . . . . . . . 73
\end{flushleft}





\begin{flushleft}
x
\end{flushleft}





\begin{flushleft}
\newpage
Chapter 1
\end{flushleft}





\begin{flushleft}
Nitric Acid
\end{flushleft}


\begin{flushleft}
Nitric acid (HNO3 ) is a highly corrosive mineral acid and powerful oxidizing agent. It
\end{flushleft}


\begin{flushleft}
occurs in nature only in the form of nitrate salts. The pure compound is a colorless
\end{flushleft}


\begin{flushleft}
liquid, but tends to acquire a yellow cast, if stored for long, due to the accumulation of
\end{flushleft}


\begin{flushleft}
oxides of nitrogen. Most commercially available nitric acid has a concentration of 5570\%, being centrally used in the production of nitrogen fertilisers, nylon and explosives.
\end{flushleft}





1.1





\begin{flushleft}
Properties
\end{flushleft}


\begin{flushleft}
Property Name
\end{flushleft}





\begin{flushleft}
Property Value
\end{flushleft}





\begin{flushleft}
Molecular Weight
\end{flushleft}





\begin{flushleft}
63.012 g/mol
\end{flushleft}





\begin{flushleft}
Colour
\end{flushleft}





\begin{flushleft}
Colourless
\end{flushleft}





\begin{flushleft}
Boiling point
\end{flushleft}





\begin{flushleft}
83◦ C
\end{flushleft}





\begin{flushleft}
Density
\end{flushleft}





\begin{flushleft}
1.512 g/cm3
\end{flushleft}





\begin{flushleft}
Azeotropic composition
\end{flushleft}





\begin{flushleft}
68\% w/w (BP = 120.5◦ C)
\end{flushleft}





\begin{flushleft}
Thermal conductivity (20◦ C)
\end{flushleft}





\begin{flushleft}
0.343 W/m.K
\end{flushleft}





\begin{flushleft}
Standard enthalpy of formation
\end{flushleft}





\begin{flushleft}
2.7474 J/g (Liquid); 2.1258 J/g (Gas)
\end{flushleft}





\begin{flushleft}
Heat of vaporization (20◦ C)
\end{flushleft}





\begin{flushleft}
626.3 J/g
\end{flushleft}





\begin{flushleft}
Specific heat (20◦ C)
\end{flushleft}





\begin{flushleft}
1.7481 J/g.K
\end{flushleft}





\begin{flushleft}
Table 1.1: Nitric Acid Properties
\end{flushleft}





1





\begin{flushleft}
\newpage
Nitric Acid
\end{flushleft}





2





\begin{flushleft}
The tabulated data above lists some of the chemical and physical properties of nitric
\end{flushleft}


\begin{flushleft}
acid, acquired from PubChem1 . Nitric acid is a nitrogen oxoacid, with a nitrogen atom
\end{flushleft}


\begin{flushleft}
bonded to a hydroxyl group and by equivalent resonating double bonds to the remaining
\end{flushleft}


\begin{flushleft}
two oxygen atoms, giving the N-atom a +5 oxidation state. This can be seen in the
\end{flushleft}


\begin{flushleft}
HNO3 structure shown below :
\end{flushleft}


\begin{flushleft}
H
\end{flushleft}


\begin{flushleft}
O
\end{flushleft}





\begin{flushleft}
O
\end{flushleft}


\begin{flushleft}
N+
\end{flushleft}


\begin{flushleft}
O$-$
\end{flushleft}





1.2





\begin{flushleft}
Uses of Nitric Acid
\end{flushleft}





\begin{flushleft}
Figure 1.1: Uses of Nitric Acid, recreated from Thyssenkrupp Nitric Acid Book [1]
\end{flushleft}





\begin{flushleft}
Nitric acid is widely used as an intermediate and its demand majorly depends on the
\end{flushleft}


\begin{flushleft}
demand of the end product for which it is used as a raw material. About 90\% of nitric
\end{flushleft}


\begin{flushleft}
acid is used for on-site consumption and only 10\% accounts for retail market. Nitric
\end{flushleft}


\begin{flushleft}
Acid is used for following purposes :
\end{flushleft}


1


\begin{flushleft}
https://pubchem.ncbi.nlm.nih.gov/compound/nitric\_acid\#section=
\end{flushleft}


\begin{flushleft}
Chemical-and-Physical-Properties
\end{flushleft}





\begin{flushleft}
\newpage
Nitric Acid
\end{flushleft}





3





\begin{flushleft}
$\bullet$ About 80\% of the world production is used as raw material in the production of
\end{flushleft}


\begin{flushleft}
Ammonium Nitrate in the fertiliser industry
\end{flushleft}


\begin{flushleft}
$\bullet$ Also used as a raw material in the production of Adipic acid, which is further employed in manufacturing Nylon-6,6 resins and fibres - widely used in the automotive
\end{flushleft}


\begin{flushleft}
industry
\end{flushleft}


\begin{flushleft}
$\bullet$ Also used as a raw material for the production of Nitrobenzene - a precursor to
\end{flushleft}


\begin{flushleft}
Aniline, which finds widespread use in the pharmaceutical industry
\end{flushleft}


\begin{flushleft}
$\bullet$ Used in manufacturing explosives in mining/construction industry
\end{flushleft}


\begin{flushleft}
$\bullet$ Used in manufacturing Polyurethane and Polyamide and also as an oxidiser in
\end{flushleft}


\begin{flushleft}
liquid fueled rockets
\end{flushleft}





1.3





\begin{flushleft}
Uses of Intermediates and Byproducts
\end{flushleft}





\begin{flushleft}
$\bullet$ NO - Nitric oxide is used together with a breathing machine (ventilator) to treat
\end{flushleft}


\begin{flushleft}
respiratory failure in premature babies. In the human body, nitric oxide expands
\end{flushleft}


\begin{flushleft}
the blood vessels, increasing blood flow and decreasing plaque growth and blood
\end{flushleft}


\begin{flushleft}
clotting.
\end{flushleft}


\begin{flushleft}
$\bullet$ NO2 - Nitrogen Dioxide is used as an oxidising agent in certain oxidation reactions,
\end{flushleft}


\begin{flushleft}
as an inhibitor to prevent polymerization of acrylates during distillation, as a
\end{flushleft}


\begin{flushleft}
nitrating agent for organic compounds, as a rocket fuel, as a flour bleaching agent
\end{flushleft}


\begin{flushleft}
and in increasing the wet strength of paper.
\end{flushleft}


\begin{flushleft}
$\bullet$ N2 O - Nitrous oxide has significant medical uses, especially in surgery and dentistry, for its anaesthetic and pain reducing effects. Its name ''laughing gas'' is due
\end{flushleft}


\begin{flushleft}
to the euphoric effects upon inhaling it, a property that has led to its recreational
\end{flushleft}


\begin{flushleft}
use as a dissociative anaesthetic.
\end{flushleft}


\begin{flushleft}
Even though these byproducts find significant uses in multiple sectors of industry, they
\end{flushleft}


\begin{flushleft}
are major pollutants when produced in large amounts, with NO and NO2 being the
\end{flushleft}


\begin{flushleft}
main causes for acid rains and air pollution and N2 O being a major greenhouse gas,
\end{flushleft}


\begin{flushleft}
contributing to the depletion of the Ozone layer and trapping of heat within the atmosphere, and hence, require to be eliminated as per Government of India norms before
\end{flushleft}


\begin{flushleft}
being expanded into the atmosphere.
\end{flushleft}





\begin{flushleft}
\newpage
Nitric Acid
\end{flushleft}





1.4





4





\begin{flushleft}
Grades of Nitric Acid
\end{flushleft}





\begin{flushleft}
There are two major commercially available grades of Nitric Acid :
\end{flushleft}





\begin{flushleft}
$\bullet$ Weak Nitric Acid (WNA) - Nitric acid having a concentration of 55-70\%
\end{flushleft}


\begin{flushleft}
wt.HNO3 . WNA is widely used in the fertiliser industry as the starting raw material for Ammonium Nitrate.
\end{flushleft}


\begin{flushleft}
$\bullet$ Concentrated Nitric Acid (CNA) - Nitric acid having a concentration of
\end{flushleft}


\begin{flushleft}
$>$90\% wt.HNO3 . CNA is widely used as a nitration agent in manufacturing explosives, nitroalkanes and nitroaromatics and also as an oxidising agent in multiple
\end{flushleft}


\begin{flushleft}
industries and laboratories.
\end{flushleft}





\begin{flushleft}
\newpage
Chapter 2
\end{flushleft}





\begin{flushleft}
Market Survey
\end{flushleft}


\begin{flushleft}
Due to the rapidly increasing population and reduced availability of land for irrigation,
\end{flushleft}


\begin{flushleft}
there is an increasing demand for fertilisers. The use of nitric acid as an intermediate
\end{flushleft}


\begin{flushleft}
in the production of ammonium nitrate is a key factor, boosting the production as well
\end{flushleft}


\begin{flushleft}
as the consumption of HNO3 . Growing automotive industry drives concentrated nitric
\end{flushleft}


\begin{flushleft}
acid market as it is utilized in manufacturing of light weight and strong body parts of
\end{flushleft}


\begin{flushleft}
vehicles. Rising demand for synthetic rubbers, elastomers and polyurethane foams in the
\end{flushleft}


\begin{flushleft}
automotive industry is anticipated to drive the product demand as well. All the findings
\end{flushleft}


\begin{flushleft}
of this section were made after referring Nitric Acid Market Research reports provided
\end{flushleft}


\begin{flushleft}
by Research and Markets[2], Grand View Research[3] and IHS Chemical Economics
\end{flushleft}


\begin{flushleft}
Handbook[4].
\end{flushleft}





2.1





\begin{flushleft}
World Consumption Patterns
\end{flushleft}





\begin{flushleft}
The largest market for nitric acid is the production of ammonium nitrate (AN) and
\end{flushleft}


\begin{flushleft}
calcium ammonium nitrate (CAN), accounting for almost 77\% of the total world consumption of nitric acid in 2016. World nitric acid consumption has exhibited a steady
\end{flushleft}


\begin{flushleft}
upward trend since 2000, with average annual growth rates of 2.0\% during 2000-17 [4].
\end{flushleft}


\begin{flushleft}
Europe, United States and China account for more than 80\% of global demand. Since
\end{flushleft}


\begin{flushleft}
2000, the largest increase in consumption has occurred in China, at an average annual
\end{flushleft}


\begin{flushleft}
growth rate of 7.3\%.
\end{flushleft}





5





\begin{flushleft}
\newpage
Market Survey
\end{flushleft}





6





\begin{flushleft}
Figure 2.1: World Nitric Acid Consumption 2017, recreated from [4]
\end{flushleft}





\begin{flushleft}
The presence of large agricultural and industrial market in Asia-Pacific is a key factor
\end{flushleft}


\begin{flushleft}
responsible for the high growth of the global nitric acid market. Also, the stringent
\end{flushleft}


\begin{flushleft}
regulations in regions like North America and Europe regarding the use of ammonium
\end{flushleft}


\begin{flushleft}
nitrate based fertilisers are likely to shift the demand to countries in the Asia-Pacific.
\end{flushleft}





2.2





\begin{flushleft}
World Production Patterns
\end{flushleft}


\begin{flushleft}
Year
\end{flushleft}





\begin{flushleft}
Nitric Acid Produced (tons)
\end{flushleft}





\begin{flushleft}
Increase (tons)
\end{flushleft}





2010





7,419,644





------





2011





7,642,233





222,589





2012





7,871,500





229,267





2013





8,107,645





236,145





2014





8,350,874





243,229





2015





8,601,401





250,526





2016





8,859,443





258,042





2017





9,125,226





265,783





\begin{flushleft}
Table 2.1: World Nitric Acid Production
\end{flushleft}





\begin{flushleft}
\newpage
Market Survey
\end{flushleft}





7





\begin{flushleft}
Figure 2.2: World Nitric Acid Production 2017, recreated from [2]
\end{flushleft}





\begin{flushleft}
Europe and North America account for more than 80\% of the global production. In
\end{flushleft}


\begin{flushleft}
North America, the United States dominated the market with a revenue share of 70\%
\end{flushleft}


\begin{flushleft}
in 2017 and it is estimated to continue the trend over the forecast period of 2018-2025.
\end{flushleft}


\begin{flushleft}
Robust manufacturing base of chemical, electronics, and automotive industries in the
\end{flushleft}


\begin{flushleft}
U.S. is anticipated to supplement the growth of the market in the region.
\end{flushleft}





\begin{flushleft}
In Western Europe, Germany represented 15\% of the overall revenue in 2017. Leading chemical companies such as BASF, Bayer and Henkel have their established manufacturing units intended for the production of application-specific bulk chemicals and
\end{flushleft}


\begin{flushleft}
speciality polymers in Germany. Aforementioned factors are influencing the growth of
\end{flushleft}


\begin{flushleft}
the market in Western Europe positively. Presence of advanced fertiliser manufacturing
\end{flushleft}


\begin{flushleft}
facilities in Russia along with abundant availability of raw materials such as ammonia
\end{flushleft}


\begin{flushleft}
is projected to be a favorable factor for the market [2].
\end{flushleft}





\begin{flushleft}
Ammonia is the key raw material used in the manufacturing of nitric acid. Ammonia production is concentrated mainly in the U.S., China, Eastern Europe and Middle
\end{flushleft}


\begin{flushleft}
East because of the availability of large natural gas reserves in these regions.
\end{flushleft}





\begin{flushleft}
\newpage
Market Survey
\end{flushleft}





2.3





8





\begin{flushleft}
Top Market Producers
\end{flushleft}





\begin{flushleft}
Figure 2.3: Nitric Acid Top Market Producers, recreated with data from [2]
\end{flushleft}


\begin{flushleft}
Table 2.2: Major Producers in Global Market
\end{flushleft}





\begin{flushleft}
BASF SE
\end{flushleft}





\begin{flushleft}
Germany
\end{flushleft}





\begin{flushleft}
Shandong Fengyang
\end{flushleft}





\begin{flushleft}
China
\end{flushleft}





\begin{flushleft}
Agrium Inc.
\end{flushleft}





\begin{flushleft}
Canada
\end{flushleft}





\begin{flushleft}
Apache Nitrogen Products, Inc.
\end{flushleft}





\begin{flushleft}
USA
\end{flushleft}





\begin{flushleft}
Yara International
\end{flushleft}





\begin{flushleft}
Norway
\end{flushleft}





\begin{flushleft}
OCI NV
\end{flushleft}





\begin{flushleft}
Netherlands
\end{flushleft}





\begin{flushleft}
CF Industries Holdings, Inc.
\end{flushleft}





\begin{flushleft}
USA
\end{flushleft}





\begin{flushleft}
Dow Chemicals
\end{flushleft}





\begin{flushleft}
USA
\end{flushleft}





\begin{flushleft}
Potash Corp of Saskatchewan
\end{flushleft}





\begin{flushleft}
Canada
\end{flushleft}





\begin{flushleft}
LSB Industries Inc.
\end{flushleft}





\begin{flushleft}
USA
\end{flushleft}





\begin{flushleft}
Table 2.3: Major Producers in Domestic Market
\end{flushleft}





\begin{flushleft}
Gujarat Narmada Valley fertilisers
\end{flushleft}





\begin{flushleft}
Bharuch, Gujarat
\end{flushleft}





\begin{flushleft}
Company Ltd.
\end{flushleft}


\begin{flushleft}
Vijay Gas Industry Pvt. Ltd
\end{flushleft}





\begin{flushleft}
Mumbai, Maharashtra
\end{flushleft}





\begin{flushleft}
Rashtriya Chemicals and Fertilisers
\end{flushleft}





\begin{flushleft}
Raigad \& Mumbai, Maharashtra
\end{flushleft}





\begin{flushleft}
Ltd.
\end{flushleft}


\begin{flushleft}
Deepak Fertilisers and Petrochem-
\end{flushleft}





\begin{flushleft}
Taloja, Maharashtra; Srikakulam,
\end{flushleft}





\begin{flushleft}
icals Corporation Ltd.
\end{flushleft}





\begin{flushleft}
A.P.; Panipat, Haryana; Dahej,
\end{flushleft}


\begin{flushleft}
Gujarat
\end{flushleft}





\begin{flushleft}
Surya Fine Chemicals
\end{flushleft}





\begin{flushleft}
Boisar \& Palghar, Maharashtra
\end{flushleft}





\begin{flushleft}
\newpage
Market Survey
\end{flushleft}





2.4





9





\begin{flushleft}
Technology Providers
\end{flushleft}





\begin{flushleft}
$\bullet$ Weatherly, a company of Chematur Engineering Group of Sweden
\end{flushleft}


\begin{flushleft}
$\bullet$ Espindesa, a company of Técnicas Reunidas of Spain
\end{flushleft}


\begin{flushleft}
$\bullet$ Borealis of Austria
\end{flushleft}


\begin{flushleft}
$\bullet$ Uhde GmbH (now ThyssenKrupp Industrial Solutions) of Germany
\end{flushleft}


\begin{flushleft}
$\bullet$ MECS Technology of USA
\end{flushleft}


\begin{flushleft}
$\bullet$ KBR of USA
\end{flushleft}


\begin{flushleft}
$\bullet$ Technip of France
\end{flushleft}





2.5





\begin{flushleft}
Location Selection for New Plant
\end{flushleft}





\begin{flushleft}
Option 1 : Ukraine
\end{flushleft}


\begin{flushleft}
$\bullet$ Cheaper raw material availability (Ammonia - \$250-\$300/ton)
\end{flushleft}


\begin{flushleft}
$\bullet$ High demand for Nitric acid in neighboring countries (Nitric Acid - \$300-\$400/ton)
\end{flushleft}


\begin{flushleft}
$\bullet$ Ease in exporting products and materials (Bordering Black Sea and close to multiple fertiliser conglomerates)
\end{flushleft}


\begin{flushleft}
$\bullet$ Europe being the largest consumer of Nitric Acid provides proximity to desired
\end{flushleft}


\begin{flushleft}
market
\end{flushleft}





\begin{flushleft}
Option 2 : India
\end{flushleft}


\begin{flushleft}
$\bullet$ Cheaper Labour costs
\end{flushleft}


\begin{flushleft}
$\bullet$ Vast import and export opportunities(surrounded by highly populated nations)
\end{flushleft}


\begin{flushleft}
$\bullet$ High local demand
\end{flushleft}


\begin{flushleft}
$\bullet$ Large agricultural markets of South East Asia provide huge selling outlets for nitric
\end{flushleft}


\begin{flushleft}
acid products like Ammonium Nitrate fertilisers
\end{flushleft}


\begin{flushleft}
$\bullet$ Stringent regulations on Nitrate fertilisers in North America and Europe can cause
\end{flushleft}


\begin{flushleft}
a shift in the production and selling of these product in and around the Asia Pacific
\end{flushleft}





\begin{flushleft}
\newpage
Market Survey
\end{flushleft}





10





\begin{flushleft}
Reasons for selecting Hazira, Gujarat :
\end{flushleft}


\begin{flushleft}
$\bullet$ Raw Material Advantage : close to largest Ammonia plant in India - KRIBHCO
\end{flushleft}


\begin{flushleft}
$\bullet$ Proximity to the Hazira Port for importing the raw material and exporting the
\end{flushleft}


\begin{flushleft}
final product
\end{flushleft}


\begin{flushleft}
$\bullet$ Well established industrial area with great connectivity to the rest of world
\end{flushleft}





\begin{flushleft}
Based on the above mentioned factors, the final location for the Nitric Acid plant was
\end{flushleft}


\begin{flushleft}
selected to be Hazira, Gujarat, India
\end{flushleft}





\begin{flushleft}
\newpage
Chapter 3
\end{flushleft}





\begin{flushleft}
Production Process
\end{flushleft}


\begin{flushleft}
This chapter provides details about various production methods for nitric acid, from
\end{flushleft}


\begin{flushleft}
laboratory methods to the widely used commercial processes, including their advantages
\end{flushleft}


\begin{flushleft}
and disadvantages, and finally concluding with determining the right process for the
\end{flushleft}


\begin{flushleft}
plant and its capacity.
\end{flushleft}





3.1





\begin{flushleft}
Production Routes
\end{flushleft}





\begin{flushleft}
$\bullet$ From Chile Saltpeter (NaNO3 ) :
\end{flushleft}


\begin{flushleft}
N aN O3 + H2 SO4 $\rightarrow$ N aHSO4 + HN O3
\end{flushleft}


\begin{flushleft}
$\bullet$ From Electric Oxidation of Air :
\end{flushleft}


\begin{flushleft}
N2 + O2 $\rightarrow$ 2N O
\end{flushleft}


\begin{flushleft}
2N O + O2 $\rightarrow$ 2N O2
\end{flushleft}


\begin{flushleft}
3N O2 + H2 O $\rightarrow$ 2HN O3 + N O
\end{flushleft}


\begin{flushleft}
$\bullet$ From thermal decomposition of Copper(II) Nitrate :
\end{flushleft}


\begin{flushleft}
2Cu(N O3 )2 (s) $\rightarrow$ 2CuO(s) + 4N O2 (g) + O2 (g)
\end{flushleft}





\begin{flushleft}
All these processes have become obsolete and the only industry used process is the
\end{flushleft}


\begin{flushleft}
Catalytic Oxidation of Ammonia by the Ostwald Process and its variants.
\end{flushleft}





11





\begin{flushleft}
\newpage
Production Process
\end{flushleft}





3.2





12





\begin{flushleft}
Ostwald Process
\end{flushleft}





\begin{flushleft}
Referring to Moulijn et. al. [5], the Ostwald Process involves three steps - (i) Ammonia
\end{flushleft}


\begin{flushleft}
Oxidation (ii) Nitric Oxide Oxidation (iii)Nitrogen Dioxide Oxidation
\end{flushleft}





3.2.1





\begin{flushleft}
Ammonia Oxidation
\end{flushleft}


\begin{flushleft}
4N H3 (g) + 5O2 (g) $\rightarrow$ 4N O(g) + 6H2 O(g), ∆H = $-$905kJ/mol
\end{flushleft}





\begin{flushleft}
A 1:9 ammonia/air mixture by volume is reacted at a temperature of 750◦ C to 900◦ C
\end{flushleft}


\begin{flushleft}
as it passes through a catalytic convertor.
\end{flushleft}


\begin{flushleft}
Catalyst: 90\% Platinum - 10\% Rhodium gauze constructed from squares of fine wire.
\end{flushleft}


\begin{flushleft}
Yield: Under these conditions the oxidation of ammonia to nitric oxide (NO) proceeds
\end{flushleft}


\begin{flushleft}
in an exothermic reaction with a range of 93 to 98 percent yield.
\end{flushleft}


\begin{flushleft}
Side Reactions:
\end{flushleft}


\begin{flushleft}
4N H3 (g) + 4O2 (g) $\rightarrow$ 2N2 O(g) + 6H2 O(g), ∆H = $-$1102kJ/mol
\end{flushleft}


\begin{flushleft}
4N H3 (g) + 3O2 (g) $\rightarrow$ 2N2 (g) + 6H2 O(g), ∆H = $-$1261kJ/mol
\end{flushleft}





\begin{flushleft}
Oxidation temperatures can vary from 750◦ C to 900◦ C. Higher catalyst temperatures
\end{flushleft}


\begin{flushleft}
increase reaction selectivity toward NO production. Lower catalyst temperatures tend
\end{flushleft}


\begin{flushleft}
to be more selective towards less useful products: nitrogen (N2 ) and nitrous oxide (N2 O).
\end{flushleft}


\begin{flushleft}
Nitric oxide is to be used further in our process, whereas nitrous oxide is known to be a
\end{flushleft}


\begin{flushleft}
greenhouse gas.
\end{flushleft}


\begin{flushleft}
High temperature and low pressure make the reaction more selective towards desired product (NO gas); also owing to the high spontaneity of the reactions, catalytic
\end{flushleft}


\begin{flushleft}
bed reactor of residence time of 10$-$4 to 10$-$3 seconds is used.
\end{flushleft}





3.2.2





\begin{flushleft}
Nitric Oxide Oxidation
\end{flushleft}





\begin{flushleft}
The process stream is passed through a series of heat exchangers and cooler-condensers
\end{flushleft}


\begin{flushleft}
and cooled to 50◦ C or less at pressures up to 4 barg. The nitric oxide reacts non
\end{flushleft}


\begin{flushleft}
catalytically with residual oxygen to form nitrogen dioxide (NO2 )
\end{flushleft}


\begin{flushleft}
2N O(g) + O2 (g) $\rightarrow$ 2N O2 (g), ∆H = $-$114kJ/mol
\end{flushleft}





\begin{flushleft}
\newpage
Production Process
\end{flushleft}





13





\begin{flushleft}
This slow, homogeneous reaction is highly temperature and pressure dependent. Operating at low temperatures and high pressures promotes maximum production of NO2
\end{flushleft}


\begin{flushleft}
with a reduced reaction time.
\end{flushleft}





3.2.3





\begin{flushleft}
Absorption
\end{flushleft}





\begin{flushleft}
The oxidised gas mixture is fed into the bottom of the absorption tower, while a liquid
\end{flushleft}


\begin{flushleft}
acid condensate, produced during nitric oxide oxidation, is added at a higher point.
\end{flushleft}


\begin{flushleft}
Deionized process water enters the top of the column. Both liquids flow counter-current
\end{flushleft}


\begin{flushleft}
to the NOX gas mixture. Oxidation takes place in the free space between the trays,
\end{flushleft}


\begin{flushleft}
while absorption occurs on the trays. The absorption trays used are sieve trays.
\end{flushleft}


\begin{flushleft}
3N O2 (g) + H2 O(l) $\rightarrow$ 2HN O3 (aq.) + N O(g), ∆H = $-$75kJ/mol
\end{flushleft}





\begin{flushleft}
A secondary air stream is introduced into the column to re-oxidise the NO that is
\end{flushleft}


\begin{flushleft}
formed in this reaction. This secondary air also removes NOX from the product acid.
\end{flushleft}


\begin{flushleft}
An aqueous solution of 55 to 65 percent (typically) nitric acid is withdrawn from the
\end{flushleft}


\begin{flushleft}
bottom of the tower. The acid concentration depends upon the temperature, pressure,
\end{flushleft}


\begin{flushleft}
number of absorption stages, and concentration of nitrogen oxides entering the absorber.
\end{flushleft}





\begin{flushleft}
Figure 3.1: Block Diagram of Ostwald Process
\end{flushleft}





\begin{flushleft}
\newpage
Production Process
\end{flushleft}





3.3





14





\begin{flushleft}
Deviations in Ostwald Process
\end{flushleft}





\begin{flushleft}
Referring to Moulijn et. al. [5], we found out that the principal design variables for the
\end{flushleft}


\begin{flushleft}
ammonia oxidation and subsequent stages of the plant are temperature, pressure and
\end{flushleft}


\begin{flushleft}
gas flow rate, and their effects on the process efficiency is tabulated below:
\end{flushleft}





\begin{flushleft}
Process Change
\end{flushleft}





\begin{flushleft}
NH3 Oxidation
\end{flushleft}





\begin{flushleft}
NO Oxidation
\end{flushleft}





\begin{flushleft}
Absorption
\end{flushleft}





\begin{flushleft}
Temperature Increase
\end{flushleft}





\begin{flushleft}
Higher Yield
\end{flushleft}





\begin{flushleft}
Lower Yield
\end{flushleft}





\begin{flushleft}
Reduced absorption
\end{flushleft}





\begin{flushleft}
Pressure Increase
\end{flushleft}





\begin{flushleft}
Oxidation
\end{flushleft}


\begin{flushleft}
duced.
\end{flushleft}





\begin{flushleft}
rate
\end{flushleft}





\begin{flushleft}
re-
\end{flushleft}





\begin{flushleft}
Higher Yield
\end{flushleft}





\begin{flushleft}
Amount of
\end{flushleft}





\begin{flushleft}
Improved absorption
\end{flushleft}





\begin{flushleft}
NH3 oxidised per unit
\end{flushleft}


\begin{flushleft}
time increases
\end{flushleft}


\begin{flushleft}
Flow Rate Increase
\end{flushleft}





\begin{flushleft}
Optimum temp.
\end{flushleft}





\begin{flushleft}
in-
\end{flushleft}





\begin{flushleft}
creases; Higher Yield
\end{flushleft}





\begin{flushleft}
Higher yield
\end{flushleft}





\begin{flushleft}
Minor
\end{flushleft}





\begin{flushleft}
improve-
\end{flushleft}





\begin{flushleft}
ments
\end{flushleft}





\begin{flushleft}
Table 3.1: Temperature and Pressure Dependence of Ostwald Process
\end{flushleft}





\begin{flushleft}
According to Thiemann et. al. [6] and Thyssenkrupp Nitric Acid Book [1], classification
\end{flushleft}


\begin{flushleft}
of plants is done on the basis of pressure in the oxidation and absorption stages:
\end{flushleft}





3.3.1


3.3.1.1





\begin{flushleft}
Single Pressure Processes
\end{flushleft}


\begin{flushleft}
Medium Pressure
\end{flushleft}





\begin{flushleft}
This type of nitric acid plant operates at around 4--5 bars abs. It can produce up to
\end{flushleft}


\begin{flushleft}
700 tonnes/day using a single combustion and a single absorption tower and around
\end{flushleft}


\begin{flushleft}
1000 tonnes/day by adding another absorption tower. Using this pressure enhances the
\end{flushleft}


\begin{flushleft}
oxidation step (NO to NO2 ), but reduces the productivity of the final absorption stage.
\end{flushleft}





3.3.1.2





\begin{flushleft}
High Pressure
\end{flushleft}





\begin{flushleft}
Pressure of around 8--12 bars abs is used. This high pressure forces the nitrogen dioxide
\end{flushleft}


\begin{flushleft}
gas into water. And thus, a single absorption tower is sufficient to produce 900 tonnes/day of nitric acid. The higher pressure means that other equipment like the burner unit,
\end{flushleft}





\begin{flushleft}
\newpage
Production Process
\end{flushleft}





15





\begin{flushleft}
piping can be made smaller so the plant occupies less space, making the initial capital
\end{flushleft}


\begin{flushleft}
investment less.
\end{flushleft}





3.3.2





\begin{flushleft}
Dual Pressure Process
\end{flushleft}





\begin{flushleft}
$\bullet$ Low to Medium dual pressure: 1--2/ 4--5 bars abs
\end{flushleft}


\begin{flushleft}
$\bullet$ Medium to High dual pressure: 4--5/ 8--12 bars abs
\end{flushleft}





\begin{flushleft}
Lower pressure for ammonia oxidation and higher pressure for absorption is favourable.
\end{flushleft}


\begin{flushleft}
The dual pressure plant uses two different pressures, a lower pressure for the oxidation
\end{flushleft}


\begin{flushleft}
step and higher pressure to optimize the absorption. This type of plant can produce up
\end{flushleft}


\begin{flushleft}
to 1600 tonnes/day of nitric acid using single combustion and absorption unit. Lower
\end{flushleft}


\begin{flushleft}
pressure for the oxidation unit also reduces the catalyst cost.
\end{flushleft}





3.4





\begin{flushleft}
Selection of Final Process
\end{flushleft}





\begin{flushleft}
The points considered while finalising the actual process to be utilised in the plant are:
\end{flushleft}





\begin{flushleft}
$\bullet$ Variable Costs
\end{flushleft}


\begin{flushleft}
$\bullet$ Capacity
\end{flushleft}


\begin{flushleft}
$\bullet$ Catalyst Life
\end{flushleft}


\begin{flushleft}
$\bullet$ Catalyst Performance and Loss
\end{flushleft}


\begin{flushleft}
$\bullet$ Environmental Impact
\end{flushleft}


\begin{flushleft}
$\bullet$ Absorption Column Efficiency
\end{flushleft}


\begin{flushleft}
$\bullet$ Desired Nitric Acid Concentration
\end{flushleft}


\begin{flushleft}
$\bullet$ Electric Power
\end{flushleft}


\begin{flushleft}
In accordance with the above mentioned points, the following comparisons1 are documented and are also an industry-wide practice:
\end{flushleft}


1





\begin{flushleft}
All comparisons are per tonne Nitric Acid (100\%)
\end{flushleft}





\begin{flushleft}
\newpage
Production Process
\end{flushleft}





16





\begin{flushleft}
$\bullet$ The dual pressure process provides larger capacities (upto 1600 tpd in single train
\end{flushleft}


\begin{flushleft}
configuration) with relatively lower variable costs, when compared to the mono
\end{flushleft}


\begin{flushleft}
pressure processes (upto 900 tpd for high pressure, upto 700 tpd for medium
\end{flushleft}


\begin{flushleft}
pressure - in single train configuration), which require larger variable costs.
\end{flushleft}


\begin{flushleft}
$\bullet$ The dual pressure process is designed to accommodate more stringent environmental pollution control requirements, namely to reduce the emissions of NOX gases
\end{flushleft}


\begin{flushleft}
into the atmosphere. The process has inherent abilities to reduce the outgoing tail
\end{flushleft}


\begin{flushleft}
gas to a composition well below the required environmental norms. This reduction
\end{flushleft}


\begin{flushleft}
in composition is a result of an increased absorption column efficiency due to the
\end{flushleft}


\begin{flushleft}
increased absorption column pressure incorporated in the dual pressure process.
\end{flushleft}


\begin{flushleft}
$\bullet$ Due to a lower load on the burner, the dual pressure process also provides with a
\end{flushleft}


\begin{flushleft}
longer operation time for a given bundle of catalyst gauzes($>$ 6--8 months), when
\end{flushleft}


\begin{flushleft}
compared to the other available processes(upto 6 months). This effect can be
\end{flushleft}


\begin{flushleft}
seen directly on the total catalyst losses for dual pressure process (0.03g/t of nitric
\end{flushleft}


\begin{flushleft}
acid produced) when compared to the mono pressure processes (0.04g/t nitric acid
\end{flushleft}


\begin{flushleft}
produced for medium pressure process and 0.08g/t nitric acid produced for high
\end{flushleft}


\begin{flushleft}
pressure processes respectively)
\end{flushleft}


\begin{flushleft}
$\bullet$ All processes can easily provide the required nitric acid concentration ($\approx$ 60\% wt)
\end{flushleft}


\begin{flushleft}
$\bullet$ When comparing all the alternatives, the dual pressure process requires the least
\end{flushleft}


\begin{flushleft}
amount of electric power per ton nitric acid produced (8.5 kWh), followed by
\end{flushleft}


\begin{flushleft}
medium pressure (9.0 kWh) and high pressure (13.5 kWh)
\end{flushleft}





\begin{flushleft}
Capacity of Proposed plant:
\end{flushleft}


\begin{flushleft}
With reference to average plant capacities using dual pressure process, keeping in mind
\end{flushleft}


\begin{flushleft}
the demand for nitric acid, which is growing at a rate of 3-5\% annually, and the incremental production of the plant i.e. 70\% in the first year, 80\% in the second year and
\end{flushleft}


\begin{flushleft}
so on, the final plant capacity has been decided to be kept at 1500 TPD of 62\% wt
\end{flushleft}


\begin{flushleft}
nitric acid (942 TPD in terms of 100\% wt nitric acid)
\end{flushleft}


\begin{flushleft}
Keeping in mind the desired production from our plant, the final route for the production of nitric was chosen to be the Dual Pressure Process, with medium pressure
\end{flushleft}


\begin{flushleft}
ammonia oxidation over Pt--Rh catalyst and high pressure absorption operations.
\end{flushleft}





\begin{flushleft}
\newpage
Production Process
\end{flushleft}





17





\begin{flushleft}
Licensor - Uhde GmbH, Germany
\end{flushleft}


\begin{flushleft}
Existing plants using this route:
\end{flushleft}





\begin{flushleft}
$\bullet$ Abu Qir Fertilisers and Chemical Ind. Co., Abu Qir, Egypt - Capacity - 1850 TPD
\end{flushleft}


\begin{flushleft}
$\bullet$ Borealis AG, Linz, Austria - Capacity - 1000 TPD
\end{flushleft}





\begin{flushleft}
(As an industry standard, plant capacities are displayed in terms of 100\% wt nitric acid)
\end{flushleft}





\begin{flushleft}
Hence, our final plant details are as follows: 1500 TPD of 62\% wt (942 TPD in terms
\end{flushleft}


\begin{flushleft}
of 100\% wt nitric acid) nitric acid plant, using the Uhde Dual Pressure Process, set
\end{flushleft}


\begin{flushleft}
up in Hazira, Gujarat, India.
\end{flushleft}





\begin{flushleft}
\newpage
Chapter 4
\end{flushleft}





\begin{flushleft}
Industrial Visit
\end{flushleft}


\begin{flushleft}
In order to gain a better understanding of the selected process and to aid selection
\end{flushleft}


\begin{flushleft}
of parameters and equipment, a visit to Deepak Fertilizers and Petrochemical Corp.
\end{flushleft}


\begin{flushleft}
Ltd.(DFPCL) [7] was made on February 6, 2019.
\end{flushleft}


\begin{flushleft}
Total Nitric Acid Capacity: 1400TPD
\end{flushleft}


\begin{flushleft}
4 nitric acid plants:
\end{flushleft}


\begin{flushleft}
2 mono high pressure plants (300TPD each, set up in 1996)
\end{flushleft}


\begin{flushleft}
1 mono medium pressure plant (350TPD, set up in 2000)
\end{flushleft}


\begin{flushleft}
1 dual pressure plant (Uhde Dual Pressure Process, 450TPD, set up in 1975, Denmark,
\end{flushleft}


\begin{flushleft}
recommissioned in 2010, India)
\end{flushleft}





\begin{flushleft}
The weak nitric acid (59.2\% wt) produced here is used further for ammonium nitrate
\end{flushleft}


\begin{flushleft}
production and concentrated nitric acid (98.5\%wt) is sold separately. Since the dual
\end{flushleft}


\begin{flushleft}
pressure plant is recommissioned, it has a few added features to adjust to the temperature and pressure in India. For example, the compressors used are designed for a lower
\end{flushleft}


\begin{flushleft}
air temperature (Denmark) whereas the ambient air temperature in India is high. Hence,
\end{flushleft}


\begin{flushleft}
air chillers are added to reduce the temperature of the air to 20◦ C before compression.
\end{flushleft}





18





\begin{flushleft}
\newpage
Industrial Visit
\end{flushleft}





4.1


4.1.1





19





\begin{flushleft}
Process Highlights
\end{flushleft}


\begin{flushleft}
Ammonia
\end{flushleft}





\begin{flushleft}
Deepak Fertilisers has a separate plant for ammonia production which usually meets the
\end{flushleft}


\begin{flushleft}
nitric acid requirement. An alternate of sourcing ammonia externally during off days is
\end{flushleft}


\begin{flushleft}
also in place. The ammonia obtained is stored in a cylindrical storage refrigerated tank
\end{flushleft}


\begin{flushleft}
at -33◦ C and 1 atm pressure.
\end{flushleft}





4.1.2





\begin{flushleft}
Ammonia Filters
\end{flushleft}





\begin{flushleft}
Ceramic tubes are used to filter oil, water and other impurities from ammonia. The
\end{flushleft}


\begin{flushleft}
ammonia feed enters radially, is filtered and exits through the axis.
\end{flushleft}





\begin{flushleft}
Figure 4.1: Ammonia Flow through Ceramic Filters
\end{flushleft}





4.1.3





\begin{flushleft}
Air Filters
\end{flushleft}





\begin{flushleft}
The air is compressed to 5 bar abs using a centrifugal compressor. Primary and Secondary Filters are used to filter out particulate matter before compression. These filters
\end{flushleft}


\begin{flushleft}
are made of Stainless Steel and present in huge numbers, incorporated with the cooling
\end{flushleft}


\begin{flushleft}
network inside the filter rack.
\end{flushleft}





4.1.4





\begin{flushleft}
Ammonia - Air Mixer
\end{flushleft}





\begin{flushleft}
Mixer used in the industry is a Stainless Steel Static Mixer, where both the gas streams
\end{flushleft}


\begin{flushleft}
are intermixed, entering perpendicular to each other.
\end{flushleft}





\begin{flushleft}
\newpage
Industrial Visit
\end{flushleft}





4.1.5





20





\begin{flushleft}
Turbine Shaft
\end{flushleft}





\begin{flushleft}
A single steam turbine is used to rotate a shaft which is connected to a series of compressors and expanders. HP steam is used to rotate the turbine at 7000 RPM. The turbine
\end{flushleft}


\begin{flushleft}
shaft is used to rotate a centrifugal compressor, which compresses air from atmospheric
\end{flushleft}


\begin{flushleft}
pressure to 5 bar abs pressure. The second equipment on this shaft is an expander; it
\end{flushleft}


\begin{flushleft}
reduces the pressure of the tail gas from approximately 12 bar abs to 1 atm. A gear
\end{flushleft}


\begin{flushleft}
(1:14) is used after that to increase the rotations per minute for the NOX compressor.
\end{flushleft}


\begin{flushleft}
The energy from steam and tail gas is efficiently used to run the compressor shafts.
\end{flushleft}





\begin{flushleft}
Figure 4.2: Turbine Shaft Structure, as seen in the DFPCL Dual Pressure Plant
\end{flushleft}





4.1.6





\begin{flushleft}
Reactor
\end{flushleft}





\begin{flushleft}
A shallow bed reactor with 8 gauzes of 95-5 Pt-Rh is used. The reaction is highly
\end{flushleft}


\begin{flushleft}
exothermic and takes place at 890◦ C. For the start-up of the process, hydrogen gas
\end{flushleft}


\begin{flushleft}
is burned to raise the temperature of the gauzes (around 8 cylinders of hydrogen are
\end{flushleft}


\begin{flushleft}
kept in the inventory (4 for main use + 4 for extra backup)). After the reaction starts,
\end{flushleft}


\begin{flushleft}
the heat of the reaction is sufficient to maintain that temperature. 96\% of ammonia
\end{flushleft}


\begin{flushleft}
is converted in a single pass. Catchment gauzes of Palladium are provided to capture
\end{flushleft}


\begin{flushleft}
catalyst chipped off with outgoing stream.
\end{flushleft}





4.1.7





\begin{flushleft}
Boiler
\end{flushleft}





\begin{flushleft}
The product stream leaving from the reactor exits at 890◦ C. This steam is to be cooled
\end{flushleft}


\begin{flushleft}
before adding to the absorption column. And therefore, a series of heat exchangers and
\end{flushleft}


\begin{flushleft}
cooler condensers are added. A waste heat boiler is attached just below the reactor
\end{flushleft}


\begin{flushleft}
setup, with hot water flowing through the tube side. In the boiler, the temperature of
\end{flushleft}


\begin{flushleft}
the process stream is reduced from 890◦ C to 410◦ C. High pressure steam is produced
\end{flushleft}


\begin{flushleft}
while cooling the process stream, which is used to run the steam turbine.
\end{flushleft}





\begin{flushleft}
\newpage
Industrial Visit
\end{flushleft}





4.1.8





21





\begin{flushleft}
Cooler Condenser
\end{flushleft}





\begin{flushleft}
There were three cooler condensers used instead of the conventional two to reduce the
\end{flushleft}


\begin{flushleft}
load on condensers.
\end{flushleft}





\begin{flushleft}
Figure 4.3: Nitric Acid Cooler Condenser Structure, provided by DFPCL [7]
\end{flushleft}





4.1.9





\begin{flushleft}
Absorption Column
\end{flushleft}





\begin{flushleft}
A sieve tray counter-current absorption column (42 m in height) absorbs incoming NOX
\end{flushleft}


\begin{flushleft}
into a deionised makeup water stream to form 59\% w/w Nitric Acid as bottom product.
\end{flushleft}


\begin{flushleft}
NO2 absorption in water occurs on the tray whereas NO oxidation to NO2 occurs between
\end{flushleft}


\begin{flushleft}
the trays. Cooling water coils at each stage are used to absorb the heat released by
\end{flushleft}


\begin{flushleft}
reactions.
\end{flushleft}





4.2





\begin{flushleft}
Conclusion
\end{flushleft}





\begin{flushleft}
The process studied from literature and the one being used in the plant was very similar. Doubts regarding the type of equipment used, exact conversions and temperature
\end{flushleft}


\begin{flushleft}
pressure conditions were solved by this visit. The size of the DFPCL Dual Pressure
\end{flushleft}


\begin{flushleft}
Plant was comparatively low and with reference to findings from the visit, changes were
\end{flushleft}


\begin{flushleft}
made to scale up the plant capacity for verification with our new plant data.
\end{flushleft}





\begin{flushleft}
\newpage
Chapter 5
\end{flushleft}





\begin{flushleft}
Process Flow Diagram
\end{flushleft}


\begin{flushleft}
This chapter showcases the Process Flow Diagram (PFD) for the Dual Pressure Nitric
\end{flushleft}


\begin{flushleft}
Acid Process. The PFD was made with inputs from Moulijn et. al. [5], Thiemann et.
\end{flushleft}


\begin{flushleft}
al. [6] and DFPCL Staff [7]. The PFD was drawn using an online flowsheet drawing
\end{flushleft}


\begin{flushleft}
software Draw.io1 .
\end{flushleft}





5.1





\begin{flushleft}
Process Flow Diagram
\end{flushleft}





\begin{flushleft}
The final PFD with the stream table for our Dual Pressure Nitric Acid Process is available in Appendix A. All pressures mentioned are in bar abs, unless specified otherwise.
\end{flushleft}





5.2





\begin{flushleft}
Process Explanation
\end{flushleft}





\begin{flushleft}
$\bullet$ Air (stream-1) is pumped into the plant at ambient conditions (1 atm, 35◦ C),
\end{flushleft}


\begin{flushleft}
entirely by the suction force provided by the centrifugal compressor, C101. The
\end{flushleft}


\begin{flushleft}
Air is filtered using primary and secondary Air Filters in F101 and fed to the
\end{flushleft}


\begin{flushleft}
compressor (stream-1A).
\end{flushleft}


\begin{flushleft}
$\bullet$ The compressor C101 is used to increase the pressure of the air feed to 5 bar, also
\end{flushleft}


\begin{flushleft}
causing an increase in the temperature to 240-250◦ C. Stream-1B is split into 2
\end{flushleft}


\begin{flushleft}
parts, 85\% going into the reaction mixture (Primary Air, stream-1C) and the rest
\end{flushleft}


\begin{flushleft}
(Secondary Air, stream-1D) is utilised in the bleaching column, B101.
\end{flushleft}


1





\begin{flushleft}
https://www.draw.io/
\end{flushleft}





22





\begin{flushleft}
\newpage
Process Flow Diagram
\end{flushleft}





23





\begin{flushleft}
$\bullet$ Liquid Ammonia from storage is pumped into the plant at -33◦ C and 19 bar, into
\end{flushleft}


\begin{flushleft}
Evaporator E101 by a Centrifugal Pump, numbered stream-2. In the Evaporator,
\end{flushleft}


\begin{flushleft}
Ammonia is converted to vapor and further superheated to 70◦ C with the help of
\end{flushleft}


\begin{flushleft}
LP Steam (4-5 bar, 150◦ C) in the upper section of Evaporator E101.
\end{flushleft}


\begin{flushleft}
$\bullet$ Ammonia Filter, F102 is utilised to filter the superheated Ammonia vapor (stream2A) of the small amounts of impurities inherent to the Ammonia feed (oils, chlorine,
\end{flushleft}


\begin{flushleft}
etc.).
\end{flushleft}


\begin{flushleft}
$\bullet$ The superheated ammonia vapor and the compressed air stream are mixed in a
\end{flushleft}


\begin{flushleft}
static mixer, with ammonia to air ratio being 1:9 by volume, giving us stream-3,
\end{flushleft}


\begin{flushleft}
at 5 bar, 220-230◦ C.
\end{flushleft}


\begin{flushleft}
$\bullet$ Stream-3 is fed into the reactor, R101, which is a shallow bed reactor, with around
\end{flushleft}


\begin{flushleft}
8-10 catalyst gauzes, made up of 90\% Pt and 10\% Rh. Hydrogen gas is utilised
\end{flushleft}


\begin{flushleft}
to heat up the gauzes to the required temperature at startup; once the reaction
\end{flushleft}


\begin{flushleft}
starts, the heat of the reaction is sufficient to sustain the reaction.
\end{flushleft}


\begin{flushleft}
$\bullet$ The catalyst is highly selective, giving a conversion of 96\% for Ammonia into
\end{flushleft}


\begin{flushleft}
Nitric Oxide, with stream-4 exiting at a temperature of 890◦ C, consisting mainly
\end{flushleft}


\begin{flushleft}
of Nitrogen and Oxygen from Air feed and Nitric Oxide formed via Ammonia
\end{flushleft}


\begin{flushleft}
oxidation. Some small amounts of N2 O is also formed (1200 ppmv) due to the
\end{flushleft}


\begin{flushleft}
occurence of the side reaction.
\end{flushleft}


\begin{flushleft}
$\bullet$ The high heat of reaction is used to make steam and heat other streams throughout
\end{flushleft}


\begin{flushleft}
the plant. Waste Heat Boiler attached to the reactor is used to produce HP Steam
\end{flushleft}


\begin{flushleft}
(40 barg), which is used to run the steam turbine, using the energy to run the
\end{flushleft}


\begin{flushleft}
compressors. Shell and Tube Heat Exchanger, HE102 is used to heat the tail gas
\end{flushleft}


\begin{flushleft}
(stream-10) coming from Absorption Column to facilitate the abatement process,
\end{flushleft}


\begin{flushleft}
and HE103 is an Economiser, used to preheat the water to boiler to 80◦ C, which
\end{flushleft}


\begin{flushleft}
results in the temperature of stream-5 to drop to 185◦ C.
\end{flushleft}


\begin{flushleft}
$\bullet$ This decrease in the temperature leads to the oxidation of Nitric Oxide to Nitrogen
\end{flushleft}


\begin{flushleft}
Dioxide (stream-6), with the conversion of the reaction increasing with a decrease
\end{flushleft}


\begin{flushleft}
in the temperature.
\end{flushleft}


\begin{flushleft}
$\bullet$ Cooler Condenser-1 (HE104), is used to cool the stream further, condense water
\end{flushleft}


\begin{flushleft}
and nitric acid and separate them out for further processing. The temperature
\end{flushleft}





\begin{flushleft}
\newpage
Process Flow Diagram
\end{flushleft}





24





\begin{flushleft}
of the outlet streams (7 \& 8) is around 50◦ C. This drop in temperature allows
\end{flushleft}


\begin{flushleft}
Nitric Oxide to oxidise to Nitrogen Dioxide at a much higher conversion rate when
\end{flushleft}


\begin{flushleft}
compared to the previous two heat exchangers (HE102 \& HE103).
\end{flushleft}


\begin{flushleft}
$\bullet$ As Nitrogen Dioxide is produced in HE104, with the drop in temperature and
\end{flushleft}


\begin{flushleft}
elevated pressures, some of it gets absorbed into the condensed water, which was
\end{flushleft}


\begin{flushleft}
formed in reactor (R101). Once absorbed, Nitrogen Dioxide combines with Water
\end{flushleft}


\begin{flushleft}
to form Nitric Acid and Nitric Oxide, which further gets oxidised to Nitrogen
\end{flushleft}


\begin{flushleft}
Dioxide, which again forms Nitric Acid and so on.
\end{flushleft}


\begin{flushleft}
$\bullet$ The Nitric Acid solution in the condensate of the Cooler Condenser-1 (HE104),
\end{flushleft}


\begin{flushleft}
with a concentration of about 35-40\% wt Nitric Acid by weight, is then pumped
\end{flushleft}


\begin{flushleft}
(stream-7) to the Absorption Column (A101) to increase the Nitric Acid concentration in the solution.
\end{flushleft}


\begin{flushleft}
$\bullet$ The vapor outlet of HE104, stream-8, mainly consisting of NOX gases, is passed
\end{flushleft}


\begin{flushleft}
through the NOX Compressor, C102, elevating the pressure of the stream to 12
\end{flushleft}


\begin{flushleft}
bar and the temperature to around 260◦ C.
\end{flushleft}


\begin{flushleft}
$\bullet$ C102 outlet, stream 8A, is the fed to Cooler Condenser-2 (HE105), where process
\end{flushleft}


\begin{flushleft}
similar to that in Cooler Condenser-1 (HE104) occurs, providing us with more
\end{flushleft}


\begin{flushleft}
weak Nitric Acid condensate, which is mixed with HE104 condensate and pumped
\end{flushleft}


\begin{flushleft}
to the Absorption Column. Stream-9 coming out of HE105 is also fed to the
\end{flushleft}


\begin{flushleft}
Absorption Column (A101).
\end{flushleft}


\begin{flushleft}
$\bullet$ The Absorption Column (A101), has 35-40 stages, with the NOX gases (stream-9)
\end{flushleft}


\begin{flushleft}
fed at the bottom-most stage, process water (stream-12) fed at the top-most stage,
\end{flushleft}


\begin{flushleft}
acid condensate from the Cooler Condensers (stream-7) fed to an intermediate
\end{flushleft}


\begin{flushleft}
stage (stage 13) and cooling water coils at every stage throughout the column to
\end{flushleft}


\begin{flushleft}
maintain the column at a reduced temperature.
\end{flushleft}


\begin{flushleft}
$\bullet$ Since the Absorption Column is operated at an elevated pressure (12 bar), absorption efficiency is increased, which results in fewer NOX emissions.
\end{flushleft}


\begin{flushleft}
$\bullet$ Inside the column, Nitric Oxide oxidation to Nitrogen Dioxide occurs between
\end{flushleft}


\begin{flushleft}
stages in the vapor phase, and absorption of Nitrogen Dioxide into water and the
\end{flushleft}


\begin{flushleft}
formation of Nitric Acid from the absorbed Nitrogen Dioxide occurs in the liquid
\end{flushleft}


\begin{flushleft}
phase on the sieve trays of the column.
\end{flushleft}





\begin{flushleft}
\newpage
Process Flow Diagram
\end{flushleft}





25





\begin{flushleft}
$\bullet$ Top outlet of the column is called Tail Gas (stream-10), with NOX concentration
\end{flushleft}


\begin{flushleft}
in the order of 550ppmv. Stream-10 is then preheated in HE106 using secondary
\end{flushleft}


\begin{flushleft}
air (stream-1D) and further heated in HE102 using stream-4 and sent to the NOX
\end{flushleft}


\begin{flushleft}
abatement process where it is expanded to provide 65\% of the energy required to
\end{flushleft}


\begin{flushleft}
run the compressor shafts and then vented into the air.
\end{flushleft}


\begin{flushleft}
$\bullet$ Bottom outlet of the column (stream-11) is called Red Fuming Nitric Acid, due to
\end{flushleft}


\begin{flushleft}
the red color imparted by the dissolved Nitrogen Dioxide.
\end{flushleft}


\begin{flushleft}
$\bullet$ Stream-11 is fed to the Bleaching Column, B101, where the secondary air strips
\end{flushleft}


\begin{flushleft}
away the Nitrogen Dioxide, resulting in a clear, colorless 62\% wt. Nitric Acid
\end{flushleft}


\begin{flushleft}
Solution. The product acid (stream-14) is further cooled and stored in cylindrical
\end{flushleft}


\begin{flushleft}
storage tanks. The bleaching air is mixed with the NOX compressor inlet and
\end{flushleft}


\begin{flushleft}
again fed to the absorption column, to ensure lower NOX emissions in the tail gas.
\end{flushleft}


\begin{flushleft}
$\bullet$ The Tail Gas (stream-10) from the column is first preheated to 130◦ C, using the
\end{flushleft}


\begin{flushleft}
secondary air to be fed to the bleaching column, in Tail Gas Preheater, HE106,
\end{flushleft}


\begin{flushleft}
and further heated to 350◦ C in HE102.
\end{flushleft}


\begin{flushleft}
$\bullet$ The tail gas can now be fed to the abatement process, where the NOX and N2 O
\end{flushleft}


\begin{flushleft}
concentration is reduced to below 40 ppmv. Ammonia and Propane are mixed
\end{flushleft}


\begin{flushleft}
with the tail gas as it enters the abatement unit.
\end{flushleft}


\begin{flushleft}
$\bullet$ After the abatement, the tail gas is expanded in the tail gas turbine, EX101,
\end{flushleft}


\begin{flushleft}
providing around 65\% of the energy to drive both the air and NOX compressors.
\end{flushleft}


\begin{flushleft}
Once expanded, the tail gas is vented into the air via Stack.
\end{flushleft}





\begin{flushleft}
\newpage
Figure 5.1: Process Flow Diagram for Dual Pressure Nitric Acid Process
\end{flushleft}





\begin{flushleft}
Process Flow Diagram
\end{flushleft}


26





\begin{flushleft}
\newpage
Process Flow Diagram
\end{flushleft}





5.3





27





\begin{flushleft}
Process Heat Integration
\end{flushleft}





\begin{flushleft}
Nitric Acid Process is one of the world's oldest and most researched industrial production
\end{flushleft}


\begin{flushleft}
process. Hence, the intense amount of research that has gone into this process has led to
\end{flushleft}


\begin{flushleft}
the latest iteration of the process flow to take shape, that we have incorporated into our
\end{flushleft}


\begin{flushleft}
PFD. This process is highly integrated, from heat to energy. There was no requirement
\end{flushleft}


\begin{flushleft}
to do a detailed heat integration study as the entire process has been well researched
\end{flushleft}


\begin{flushleft}
and provided with the ample amount of heat and energy integration required to make
\end{flushleft}


\begin{flushleft}
any nitric acid plant profitable in the smallest period of time.
\end{flushleft}


\begin{flushleft}
The process has one main hot stream - Reactor outlet gas, coming out at a temperature of
\end{flushleft}


\begin{flushleft}
890◦ C. It has to be cooled down to a temperature of 50◦ C before it enters the absorption
\end{flushleft}


\begin{flushleft}
column. The first heat exchanger after the reactor is the Waste Heat Boiler, attached
\end{flushleft}


\begin{flushleft}
directly to the reactor burner head. The gas flows through the shell side of this shell and
\end{flushleft}


\begin{flushleft}
tube heat exchanger, with boiler feed water passing through the special boiler tubes,
\end{flushleft}


\begin{flushleft}
made out of SA 192 and SA 213 - T12 Grade, to account for the evaporation and super
\end{flushleft}


\begin{flushleft}
heating of the High Pressure steam produced in the boiler. This causes the temperature
\end{flushleft}


\begin{flushleft}
of reaction mixture to drop to 427◦ C.
\end{flushleft}


\begin{flushleft}
This HP Steam is used to run the Steam turbine used to run the common shaft, which in
\end{flushleft}


\begin{flushleft}
turn drives the compressors. The LP Steam from the turbine is then used to superheat
\end{flushleft}


\begin{flushleft}
the evaporated ammonia to a temperature of 70◦ C. Ammonia evaporation utilises cooling
\end{flushleft}


\begin{flushleft}
water to evaporate the liquid ammonia. The water in turn is cooled down to 7◦ C, where
\end{flushleft}


\begin{flushleft}
it can be used as chilled water, used for cooling the top section of the absorption column.
\end{flushleft}


\begin{flushleft}
After the Waste Heat Boiler, the NO gas is further cooled down to 300◦ C in the Tail
\end{flushleft}


\begin{flushleft}
Gas Heater, by utilising the cold tail gas emanating from the absorption column. This
\end{flushleft}


\begin{flushleft}
heats the tail gas (stream-10) to 350◦ C, temperature required by the abatement unit
\end{flushleft}


\begin{flushleft}
to function properly. The NO gas is further cooled down to 185◦ C in the economiser.
\end{flushleft}


\begin{flushleft}
The cooling stream here is the boiler feed water used in the Waste Heat Boiler, whose
\end{flushleft}


\begin{flushleft}
temperature is increased from 40◦ C to 80◦ C.
\end{flushleft}


\begin{flushleft}
The next two shell and tube heat exchangers are the largest heat exchanger in the plant.
\end{flushleft}


\begin{flushleft}
The cooler condensers are utilised to cool down the reaction gases to 50◦ C, which helps
\end{flushleft}


\begin{flushleft}
in forming the acid condensate. Cooling water is used in both the cooler condensers,
\end{flushleft}


\begin{flushleft}
one before and one after the NOX compression.
\end{flushleft}





\begin{flushleft}
\newpage
Process Flow Diagram
\end{flushleft}





28





\begin{flushleft}
The tail gas coming out of the Absorption Column is reduced in order to meet the
\end{flushleft}


\begin{flushleft}
government norms for NOX emissions. The abatement process requires temperatures of
\end{flushleft}


\begin{flushleft}
around 350◦ C (stream-15), and hence the tail gas, coming out at 10◦ C, is first preheated
\end{flushleft}


\begin{flushleft}
in the Tail Gas Preheater to 130◦ C (stream-10), with the help of the hot secondary air
\end{flushleft}


\begin{flushleft}
(stream-1D) used in the bleaching column, and further heated in the Tail Gas Heater
\end{flushleft}


\begin{flushleft}
(HE102) as mentioned in previous points. Once abated, the tail gas is used to drive a
\end{flushleft}


\begin{flushleft}
turbine, which provides around 65\% of the power required to drive the air and NOX
\end{flushleft}


\begin{flushleft}
compressors (C101 and C102).
\end{flushleft}


\begin{flushleft}
This integration is followed in all Dual Pressure Nitric Acid Processes, with slight variations in stream temperatures for processes from different technology providers. The net
\end{flushleft}


\begin{flushleft}
utility data is produced in a tabulated form below:
\end{flushleft}





\begin{flushleft}
Utility Required
\end{flushleft}





\begin{flushleft}
Net Amount Required
\end{flushleft}





\begin{flushleft}
Utility
\end{flushleft}





\begin{flushleft}
Amount
\end{flushleft}





\begin{flushleft}
Produced
\end{flushleft}





\begin{flushleft}
Produced
\end{flushleft}





\begin{flushleft}
Electricity
\end{flushleft}





\begin{flushleft}
2465 kW
\end{flushleft}





\begin{flushleft}
Cooling Water
\end{flushleft}





\begin{flushleft}
0.437 MGal/hr
\end{flushleft}





\begin{flushleft}
HP Steam
\end{flushleft}





\begin{flushleft}
Make Up Chilled Water
\end{flushleft}





\begin{flushleft}
0.0134 MGal/hr
\end{flushleft}





\begin{flushleft}
(40 barg)
\end{flushleft}





\begin{flushleft}
LP Steam
\end{flushleft}





\begin{flushleft}
12600 kg/hr
\end{flushleft}





\begin{flushleft}
26680 kg/hr
\end{flushleft}





\begin{flushleft}
Table 5.1: Utility Data after Heat Integration
\end{flushleft}





\begin{flushleft}
The above mentioned Heat and Energy integration leads to the following heat exchanger
\end{flushleft}


\begin{flushleft}
network:
\end{flushleft}





\begin{flushleft}
\newpage
Figure 5.2: Heat Exchanger Network
\end{flushleft}





\begin{flushleft}
Process Flow Diagram
\end{flushleft}


29





\begin{flushleft}
\newpage
Chapter 6
\end{flushleft}





\begin{flushleft}
Mass Balance
\end{flushleft}


\begin{flushleft}
This chapter provides a look at the mass balance performed on the process, with inputs
\end{flushleft}


\begin{flushleft}
from Ray et. al. [8], source of the reference being DFPCL [7]. Some sample calculations
\end{flushleft}


\begin{flushleft}
are also provided for some major equipment taking 100kmol/hr Ammonia as the basis.
\end{flushleft}


\begin{flushleft}
All values were scaled up for the desired plant capacity.
\end{flushleft}





6.1


6.1.1





\begin{flushleft}
Sample Calculations
\end{flushleft}


\begin{flushleft}
Reactor - R101
\end{flushleft}





\begin{flushleft}
Air is passed through a combination of primary and secondary filters and compressed
\end{flushleft}


\begin{flushleft}
to 5 bar abs (240◦ C) from ambient conditions. Ammonia is evaporated from sub-zero
\end{flushleft}


\begin{flushleft}
temperatures using water, passed through a preheater to raise the temperature to 70◦ C
\end{flushleft}


\begin{flushleft}
and filtered to remove oils and particulate matter using the ammonia filters. Ammonia
\end{flushleft}


\begin{flushleft}
\& air are mixed in 1:9 ratio and the mixture enters the reactor at 230◦ C.
\end{flushleft}


\begin{flushleft}
Reactor residence time is in few milliseconds where Pt-Rh catalyst provides the site for
\end{flushleft}


\begin{flushleft}
reaction. Reaction occurs at 890◦ C which is maintained by the high exothermicity of
\end{flushleft}


\begin{flushleft}
the reaction.
\end{flushleft}


\begin{flushleft}
Main Reaction:
\end{flushleft}


\begin{flushleft}
4N H3 (g) + 5O2 (g) $\rightarrow$ 4N O + 6H2 O, ∆H = $-$905kJ/mol
\end{flushleft}





30





\begin{flushleft}
\newpage
Mass Balance
\end{flushleft}





31





\begin{flushleft}
96\% ammonia conversion is towards main reaction producing NO and the rest towards
\end{flushleft}


\begin{flushleft}
side reactions:
\end{flushleft}


\begin{flushleft}
Ammonia entering reactor = 100 kmol
\end{flushleft}


\begin{flushleft}
NO formed: 100*(96/100) = 96 kmol
\end{flushleft}


\begin{flushleft}
Required O2 for main reaction: 96*5/4 = 120 kmol
\end{flushleft}


\begin{flushleft}
H2 O formed in main reaction: 96*3/2 = 144 kmol
\end{flushleft}


\begin{flushleft}
Side Reaction 1:
\end{flushleft}


\begin{flushleft}
4N H3 (g) + 3O2 (g) $\rightarrow$ 2N2 (g) + 6H2 O(g), ∆H = $-$1261kJ/mol
\end{flushleft}


\begin{flushleft}
3\% ammonia is consumed in side reaction 1:
\end{flushleft}


\begin{flushleft}
Amount of N2 produced = 3/2 = 1.5 kmol
\end{flushleft}


\begin{flushleft}
Amount of O2 required for side reaction = 1.5*3/2 = 2.25 kmol
\end{flushleft}


\begin{flushleft}
Amount of H2 O produced in side reaction = 2.25*2 = 4.5 kmol
\end{flushleft}


\begin{flushleft}
Side Reaction 2:
\end{flushleft}


\begin{flushleft}
4N H3 (g) + 4O2 (g) $\rightarrow$ 2N2 O(g) + 6H2 O(g), ∆H = $-$1102kJ/mol
\end{flushleft}


\begin{flushleft}
1\% ammonia is consumed in side reaction 1:
\end{flushleft}


\begin{flushleft}
Amount of N2 O produced = 1/2 = 0.5 kmol
\end{flushleft}


\begin{flushleft}
Amount of O2 required for side reaction = 0.5*3/2 = 0.75 kmol
\end{flushleft}


\begin{flushleft}
Amount of H2 O produced in side reaction = 0.75*2 = 1.5 kmol
\end{flushleft}


\begin{flushleft}
Total water produced = 144 + 4.5 + 1.5 = 150 kmol
\end{flushleft}


\begin{flushleft}
Total O2 consumed = 120 + 2.25 + 0.75 = 123 kmol
\end{flushleft}


\begin{flushleft}
Similar calculations are made for all other equipment using their conversions and desired
\end{flushleft}


\begin{flushleft}
outputs.
\end{flushleft}





\begin{flushleft}
Components
\end{flushleft}





\begin{flushleft}
Reactor Feed (kmol/hr)
\end{flushleft}





\begin{flushleft}
Reactor Outlet (kmol/hr)
\end{flushleft}





\begin{flushleft}
NH3
\end{flushleft}





100





0





\begin{flushleft}
O2
\end{flushleft}





189





66





\begin{flushleft}
N2
\end{flushleft}





711





712.5





\begin{flushleft}
N2 O
\end{flushleft}





0





0.5





\begin{flushleft}
H2 O
\end{flushleft}





0





150





\begin{flushleft}
NO
\end{flushleft}





0





96





\begin{flushleft}
Table 6.1: Reactor Balance
\end{flushleft}





\begin{flushleft}
\newpage
Mass Balance
\end{flushleft}





6.1.2





32





\begin{flushleft}
Heat Exchangers and Cooler Condenser
\end{flushleft}





\begin{flushleft}
The primary reaction that takes place next is the oxidation of nitric oxide to nitrogen
\end{flushleft}


\begin{flushleft}
dioxide.
\end{flushleft}


\begin{flushleft}
2N O(g) + O2 (g) $\rightarrow$ 2N O2 (g), ∆H = $-$114kJ/mol
\end{flushleft}


\begin{flushleft}
This reaction takes place when the process gas is passed through a series of heat exchangers and the conversion increases with reducing temperature. The process gas first passes
\end{flushleft}


\begin{flushleft}
through Tail Gas Heater-1, where its heat is used to preheat the tail gas exiting from
\end{flushleft}


\begin{flushleft}
the absorption column. It further passes through an economizer, used to preheat boiler
\end{flushleft}


\begin{flushleft}
feed water and then through cooler condenser-1. By this point, the temperature of the
\end{flushleft}


\begin{flushleft}
process gas comes down to 50◦ C, which is favourable for high nitric oxide conversion.
\end{flushleft}


\begin{flushleft}
Simultaneously, water formed in the ammonia oxidation condenses, thereby dissolving
\end{flushleft}


\begin{flushleft}
NO2 and forming HNO3 in small amounts as condensate.
\end{flushleft}


\begin{flushleft}
3N O2 (g) + H2 O(l) $\rightarrow$ 2HN O3 (aq.) + N O(g), ∆H = $-$75kJ/mol
\end{flushleft}


\begin{flushleft}
The conversion of NO in Tail gas heater-1 is $\approx$ 15\%, in the economiser is $\approx$ 27\% and
\end{flushleft}


\begin{flushleft}
the concentration of acid condensate is $\approx$ 38\%. These figures are similar to the DFPCL
\end{flushleft}


\begin{flushleft}
data [7].
\end{flushleft}





\begin{flushleft}
NOx
\end{flushleft}


\begin{flushleft}
Reactor
\end{flushleft}


\begin{flushleft}
Component
\end{flushleft}





\begin{flushleft}
HE102
\end{flushleft}





\begin{flushleft}
HE103
\end{flushleft}





\begin{flushleft}
Acid
\end{flushleft}


\begin{flushleft}
Comp.
\end{flushleft}





\begin{flushleft}
Outlet
\end{flushleft}





\begin{flushleft}
Outlet
\end{flushleft}





\begin{flushleft}
Outlet
\end{flushleft}





\begin{flushleft}
Condensate
\end{flushleft}


\begin{flushleft}
Inlet
\end{flushleft}





\begin{flushleft}
(kmol/hr)
\end{flushleft}





\begin{flushleft}
(kmol/hr)
\end{flushleft}





\begin{flushleft}
(kmol/hr)
\end{flushleft}





\begin{flushleft}
(kmol/hr)
\end{flushleft}


\begin{flushleft}
(kmol/hr)
\end{flushleft}





\begin{flushleft}
O2
\end{flushleft}





66





59





48





20





0





\begin{flushleft}
N2
\end{flushleft}





712.5





712.5





712.5





712.5





0





\begin{flushleft}
H2 O
\end{flushleft}





150





150





150





0





138





\begin{flushleft}
N2 O
\end{flushleft}





0.5





0.5





0.5





0.5





0





\begin{flushleft}
NO2
\end{flushleft}





0





14





36





56





0





\begin{flushleft}
NO
\end{flushleft}





96





82





60





16





0





\begin{flushleft}
HNO3
\end{flushleft}





0





0





0





0





24





\begin{flushleft}
Table 6.2: Heat Exchanger Balance
\end{flushleft}





\begin{flushleft}
\newpage
Mass Balance
\end{flushleft}





6.1.3





33





\begin{flushleft}
Absorption Column and Cooler Condenser
\end{flushleft}





\begin{flushleft}
The process gas (known as NOX gas) is compressed to 12 bar abs (260◦ C) and fed to
\end{flushleft}


\begin{flushleft}
cooler condenser 2. The gas stream is then fed to the absorption column at the bottom
\end{flushleft}


\begin{flushleft}
and acid condensate from cooler condenser 1 is fed at the middle of the column for
\end{flushleft}


\begin{flushleft}
further concentration. Process water added at the top of the column at 10◦ C absorbs
\end{flushleft}


\begin{flushleft}
the NO2 forming HNO3 and NO formed as a side product gets oxidised back to NO2 by
\end{flushleft}


\begin{flushleft}
excess air added to the column.
\end{flushleft}


\begin{flushleft}
The acid leaving the column from the bottoms is 55\%-60\%wt which then passes through
\end{flushleft}


\begin{flushleft}
the bleaching column to be stripped off of dissolved NOX gases. The tail gas leaving
\end{flushleft}


\begin{flushleft}
from the tower of the tower is further sent for NOX abatement process to bring down
\end{flushleft}


\begin{flushleft}
the NOX content to permissible levels. The basis for calculations on the column was the
\end{flushleft}


\begin{flushleft}
empirical concentration of different NOX gases in the tail gas as 550ppmv.
\end{flushleft}





\begin{flushleft}
Cooler
\end{flushleft}


\begin{flushleft}
Acid
\end{flushleft}


\begin{flushleft}
Component
\end{flushleft}





\begin{flushleft}
Condenser 2
\end{flushleft}





\begin{flushleft}
Nitric Acid
\end{flushleft}


\begin{flushleft}
Tail Gas
\end{flushleft}





\begin{flushleft}
Condensate
\end{flushleft}


\begin{flushleft}
Outlet
\end{flushleft}





\begin{flushleft}
Bottom
\end{flushleft}


\begin{flushleft}
(kmol/hr)
\end{flushleft}





\begin{flushleft}
(kmol/hr)
\end{flushleft}





\begin{flushleft}
(kmol/hr)
\end{flushleft}





\begin{flushleft}
(kmol/hr)
\end{flushleft}


\begin{flushleft}
NH3
\end{flushleft}





0





0





0





0





\begin{flushleft}
O2
\end{flushleft}





12.5





0





19.5





0





\begin{flushleft}
N2
\end{flushleft}





713





0





808





0





\begin{flushleft}
H2 O
\end{flushleft}





0





138





0





202.54





\begin{flushleft}
N2 O
\end{flushleft}





0.5





0





0.5





0





\begin{flushleft}
NO2
\end{flushleft}





71





0





0.06





7.82





\begin{flushleft}
NO
\end{flushleft}





1





0





0.6





0





\begin{flushleft}
HNO3
\end{flushleft}





0





24





0





94.42





\begin{flushleft}
Table 6.3: Absorption Column and Cooler Condenser Balance
\end{flushleft}





6.2





\begin{flushleft}
Final Mass Balance Table
\end{flushleft}





\begin{flushleft}
Above calculations were scaled to our desired production of 1500TPD of 62\% Nitric Acid
\end{flushleft}


\begin{flushleft}
Solution, providing us with the following results 1 :
\end{flushleft}


1





\begin{flushleft}
All values are in kg/hr
\end{flushleft}





\begin{flushleft}
\newpage
Stream Comp.
\end{flushleft}





\begin{flushleft}
Phase
\end{flushleft}





\begin{flushleft}
NH3
\end{flushleft}





\begin{flushleft}
O2
\end{flushleft}





\begin{flushleft}
N2
\end{flushleft}





\begin{flushleft}
H2 O
\end{flushleft}





\begin{flushleft}
N2 O
\end{flushleft}





\begin{flushleft}
NO2
\end{flushleft}





\begin{flushleft}
NO
\end{flushleft}





\begin{flushleft}
HNO3
\end{flushleft}





\begin{flushleft}
CO2
\end{flushleft}





\begin{flushleft}
C3H8
\end{flushleft}





\begin{flushleft}
Total
\end{flushleft}





\begin{flushleft}
Pressure(bar abs)
\end{flushleft}





\begin{flushleft}
Temperature (◦ C)
\end{flushleft}





1





\begin{flushleft}
Air feed
\end{flushleft}





\begin{flushleft}
g
\end{flushleft}





0





43519





145541





0





0





0





0





0





0





0





189060





1





35





\begin{flushleft}
1C
\end{flushleft}





\begin{flushleft}
Primary Air
\end{flushleft}





\begin{flushleft}
g
\end{flushleft}





0





39005





128387





0





0





0





0





0





0





0





167391





5





240





\begin{flushleft}
1D
\end{flushleft}





\begin{flushleft}
Secondary Air
\end{flushleft}





\begin{flushleft}
g
\end{flushleft}





0





4514





17154





0





0





0





0





0





0





0





21669





5





240





2





\begin{flushleft}
Ammonia Feed
\end{flushleft}





\begin{flushleft}
l
\end{flushleft}





11003





0





0





0





0





0





0





0





0





0





11003





1





-33





\begin{flushleft}
2A
\end{flushleft}





\begin{flushleft}
Evaporated NH3
\end{flushleft}





\begin{flushleft}
g
\end{flushleft}





10963





0





0





0





0





0





0





0





0





0





10963





5





70





3





\begin{flushleft}
Reactor Inlet
\end{flushleft}





\begin{flushleft}
g
\end{flushleft}





10963





39005





128387





0





0





0





0





0





0





0





178355





5





230





4





\begin{flushleft}
Reactor Outlet
\end{flushleft}





\begin{flushleft}
g
\end{flushleft}





0





13570





128658





17412





142





0





18573





0





0





0





178355





5





427





5





\begin{flushleft}
HE102 Outlet
\end{flushleft}





\begin{flushleft}
g
\end{flushleft}





0





12125





128658





17412





142





4153





15865





0





0





0





178355





5





300





6





\begin{flushleft}
HE103 Outlet
\end{flushleft}





\begin{flushleft}
g
\end{flushleft}





0





9856





128658





17412





142





10680





11608





0





0





0





178355





5





185





7





\begin{flushleft}
Acid Condensate
\end{flushleft}





\begin{flushleft}
l
\end{flushleft}





0





0





0





16019





0





0





0





9751





0





0





25770





12





50





8





\begin{flushleft}
NOx Comp. inlet
\end{flushleft}





\begin{flushleft}
g
\end{flushleft}





0





8591





145812





0





142





18932





3096





0





0





0





176573





5





50





\begin{flushleft}
8A
\end{flushleft}





\begin{flushleft}
NOx Comp. Outlet
\end{flushleft}





\begin{flushleft}
g
\end{flushleft}





0





8591





145812





0





142





18932





3096





0





0





0





176573





12





260





9





\begin{flushleft}
Absorption Inlet
\end{flushleft}





\begin{flushleft}
g
\end{flushleft}





0





7043





145812





0





142





23382





193





0





0





0





176573





12





50





10





\begin{flushleft}
Tail Gas
\end{flushleft}





\begin{flushleft}
g
\end{flushleft}





0





3307





145812





0





142





18





58





0





0





0





149336





12





130





11





\begin{flushleft}
Absorption Outlet
\end{flushleft}





\begin{flushleft}
l
\end{flushleft}





0





0





0





23816





0





2320





0





38858





0





0





64993





12





50





12





\begin{flushleft}
Process Water
\end{flushleft}





\begin{flushleft}
l
\end{flushleft}





0





0





0





11987





0





0





0





0





0





0





11987





12





10





13





\begin{flushleft}
Bleaching Air
\end{flushleft}





\begin{flushleft}
g
\end{flushleft}





0





4514





17154





0





0





2320





0





0





0





0





23988





5





80





14





\begin{flushleft}
Nitric Acid Pdt
\end{flushleft}





\begin{flushleft}
l
\end{flushleft}





0





0





0





23816





0





0





0





38858





0





0





62673





5





50





15





\begin{flushleft}
Mixed Tail Gas
\end{flushleft}





\begin{flushleft}
g
\end{flushleft}





40





3307





145812





0





142





18





58





0





0





0





149376





12





350





16





\begin{flushleft}
Reduced Tail Gas
\end{flushleft}





\begin{flushleft}
g
\end{flushleft}





0





3292





145966





87





2





1





2





0





42





0





149391





12





350





17





\begin{flushleft}
Tail Gas Out
\end{flushleft}





\begin{flushleft}
g
\end{flushleft}





0





3292





145966





87





2





1





2





0





42





0





149391





1





35





18





\begin{flushleft}
Reducing Ammonia
\end{flushleft}





\begin{flushleft}
g
\end{flushleft}





40





0





0





0





0





0





0





0





0





0





40





12





350





19





\begin{flushleft}
Propane
\end{flushleft}





\begin{flushleft}
g
\end{flushleft}





0





0





0





0





0





0





0





0





0





15





15





12





350





\begin{flushleft}
Mass Balance
\end{flushleft}





\begin{flushleft}
Line No.
\end{flushleft}





\begin{flushleft}
Table 6.4: Final Mass Balance
\end{flushleft}





34





\begin{flushleft}
\newpage
Chapter 7
\end{flushleft}





\begin{flushleft}
DWSIM Flowsheet Simulation
\end{flushleft}


\begin{flushleft}
We employed the DWSIM Chemical Process Simulator for process simulations of the
\end{flushleft}


\begin{flushleft}
Nitric Acid plant. DWSIM is a multiplatform, CAPE-OPEN compliant chemical process
\end{flushleft}


\begin{flushleft}
simulator featuring a rich graphical interface with many features previously available
\end{flushleft}


\begin{flushleft}
only in commercial simulators. DWSIM being a free and open source software provides
\end{flushleft}


\begin{flushleft}
everyone with the opportunity to simulate process free of cost. The direct benefit of this
\end{flushleft}


\begin{flushleft}
is that DWSIM can directly replace commercial simulators like ASPEN or can at the
\end{flushleft}


\begin{flushleft}
least be used for multiple simultaneous test-simulations in companies where there is a
\end{flushleft}


\begin{flushleft}
shortage of ASPEN licenses. The simulations helped us not only validate our data from
\end{flushleft}


\begin{flushleft}
linear mass balance and the visited DFPCL plant, but also optimize process operating
\end{flushleft}


\begin{flushleft}
conditions.
\end{flushleft}


\begin{flushleft}
Our DWSIM flowsheet is available online on the FOSSEE Flowsheeting Project portal
\end{flushleft}


\begin{flushleft}
- https://dwsim.fossee.in/flowsheeting-project/dwsim-flowsheet-run/335
\end{flushleft}





7.1





\begin{flushleft}
Kinetics and Specifications used in DWSIM
\end{flushleft}





\begin{flushleft}
The process explained in Chapter 5 is used to simulate a DWSIM flowsheet. Due to
\end{flushleft}


\begin{flushleft}
unavailability of a few equipment in DWSIM, their modelling is done using different
\end{flushleft}


\begin{flushleft}
alternatives. Centrifugal compressors and shell and tube heat exchangers are used where
\end{flushleft}


\begin{flushleft}
ever required.
\end{flushleft}





35





\begin{flushleft}
\newpage
DWSIM Flowsheet Simulation
\end{flushleft}





7.1.1





36





\begin{flushleft}
Ammonia Oxidation
\end{flushleft}





\begin{flushleft}
Ammonia oxidation reaction in presence of Pt-Rh catalyst is highly selective and occurs
\end{flushleft}


\begin{flushleft}
almost instantaneously, with residence time in the order of 10-4 - 10-3 seconds. Hence,
\end{flushleft}


\begin{flushleft}
the available Conversion Reactor is used as an equivalent for the Shallow Bed Reactor
\end{flushleft}


\begin{flushleft}
for optimised temperature and pressure conditions, giving a conversion of 96\% for the
\end{flushleft}


\begin{flushleft}
desired reaction. It is also assumed that almost all ammonia is consumed in the reactor,
\end{flushleft}


\begin{flushleft}
either in the main reaction or in one of the side reactions.
\end{flushleft}


\begin{flushleft}
The instantaneous nature of the reactions call for a millisecond reactor, which is why it
\end{flushleft}


\begin{flushleft}
is difficult to use a kinetic model of this reaction in the simulation of the reactor. Only
\end{flushleft}


\begin{flushleft}
one of the side reactions is considered in the DWSIM Flowsheet.
\end{flushleft}





\begin{flushleft}
Figure 7.1: DWSIM Flowsheet - Ammonia Oxidation
\end{flushleft}





7.1.2





\begin{flushleft}
Nitric Oxide Oxidation
\end{flushleft}





\begin{flushleft}
In this step, the temperature of the stream is reduced which leads to the kinetic oxidation of NO to NO2 to move in the forward direction. Since the heat exchangers
\end{flushleft}


\begin{flushleft}
cannot be modelled to include a reaction in them, plug flow reactors are used instead.
\end{flushleft}


\begin{flushleft}
DWSIM PFRs give us the ability to calculate the heat duty for our required process,
\end{flushleft}


\begin{flushleft}
while enabling us to model kinetic reactions, satisfying our requirements of reactive heat
\end{flushleft}


\begin{flushleft}
exchangers.
\end{flushleft}


\begin{flushleft}
2N O(g) + O2 (g) $\rightarrow$ 2N O2 (g), ∆H = $-$114kJ/mol
\end{flushleft}





\begin{flushleft}
\newpage
DWSIM Flowsheet Simulation
\end{flushleft}





37





\begin{flushleft}
NO Oxidation is a third order reaction, with reaction rate and constant as,
\end{flushleft}


2


\begin{flushleft}
Rate = 2kp CN
\end{flushleft}


\begin{flushleft}
O CO 2
\end{flushleft}





\begin{flushleft}
logkp =
\end{flushleft}





530


+ 7.09


\begin{flushleft}
T
\end{flushleft}





\begin{flushleft}
Hence, the Arrhenius parameters required become, A = 1200 and E = -4406.42, values
\end{flushleft}


\begin{flushleft}
taken from Tsukahara et. al. [9]
\end{flushleft}


\begin{flushleft}
Water formed in the ammonia oxidation reaction reacts with absorbed NO2 to form
\end{flushleft}


\begin{flushleft}
HNO3 . Since the reaction is mass transfer controlled, the kinetics involved in this
\end{flushleft}


\begin{flushleft}
reaction are very difficult to model and therefore a simpler alternative used for this
\end{flushleft}


\begin{flushleft}
reaction was to model an equilibrium reactor, with the equilibrium reaction,
\end{flushleft}


\begin{flushleft}
3N O2 (g) + H2 O(l)
\end{flushleft}





\begin{flushleft}
2HN O3 + N O(g), ∆H = $-$75kJ/mol
\end{flushleft}





\begin{flushleft}
with Kp being the equilibrium constant, given by Thiemann et. al. [6] as,
\end{flushleft}


\begin{flushleft}
logKp = $-$7.35 + 2.64/T
\end{flushleft}


\begin{flushleft}
Cooler Condenser is modelled using a PFR to take into account the NO Oxidation,
\end{flushleft}


\begin{flushleft}
followed by an Equilibrium Reactor to model the Nitric Acid formation and a GasLiquid Separator to produce Acid Condensate and NOX gas.
\end{flushleft}





\begin{flushleft}
Figure 7.2: DWSIM Flowsheet - NO Oxidation and Heat Exchanger Network
\end{flushleft}





7.1.3





\begin{flushleft}
Absorption Column
\end{flushleft}





\begin{flushleft}
The separated streams are sent to the absorption column. In the absorption column,
\end{flushleft}


\begin{flushleft}
the NO2 is absorbed in water and reacts to form HNO3 and NO, which is again oxidised
\end{flushleft}


\begin{flushleft}
to form NO2 , and so on. Modelling of a reactive absorption column was done using a
\end{flushleft}


\begin{flushleft}
ChemSep column, with the specified reactions used in the unit. Issues propped up as
\end{flushleft}





\begin{flushleft}
\newpage
DWSIM Flowsheet Simulation
\end{flushleft}





38





\begin{flushleft}
the desired conversion couldnt be achieved, hence a different approach was used. The
\end{flushleft}


\begin{flushleft}
column was divided into two parts - Absorption and Reaction. A Simple Absorber was
\end{flushleft}


\begin{flushleft}
used to facilitate NOX absorption into water, followed by a series of reactors to model
\end{flushleft}


\begin{flushleft}
the NO Oxidation and Nitric Acid formation reactions. Two recycle streams were used
\end{flushleft}


\begin{flushleft}
in this part, to recycle back the unconverted gases, to achieve the given conversions and
\end{flushleft}


\begin{flushleft}
concentrations.
\end{flushleft}





\begin{flushleft}
Figure 7.3: DWSIM Flowsheet - Reactive Absorption Column Modelling
\end{flushleft}





\begin{flushleft}
The complete DWSIM flowsheet is available in Appendix D
\end{flushleft}





7.2





\begin{flushleft}
Comparisons
\end{flushleft}





\begin{flushleft}
** Note - All flow rates mentioned in the tables below are in kg/hr
\end{flushleft}





\begin{flushleft}
\newpage
DWSIM Flowsheet Simulation
\end{flushleft}





7.2.1





39





\begin{flushleft}
Ammonia Reactor
\end{flushleft}


\begin{flushleft}
Reactor Inlet
\end{flushleft}





\begin{flushleft}
Reactor Outlet
\end{flushleft}





\begin{flushleft}
Manual
\end{flushleft}





\begin{flushleft}
DWSIM
\end{flushleft}





\begin{flushleft}
Manual
\end{flushleft}





\begin{flushleft}
DWSIM
\end{flushleft}





\begin{flushleft}
NH3
\end{flushleft}





10963





11000





0





0





\begin{flushleft}
NO
\end{flushleft}





0





0





18573





18605





\begin{flushleft}
NO2
\end{flushleft}





0





0





0





0





\begin{flushleft}
HNO3
\end{flushleft}





0





0





0





0





\begin{flushleft}
H2 O
\end{flushleft}





0





0





17412





17452





\begin{flushleft}
N2
\end{flushleft}





128387





128397





128658





128775





\begin{flushleft}
O2
\end{flushleft}





39005





38993





13570





13568





\begin{flushleft}
Total
\end{flushleft}





178353





178400





178353





178400





\begin{flushleft}
Table 7.1: Mass Balance Comparison for Reactor
\end{flushleft}





7.2.2





\begin{flushleft}
Heat Exchangers
\end{flushleft}


\begin{flushleft}
Reactor Outlet
\end{flushleft}





\begin{flushleft}
Cooler Condenser Inlet
\end{flushleft}





\begin{flushleft}
Manual
\end{flushleft}





\begin{flushleft}
DWSIM
\end{flushleft}





\begin{flushleft}
Manual
\end{flushleft}





\begin{flushleft}
DWSIM
\end{flushleft}





\begin{flushleft}
NH3
\end{flushleft}





0





0





0





0





\begin{flushleft}
NO
\end{flushleft}





18573





18605





11608





12024





\begin{flushleft}
NO2
\end{flushleft}





0





0





10680





10044





\begin{flushleft}
HNO3
\end{flushleft}





0





0





0





0





\begin{flushleft}
H2 O
\end{flushleft}





17412





17452





17412





17452





\begin{flushleft}
N2
\end{flushleft}





128658





128775





128658





128775





\begin{flushleft}
O2
\end{flushleft}





13570





13568





9856





10069





\begin{flushleft}
Total
\end{flushleft}





178355





178400





178353





178400





\begin{flushleft}
Table 7.2: Mass Balance Comparison for Heat Exchanger network
\end{flushleft}





\begin{flushleft}
\newpage
DWSIM Flowsheet Simulation
\end{flushleft}





7.2.3





40





\begin{flushleft}
Absorption Column
\end{flushleft}


\begin{flushleft}
Gas Inlet
\end{flushleft}





\begin{flushleft}
Combined Outlet
\end{flushleft}





\begin{flushleft}
Manual
\end{flushleft}





\begin{flushleft}
DWSIM
\end{flushleft}





\begin{flushleft}
Manual
\end{flushleft}





\begin{flushleft}
DWSIM
\end{flushleft}





\begin{flushleft}
NH3
\end{flushleft}





0





0





0





0





\begin{flushleft}
NOx
\end{flushleft}





22288





22068





76





100





\begin{flushleft}
HNO3
\end{flushleft}





0





0





38858





38800





\begin{flushleft}
H2 O
\end{flushleft}





17412





17452





23816





24000





\begin{flushleft}
O2
\end{flushleft}





9856





10069





3307





3500





\begin{flushleft}
Total Inlet/Outlet Flow rate
\end{flushleft}





178355





178400





178355





178400





\begin{flushleft}
Table 7.3: Mass Balance Comparison for Absorption Column
\end{flushleft}





\begin{flushleft}
\newpage
Chapter 8
\end{flushleft}





\begin{flushleft}
Sensitivity Analysis
\end{flushleft}


8.1





\begin{flushleft}
Reactor Analysis
\end{flushleft}





\begin{flushleft}
Figure 8.1: Sensitivity Analysis for Reactor
\end{flushleft}





\begin{flushleft}
The above results are produced for the operation of the reactor at 4 barg pressure. To
\end{flushleft}


\begin{flushleft}
get high values of conversion for our reaction, the optimum temperature at which the
\end{flushleft}


\begin{flushleft}
reaction should be done is between 860 - 920◦ C. The highest possible conversion at 4barg
\end{flushleft}


\begin{flushleft}
is 96\%, which occurs when the reactor is operated at 890◦ C. Hence, 890◦ C was chosen
\end{flushleft}


\begin{flushleft}
as the operating temperature of the Ammonia Oxidation Reactor.
\end{flushleft}


41





\begin{flushleft}
\newpage
Sensitivity Analysis
\end{flushleft}





8.2





42





\begin{flushleft}
Heat Exchanger simulated as PFR
\end{flushleft}





\begin{flushleft}
Figure 8.2: Sensitivity Analysis for Heat Exchanger
\end{flushleft}





\begin{flushleft}
As explained in Chapter 7, modelling of reactive heat exchangers is difficult in any
\end{flushleft}


\begin{flushleft}
simulation software, hence to ease this process, they were modelled as PFRs. The above
\end{flushleft}


\begin{flushleft}
results are produced for the amount of NO2 in the Absorption Column inlet. We want
\end{flushleft}


\begin{flushleft}
to achieve a high conversion of NO to NO2 in the heat exchangers, but also want to
\end{flushleft}


\begin{flushleft}
reduce the cost involved in doing so. On increasing the volume of the PFR, the mole
\end{flushleft}


\begin{flushleft}
fraction of NO2 increases to a certain value, and becomes constant after around 0.9 m3 .
\end{flushleft}


\begin{flushleft}
Hence, a volume of 0.9 m3 is used for this particular heat exchanger, to give us optimum
\end{flushleft}


\begin{flushleft}
operation.
\end{flushleft}





\begin{flushleft}
\newpage
Sensitivity Analysis
\end{flushleft}





8.3





43





\begin{flushleft}
Absorption Column Analysis
\end{flushleft}





\begin{flushleft}
Figure 8.3: Sensitivity Analysis for Absorption Column - 1
\end{flushleft}





\begin{flushleft}
Figure 8.4: Sensitivity Analysis for Absorption Column - 2
\end{flushleft}





\begin{flushleft}
The above analysis compares the Nitric Acid Solution Flow rate and Concentration as
\end{flushleft}


\begin{flushleft}
a function of extra added air. The minimum amount of extra O2 required is 1kg/s
\end{flushleft}


\begin{flushleft}
which provides us with the best results, the selected O2 rate was set at 1.25 kg/s, which
\end{flushleft}





\begin{flushleft}
\newpage
Sensitivity Analysis
\end{flushleft}





44





\begin{flushleft}
provided us with the least amount of NOX in the tail gas as well. The mentioned
\end{flushleft}


\begin{flushleft}
comparisons are for the said extra O2 values, which are obtained from the secondary
\end{flushleft}


\begin{flushleft}
air, and can be manipulated using the split ratio and intake of air.
\end{flushleft}





\begin{flushleft}
Another important parameter to take care of while designing the absorption column
\end{flushleft}


\begin{flushleft}
is the total pressure drop across the column. Since we want to use the pressurised tail
\end{flushleft}


\begin{flushleft}
gas to generate energy, low pressure drops are desired. The pressure drop is a function
\end{flushleft}


\begin{flushleft}
of the diameter of the column, which in turn is a function of the tray spacing. Lower
\end{flushleft}


\begin{flushleft}
spacing (0.3-0.6 m) require smaller diameters and hence a relatively lower capital cost,
\end{flushleft}


\begin{flushleft}
but cause a larger pressure drop. Whereas larger spacing (1.0-1.2 m) require larger
\end{flushleft}


\begin{flushleft}
diameters but cause smaller pressure drops, helping us reduce our operating costs by
\end{flushleft}


\begin{flushleft}
providing higher amounts of energy. Hence, a diameter of 5 m, with spacings of 1.0 m
\end{flushleft}


\begin{flushleft}
in the bottom part (trays 1-13) and 1.2 m in the top part (trays 13-36) is chosen for the
\end{flushleft}


\begin{flushleft}
column.
\end{flushleft}





\begin{flushleft}
\newpage
Chapter 9
\end{flushleft}





\begin{flushleft}
Equipment Sizing and Costing
\end{flushleft}


\begin{flushleft}
In this chapter, we size the major equipments used in the plant and also estimate the
\end{flushleft}


\begin{flushleft}
cost of each of the equipment. They are listed in the following sections. The following
\end{flushleft}


\begin{flushleft}
data is used in the cost estimation:
\end{flushleft}





\begin{flushleft}
$\bullet$ India location index w.r.t US = 0.7
\end{flushleft}


\begin{flushleft}
$\bullet$ Dollar to INR = 70
\end{flushleft}


\begin{flushleft}
$\bullet$ Cost Index - Chemical Engineering Plant Cost Index
\end{flushleft}


\begin{flushleft}
$\bullet$ All equipment cost are delivered cost (1.1*fob cost)
\end{flushleft}


\begin{flushleft}
$\bullet$ Direct and Indirect Costs in FCI are calculated using percentage normalisation
\end{flushleft}


\begin{flushleft}
from Peters et. al. [10]
\end{flushleft}


\begin{flushleft}
$\bullet$ Equipment Cost is estimated using Nomograms from Peters et. al. [10] and Cost
\end{flushleft}


\begin{flushleft}
relations from Walas et. al. [11]
\end{flushleft}


\begin{flushleft}
$\bullet$ All equipment are mapped to the average 2018 Cost Index - 603.1 (As of Dec 18),
\end{flushleft}


\begin{flushleft}
data acquired from ChemEngOnline Website[12]
\end{flushleft}





9.1





\begin{flushleft}
Ammonia Oxidation Reactor
\end{flushleft}





\begin{flushleft}
The Ammonia Oxidation Reactor is a shallow bed reactor, with a burner head and 90-10
\end{flushleft}


\begin{flushleft}
Pt-Rh gauzes as its main parts. Since Pt-Rh Gauzes have a lifetime of 6-8 months, only
\end{flushleft}


45





\begin{flushleft}
\newpage
Equipment Sizing and Costing
\end{flushleft}





46





\begin{flushleft}
first 24 months of the catalyst gauzes will be capitalised and the rest will be incorporated
\end{flushleft}


\begin{flushleft}
into the Operating Costs. The burner head is a pressure vessel head, with a conical shape
\end{flushleft}


\begin{flushleft}
and operating pressure of 5 bar abs. Since the burner head doesnt come under the direct
\end{flushleft}


\begin{flushleft}
heat of the reaction gases (890◦ C), we dont require any special material of construction
\end{flushleft}


\begin{flushleft}
for the said equipment.
\end{flushleft}





9.1.1





\begin{flushleft}
Diameter Calculation
\end{flushleft}





\begin{flushleft}
The diameter of the gauzes and that of the burner head are calculated using the following
\end{flushleft}


\begin{flushleft}
data provided by Krupp Uhde[13]
\end{flushleft}


\begin{flushleft}
1. Gauze Density - 400g Pt/m2 of gauze
\end{flushleft}


\begin{flushleft}
2. Amount of Nitrogen allowed on Gauze to provide desired conversion - 10 Tonnes
\end{flushleft}


\begin{flushleft}
per day of N/m2 of gauze
\end{flushleft}


\begin{flushleft}
3. Platinum required for given Nitrogen - 320g Pt/Tonne per day Nitrogen
\end{flushleft}


\begin{flushleft}
NH3 consumed = 3.056 kg NH3 /s = 0.17974 kmol NH3 /s = 179.74 mol N/s
\end{flushleft}


\begin{flushleft}
= 2.516 kg N/s = 219.608 tonnes N/day
\end{flushleft}


\begin{flushleft}
From 2., we get, Area required, A = 21.961 m2
\end{flushleft}


\begin{flushleft}
Hence, Diameter = 5.288 m
\end{flushleft}


\begin{flushleft}
From 3., we get, Pt required = 70.275 kg
\end{flushleft}


\begin{flushleft}
From 1., we get, Pt/gauze = 8.784 kg
\end{flushleft}


\begin{flushleft}
Hence, we get, Number of Gauzes = 70.275/8.784 = 8 gauzes
\end{flushleft}





9.1.2





\begin{flushleft}
Catalyst Costs
\end{flushleft}





\begin{flushleft}
Platinum and Rhodium prices were acquired from [14] and [15] respectively. Since the
\end{flushleft}


\begin{flushleft}
catalyst life is 6 months, we are using 2 batches of catalyst gauzes per year. We are also
\end{flushleft}


\begin{flushleft}
capitalising 24 months worth of catalyst.
\end{flushleft}


\begin{flushleft}
Catalyst Required / year
\end{flushleft}





\begin{flushleft}
Catalyst Cost / kg
\end{flushleft}





\begin{flushleft}
Catalyst Cost / year
\end{flushleft}





\begin{flushleft}
(kg)
\end{flushleft}





\begin{flushleft}
(USD)
\end{flushleft}





\begin{flushleft}
(INR)
\end{flushleft}





140.55





34,750





\begin{flushleft}
48.837 Lakhs
\end{flushleft}





\begin{flushleft}
Table 9.1: Catalyst Costs
\end{flushleft}





\begin{flushleft}
\newpage
Equipment Sizing and Costing
\end{flushleft}





47





\begin{flushleft}
Capitalised Catalyst Cost = 140.55 * 2 * 34750 * 70
\end{flushleft}


\begin{flushleft}
= INR 97,67,475 = INR 97.675 Lakhs
\end{flushleft}





9.1.3





\begin{flushleft}
Burner Head Calculation
\end{flushleft}





\begin{flushleft}
The burner head is vertical, conical vessel with the following specifications:
\end{flushleft}


\begin{flushleft}
Corrosion
\end{flushleft}


\begin{flushleft}
$\alpha$(◦ )
\end{flushleft}





\begin{flushleft}
Di (m)
\end{flushleft}





\begin{flushleft}
MoC
\end{flushleft}





\begin{flushleft}
Milling
\end{flushleft}





\begin{flushleft}
Allowance
\end{flushleft}





\begin{flushleft}
Tolerance
\end{flushleft}





\begin{flushleft}
(c)
\end{flushleft}





\begin{flushleft}
(m)
\end{flushleft}





\begin{flushleft}
0.5 mm
\end{flushleft}





12.5





\begin{flushleft}
Sa
\end{flushleft}





\begin{flushleft}
Design
\end{flushleft}





\begin{flushleft}
Design
\end{flushleft}





\begin{flushleft}
Temp.
\end{flushleft}





\begin{flushleft}
Pressure
\end{flushleft}





\begin{flushleft}
(C)
\end{flushleft}





\begin{flushleft}
(barg)(P)
\end{flushleft}





500





4.4





\begin{flushleft}
Weld Joint
\end{flushleft}





\begin{flushleft}
(MPa)
\end{flushleft}


45





5.3





\begin{flushleft}
SS 321
\end{flushleft}





190





\begin{flushleft}
Efficiency (E)
\end{flushleft}


1





\begin{flushleft}
Table 9.2: Burner Head Design Parameters
\end{flushleft}





\begin{flushleft}
According to the Internal Pressure Vessel calculations, to find the regulation thickness,
\end{flushleft}


\begin{flushleft}
t, referring to CL 407 Handouts [16],
\end{flushleft}


\begin{flushleft}
t=
\end{flushleft}





\begin{flushleft}
P Di
\end{flushleft}


\begin{flushleft}
2 cos $\alpha$[Sa E $-$ 0.6P ]
\end{flushleft}





\begin{flushleft}
Hence we get, t = 11.7 mm. Now to get the trec , we have,
\end{flushleft}


\begin{flushleft}
trec =
\end{flushleft}





\begin{flushleft}
t+c
\end{flushleft}


\begin{flushleft}
m = 14mm
\end{flushleft}


1 $-$ 100





\begin{flushleft}
Hence, weight of the Burner head = 6,336.2 lbs
\end{flushleft}


\begin{flushleft}
From Walas et. al. [11], we have Vertical Pressure Vessels cost, C as (The formula
\end{flushleft}


\begin{flushleft}
provides values for 1984 in US\$ with CE Cost Index = 315),
\end{flushleft}


\begin{flushleft}
C = FM Cb + Ca
\end{flushleft}


\begin{flushleft}
Cb = 1.672exp[9.1 + 0.2889(lnW ) + 0.04576(lnW )2 ]
\end{flushleft}


\begin{flushleft}
Ca = 480D0.7396 L0.7066
\end{flushleft}


\begin{flushleft}
where W - shell weight in lbs, D - Diameter in ft and L - Height in ft. On using the
\end{flushleft}


\begin{flushleft}
required values we get the following results:
\end{flushleft}





\begin{flushleft}
W (lbs)
\end{flushleft}


6,336.20





\begin{flushleft}
D
\end{flushleft}





\begin{flushleft}
L
\end{flushleft}





\begin{flushleft}
(ft)
\end{flushleft}





\begin{flushleft}
(ft)
\end{flushleft}





6.6





8.7





\begin{flushleft}
Cb
\end{flushleft}


23805.9





\begin{flushleft}
Ca
\end{flushleft}


4562.2





\begin{flushleft}
FM
\end{flushleft}





\begin{flushleft}
Cost
\end{flushleft}





\begin{flushleft}
Cost
\end{flushleft}





\begin{flushleft}
(SS 321)
\end{flushleft}





\begin{flushleft}
(\$, in 1984)
\end{flushleft}





\begin{flushleft}
(\$, in 2018)
\end{flushleft}





1.87





49,079.10





105,883.20





\begin{flushleft}
Cost (INR)
\end{flushleft}





\begin{flushleft}
Table 9.3: Burner Head Costing Parameters
\end{flushleft}





\begin{flushleft}
Hence, cost to construct the burner head = INR 72.36 Lakhs
\end{flushleft}





\begin{flushleft}
72.36 Lakhs
\end{flushleft}





\begin{flushleft}
\newpage
Equipment Sizing and Costing
\end{flushleft}





9.2





48





\begin{flushleft}
Storage Tanks
\end{flushleft}





\begin{flushleft}
We produce 62\% Nitric Acid solution at 1500 tonnes/day, which corresponds to 45.2
\end{flushleft}


\begin{flushleft}
m3 /h. Referring to Walas et. al. [11] we will require a Field Erected Storage Tank
\end{flushleft}


\begin{flushleft}
of height 20 m and diameter 33 m with a capacity of 4.5 million gallons, keeping an
\end{flushleft}


\begin{flushleft}
inventory of 15 days of Nitric Acid. Nitric Acid Solution is stored under ambient atmospheric conditions. We also require an Ammonia Storage Tank to store our Ammonia
\end{flushleft}


\begin{flushleft}
for a period of one month to account for imports and procurement of the raw material.
\end{flushleft}


\begin{flushleft}
Ammonia is stored at -33◦ C and 1 atm abs. For this, special Double walled Tanks are
\end{flushleft}


\begin{flushleft}
constructed on site with a refrigeration unit accompanying the same. Ammonia required
\end{flushleft}


\begin{flushleft}
is 267 tonnes/day, which corresponds to 16.5 m3 /h. All the calculations include pumps
\end{flushleft}


\begin{flushleft}
installed in the tank farm as a factor of the total equipment cost [13]. Referring to
\end{flushleft}


\begin{flushleft}
Fertilizers Manual [17], we will require a LT Carbon Steel Refrigerated Double Walled
\end{flushleft}


\begin{flushleft}
Storage Tank, of height 20 m and diameter 28 m with a capacity of 3.25 million gallons.
\end{flushleft}


\begin{flushleft}
Suggestions for inventory were given by Krupp Uhde [13]. Referring Fertilizers Manual
\end{flushleft}


\begin{flushleft}
[17], Refrigerated Double Walled Ammonia Storage Tank incurred a capital cost of \$
\end{flushleft}


\begin{flushleft}
3.22 million in 1992-1994. Using CE Cost Index of 360 for the said period, multiplying
\end{flushleft}


\begin{flushleft}
CE Cost Index of 2018, 603.1, and multiplying \$ to INR conversion factor of 70 and 1.1
\end{flushleft}


\begin{flushleft}
we get,
\end{flushleft}


\begin{flushleft}
Refrigerated Ammonia Storage Cost = INR 41.52 Crores
\end{flushleft}


\begin{flushleft}
The following relation is used for costing of Field Erected Storage Tanks (C calculated
\end{flushleft}


\begin{flushleft}
in \$, all values for 1984 with CE Cost Index = 315):
\end{flushleft}


\begin{flushleft}
C = FM exp[11.662 $-$ 0.6104(lnV ) + 0.04536(lnV )2 ]
\end{flushleft}


\begin{flushleft}
where V - Volume in Gallons and FM - MoC Factor
\end{flushleft}


\begin{flushleft}
Temp.
\end{flushleft}


\begin{flushleft}
Storage
\end{flushleft}





\begin{flushleft}
Dim.
\end{flushleft}





\begin{flushleft}
Volume
\end{flushleft}


\begin{flushleft}
And
\end{flushleft}





\begin{flushleft}
Tank
\end{flushleft}





\begin{flushleft}
(m)
\end{flushleft}





\begin{flushleft}
MoC
\end{flushleft}





\begin{flushleft}
FM
\end{flushleft}





\begin{flushleft}
(gal)
\end{flushleft}





\begin{flushleft}
C
\end{flushleft}





\begin{flushleft}
C
\end{flushleft}





(\$, 1984)





(\$, 2018)





1,019,978.60





1,952,854.30





\begin{flushleft}
Cost (INR)
\end{flushleft}





\begin{flushleft}
Pressure
\end{flushleft}


\begin{flushleft}
Nitric Acid
\end{flushleft}





\begin{flushleft}
D = 33
\end{flushleft}





\begin{flushleft}
T = 30◦ C
\end{flushleft}





4.52





\begin{flushleft}
Storage
\end{flushleft}





\begin{flushleft}
H = 20
\end{flushleft}





\begin{flushleft}
P = 1 atm
\end{flushleft}





\begin{flushleft}
million
\end{flushleft}





\begin{flushleft}
SS 304L
\end{flushleft}





2.4





\begin{flushleft}
Table 9.4: Nitric Acid Storage Tank Costing
\end{flushleft}





\begin{flushleft}
Hence, the total cost required for Storage Tanks = INR 56.56 Crores
\end{flushleft}





\begin{flushleft}
15.04 Crores
\end{flushleft}





\begin{flushleft}
\newpage
Equipment Sizing and Costing
\end{flushleft}





9.3





49





\begin{flushleft}
Compressors
\end{flushleft}





\begin{flushleft}
There are two compressors in the plant - C101 to compress the Air feed from ambient
\end{flushleft}


\begin{flushleft}
conditions to 5 bar abs before the Ammonia Oxidation Reactor and C102 to compress
\end{flushleft}


\begin{flushleft}
the oxidised NOX gases to a pressure of 12 bar, for the high pressure absorption column.
\end{flushleft}


\begin{flushleft}
Both the compressors are Two Stage Centrifugal Compressors with turbines, with the
\end{flushleft}


\begin{flushleft}
Air Compressor, C101 having a compression ratio of 2.236 and the NOX Compressor,
\end{flushleft}


\begin{flushleft}
C102 having a compression ratio of 1.549. The choice to use a multi-stage compressor
\end{flushleft}


\begin{flushleft}
was made because the temperature increase due to compression in one single stage
\end{flushleft}


\begin{flushleft}
could have damaged the compressor in both the cases and also due to the fact that the
\end{flushleft}


\begin{flushleft}
power requirements and the flow rates were in the range of the operating conditions for
\end{flushleft}


\begin{flushleft}
centrifugal compressors. [13]
\end{flushleft}


\begin{flushleft}
Calculations are shown for the first stage of the C101, Air Compressor, with Air coming
\end{flushleft}


\begin{flushleft}
at 189.06 tonnes/hour.
\end{flushleft}


\begin{flushleft}
Power = Work rate / Isentropic Efficiency = W / $\eta$
\end{flushleft}


\begin{flushleft}
W = P1 V 1
\end{flushleft}





\begin{flushleft}
n
\end{flushleft}


\begin{flushleft}
P2 n$-$1
\end{flushleft}


\begin{flushleft}
[( ) n $-$ 1]
\end{flushleft}


\begin{flushleft}
n $-$ 1 P1
\end{flushleft}





\begin{flushleft}
P1 - Inlet Pressure = 1 bar abs
\end{flushleft}


\begin{flushleft}
P2 - Outlet Pressure = 2.266 bar abs
\end{flushleft}


\begin{flushleft}
V1 - Volumetric Flow = 49.21 m3 /s
\end{flushleft}


\begin{flushleft}
n - Heat Capacity Ratio = 1.4
\end{flushleft}


\begin{flushleft}
Isentropic Efficiency, $\eta$ = 0.85 [13]
\end{flushleft}


\begin{flushleft}
On solving the given equation, we get, Power = 5307.05 kW
\end{flushleft}


\begin{flushleft}
Similar calculations for other stages and compressors yield the following results:
\end{flushleft}





\begin{flushleft}
Compressor
\end{flushleft}





\begin{flushleft}
Stage
\end{flushleft}





\begin{flushleft}
Inlet Pressure
\end{flushleft}





\begin{flushleft}
Outlet Pressure
\end{flushleft}





\begin{flushleft}
Power Required
\end{flushleft}





\begin{flushleft}
(bar abs)
\end{flushleft}





\begin{flushleft}
(bar abs)
\end{flushleft}





\begin{flushleft}
(kW)
\end{flushleft}





\begin{flushleft}
C101
\end{flushleft}





1





1





2.236





5237.5





\begin{flushleft}
C101
\end{flushleft}





2





2.236





5





7042.9





\begin{flushleft}
C102
\end{flushleft}





1





5





7.745





2037.2





\begin{flushleft}
C102
\end{flushleft}





2





7.745





12





2350.1





\begin{flushleft}
Table 9.5: Compressor Power Calculation
\end{flushleft}





\begin{flushleft}
\newpage
Equipment Sizing and Costing
\end{flushleft}





50





\begin{flushleft}
For cost estimation, Nomograms from Peters et. al. [10] are used. Referring the nomograms, we get the following results for the compressor cost estimation:
\end{flushleft}


\begin{flushleft}
** Note that both the compressors are Two Stage Centrifugal Compressors
\end{flushleft}


\begin{flushleft}
with Turbines
\end{flushleft}





\begin{flushleft}
Equip.
\end{flushleft}


\begin{flushleft}
No. -
\end{flushleft}





\begin{flushleft}
Power
\end{flushleft}


\begin{flushleft}
Pin - Pout
\end{flushleft}





\begin{flushleft}
Process
\end{flushleft}





\begin{flushleft}
(bar abs)
\end{flushleft}





\begin{flushleft}
Fluid
\end{flushleft}





\begin{flushleft}
Required
\end{flushleft}





\begin{flushleft}
Stage
\end{flushleft}


\begin{flushleft}
C101 - 1
\end{flushleft}





\begin{flushleft}
MoC
\end{flushleft}





\begin{flushleft}
Cost
\end{flushleft}





\begin{flushleft}
Cost
\end{flushleft}





(\$, 2002)





(\$, 2018)





\begin{flushleft}
3.1 million
\end{flushleft}





\begin{flushleft}
4.789 million
\end{flushleft}





\begin{flushleft}
Cost INR
\end{flushleft}





\begin{flushleft}
(kW)
\end{flushleft}


\begin{flushleft}
P1 = 1
\end{flushleft}





\begin{flushleft}
Air
\end{flushleft}





5237.5





\begin{flushleft}
CS
\end{flushleft}





\begin{flushleft}
P2 = 2.236
\end{flushleft}


\begin{flushleft}
C101 - 2
\end{flushleft}





\begin{flushleft}
P1 = 2.236
\end{flushleft}





\begin{flushleft}
Crores
\end{flushleft}


\begin{flushleft}
Air
\end{flushleft}





7042.9





\begin{flushleft}
CS
\end{flushleft}





\begin{flushleft}
4.2 million
\end{flushleft}





\begin{flushleft}
6.488 million
\end{flushleft}





\begin{flushleft}
P2 = 5
\end{flushleft}


\begin{flushleft}
C102 - 1
\end{flushleft}





\begin{flushleft}
P1 = 5
\end{flushleft}





\begin{flushleft}
P1 = 7.745
\end{flushleft}





49.96


\begin{flushleft}
Crores
\end{flushleft}





\begin{flushleft}
NOX Gas
\end{flushleft}





2037.2





\begin{flushleft}
SS 304L
\end{flushleft}





\begin{flushleft}
1.4 million
\end{flushleft}





\begin{flushleft}
2.163 million
\end{flushleft}





\begin{flushleft}
P2 = 7.745
\end{flushleft}


\begin{flushleft}
C102 - 2
\end{flushleft}





36.88





16.65


\begin{flushleft}
Crores
\end{flushleft}





\begin{flushleft}
NOX Gas
\end{flushleft}





2350.1





\begin{flushleft}
SS 304L
\end{flushleft}





\begin{flushleft}
1.6 million
\end{flushleft}





\begin{flushleft}
2.472 million
\end{flushleft}





\begin{flushleft}
P2 = 12
\end{flushleft}





19.04


\begin{flushleft}
Crores
\end{flushleft}





\begin{flushleft}
Table 9.6: Costing of Multi Stage Compressors with Turbines
\end{flushleft}





\begin{flushleft}
Hence, total cost required for Compressors = INR 122.53 Crores
\end{flushleft}





9.4





\begin{flushleft}
Pumps
\end{flushleft}





\begin{flushleft}
There are a total of 6 estimated pumps in the plant. Including standby pumps, we have
\end{flushleft}


\begin{flushleft}
a total of 12 pumps. Out of these 12, 6 are part of the cooling circuit of absorption
\end{flushleft}


\begin{flushleft}
column and their cost is incorporated into the costing of the absorption column using
\end{flushleft}


\begin{flushleft}
the data provided by Krupp Uhde [13]. Out of the remaining pumps, one set is used to
\end{flushleft}


\begin{flushleft}
pump liquid ammonia into the plant, one set is used to pump the acid condensate to the
\end{flushleft}


\begin{flushleft}
absorption column and the last set is used to pump the required water into the boiler for
\end{flushleft}


\begin{flushleft}
production of steam. Sample calculation for the liquid Ammonia pump is shown below.
\end{flushleft}


\begin{flushleft}
Liquid Ammonia is fed at 19 bar to the Ammonia evaporator from the Ammonia Storage,
\end{flushleft}


\begin{flushleft}
which stores ammonia at 1 bar and -33◦ C. Since the temperature is so low, LT Carbon
\end{flushleft}


\begin{flushleft}
Steel is used as the MoC of this pump.
\end{flushleft}


\begin{flushleft}
Flow rate = 3.056 kg/s; Pressure difference = 18 bar; Density of liquid = 610 kg/m3
\end{flushleft}


\begin{flushleft}
We know that, Pump Work,
\end{flushleft}





\begin{flushleft}
\newpage
Equipment Sizing and Costing
\end{flushleft}





51





\begin{flushleft}
WP =
\end{flushleft}





\begin{flushleft}
∆Pf
\end{flushleft}


\begin{flushleft}
∆P
\end{flushleft}


\begin{flushleft}
+ g∆z +
\end{flushleft}


\begin{flushleft}
$\rho$
\end{flushleft}


\begin{flushleft}
$\rho$
\end{flushleft}





\begin{flushleft}
Hence we get, Pump Work, WP = 2.992 kJ/kg
\end{flushleft}


\begin{flushleft}
Power required = WP * Flow Rate/pump efficiency(assumed to be 0.85) = 10.757 kW
\end{flushleft}


\begin{flushleft}
All pumps used here fall under the zone of Centrifugal Pump considering the flowrate
\end{flushleft}


\begin{flushleft}
and the head required. The following relations from Walas et. al. [11] are used to
\end{flushleft}


\begin{flushleft}
estimate individual pump costs, C (Calculated values in \$ for 1984, with CE Cost Index
\end{flushleft}


= 315):


\begin{flushleft}
C = FM FT Cb , base cast $-$ iron, 3550 rpm, V SC
\end{flushleft}


$\surd$


$\surd$


\begin{flushleft}
Cb = 3exp[8.833 + 0.6019(lnQ H) + 0.0519(lnQ H)2 ]
\end{flushleft}


$\surd$


$\surd$


\begin{flushleft}
FT = exp[b1 + b2 (lnQ H) + b3 (lnQ H)2 ]
\end{flushleft}


\begin{flushleft}
Type
\end{flushleft}





\begin{flushleft}
b1
\end{flushleft}





\begin{flushleft}
b2
\end{flushleft}





\begin{flushleft}
b3
\end{flushleft}





\begin{flushleft}
One-stage, 1750 rpm, VSC
\end{flushleft}





5.1029





-1.2217





0.0771





\begin{flushleft}
One-stage, 3550 rpm, HSC
\end{flushleft}





0.0632





0.2744





-0.0253





\begin{flushleft}
One-stage, 1750 rpm, HSC
\end{flushleft}





2.029





-0.2371





0.0102





\begin{flushleft}
Two-stage, 3550 rpm, HSC
\end{flushleft}





13.7321





-2.8304





0.1542





\begin{flushleft}
Multistage, 3550 rpm, HSC
\end{flushleft}





9.8849





-1.6164





0.0834





\begin{flushleft}
Table 9.7: Pump Costing Factors
\end{flushleft}





\begin{flushleft}
Flow Range
\end{flushleft}





\begin{flushleft}
Head Range
\end{flushleft}





\begin{flushleft}
HP
\end{flushleft}





\begin{flushleft}
(gpm)
\end{flushleft}





\begin{flushleft}
(ft)
\end{flushleft}





\begin{flushleft}
(max)
\end{flushleft}





\begin{flushleft}
One-stage, 3550 rpm, VSC
\end{flushleft}





50-900





50-400





75





\begin{flushleft}
One-stage, 1750 rpm, VSC
\end{flushleft}





50-3500





50-200





200





\begin{flushleft}
One-stage, 3550 rpm, HSC
\end{flushleft}





100-1500





100-450





150





\begin{flushleft}
One-stage, 1750 rpm, HSC
\end{flushleft}





250-5000





50-500





250





\begin{flushleft}
Two-stage, 3550 rpm, HSC
\end{flushleft}





50-1100





300-1100





250





\begin{flushleft}
Multi-stage, 3550 rpm, HSC
\end{flushleft}





100-1500





650-3200





1450





\begin{flushleft}
Type
\end{flushleft}





\begin{flushleft}
Table 9.8: Pump Factors
\end{flushleft}





\begin{flushleft}
Based on the tables and cost relations provided above, we get the following results for
\end{flushleft}


\begin{flushleft}
the pump cost estimations:
\end{flushleft}





\begin{flushleft}
\newpage
Equipment Sizing and Costing
\end{flushleft}





52





\begin{flushleft}
** Note - All the above mentioned Pumps are Centrifugal pumps. Cost
\end{flushleft}


\begin{flushleft}
include the costing of both A/B pumps for each set
\end{flushleft}





\begin{flushleft}
Flow
\end{flushleft}


\begin{flushleft}
Pump
\end{flushleft}





\begin{flushleft}
Process
\end{flushleft}





\begin{flushleft}
Head
\end{flushleft}





\begin{flushleft}
Type -
\end{flushleft}





\begin{flushleft}
Cost
\end{flushleft}





\begin{flushleft}
Cost
\end{flushleft}





\begin{flushleft}
(ft)
\end{flushleft}





\begin{flushleft}
MoC
\end{flushleft}





(\$, 1984)





(\$, 2018)





17,233.40





32,995.10





\begin{flushleft}
25.41 Lakhs
\end{flushleft}





17,495.30





33,496.50





\begin{flushleft}
25.80 Lakhs
\end{flushleft}





7,721.50





14,783.60





\begin{flushleft}
11.39 Lakhs
\end{flushleft}





\begin{flushleft}
rate
\end{flushleft}


\begin{flushleft}
Fluid
\end{flushleft}





\begin{flushleft}
Cost INR
\end{flushleft}





\begin{flushleft}
(gpm)
\end{flushleft}


\begin{flushleft}
Two-Stage,
\end{flushleft}


\begin{flushleft}
P101
\end{flushleft}





\begin{flushleft}
Liquid
\end{flushleft}





\begin{flushleft}
(A/B)
\end{flushleft}





\begin{flushleft}
Ammonia
\end{flushleft}





79.5





1001.2





\begin{flushleft}
3550 rpm,
\end{flushleft}


\begin{flushleft}
HSC - CS
\end{flushleft}


\begin{flushleft}
One-Stage,
\end{flushleft}





\begin{flushleft}
P102
\end{flushleft}





\begin{flushleft}
Nitric
\end{flushleft}





\begin{flushleft}
(A/B)
\end{flushleft}





\begin{flushleft}
Acid
\end{flushleft}





127.7





250.1





\begin{flushleft}
3550 rpm,
\end{flushleft}


\begin{flushleft}
HSC - SS 304L
\end{flushleft}





\begin{flushleft}
Boiler
\end{flushleft}


\begin{flushleft}
P103
\end{flushleft}


\begin{flushleft}
Feed
\end{flushleft}





\begin{flushleft}
One-Stage,
\end{flushleft}


52.6





322.5





\begin{flushleft}
3550 rpm,
\end{flushleft}





\begin{flushleft}
(A/B)
\end{flushleft}


\begin{flushleft}
Water
\end{flushleft}





\begin{flushleft}
VSC - CS
\end{flushleft}





\begin{flushleft}
Table 9.9: Costing of Centrifugal Pumps
\end{flushleft}





\begin{flushleft}
Hence, the total cost required for Pumps = INR 62.60 Lakhs
\end{flushleft}





9.5





\begin{flushleft}
Heat Exchangers
\end{flushleft}





\begin{flushleft}
There are a total of 8 Heat Exchangers in the plant. All the heat exchangers are Shell
\end{flushleft}


\begin{flushleft}
and Tube Heat Exchangers, as cross checked with Krupp Uhde [13] and DFPCL [7].
\end{flushleft}


\begin{flushleft}
The waste heat boiler is sized below.
\end{flushleft}


\begin{flushleft}
Cooling Agent - Water at 80◦ C, 6 bar
\end{flushleft}


\begin{flushleft}
NOX Gas from Reactor is cooled from 890◦ C to 427◦ C
\end{flushleft}


\begin{flushleft}
Water boils and is converted to steam, which is superheated to 380◦ C
\end{flushleft}


\begin{flushleft}
TLMTD = 405.4◦ C
\end{flushleft}


\begin{flushleft}
Overall Heat Transfer Coefficient for Gas-Water Heat Exchange, U = 150 W/(m2 .K)
\end{flushleft}


\begin{flushleft}
Q = 26063.9 kW and we know that, Q = U A ∆TLMTD . Hence, we get,
\end{flushleft}


\begin{flushleft}
A = Q / (U * TLMTD ) = 428.62 m2
\end{flushleft}





\begin{flushleft}
\newpage
Equipment Sizing and Costing
\end{flushleft}





53





\begin{flushleft}
The cost estimate for heat exchangers is performed using the nomograms available in
\end{flushleft}


\begin{flushleft}
Peters et. al. [10], where the cost from nomogram was in \$ for 2002, with CE Cost
\end{flushleft}


\begin{flushleft}
Index = 390.4:
\end{flushleft}





\begin{flushleft}
HX No.
\end{flushleft}





\begin{flushleft}
Hot Stream
\end{flushleft}





\begin{flushleft}
HE101-
\end{flushleft}





\begin{flushleft}
HE101-
\end{flushleft}





\begin{flushleft}
AE
\end{flushleft}





\begin{flushleft}
AS
\end{flushleft}





\begin{flushleft}
Cooling
\end{flushleft}





\begin{flushleft}
WHB
\end{flushleft}





\begin{flushleft}
HE102
\end{flushleft}





\begin{flushleft}
HE103
\end{flushleft}





\begin{flushleft}
HE104
\end{flushleft}





\begin{flushleft}
HE105
\end{flushleft}





\begin{flushleft}
HE106
\end{flushleft}





\begin{flushleft}
LP Steam
\end{flushleft}





\begin{flushleft}
NO Gas
\end{flushleft}





\begin{flushleft}
NO Gas
\end{flushleft}





\begin{flushleft}
NO Gas
\end{flushleft}





\begin{flushleft}
NO Gas
\end{flushleft}





\begin{flushleft}
NO Gas
\end{flushleft}





\begin{flushleft}
Air
\end{flushleft}





\begin{flushleft}
Water
\end{flushleft}


\begin{flushleft}
Thot, in (K)
\end{flushleft}





306





423





1123





700





568





457





530





493





\begin{flushleft}
Thot, out (K)
\end{flushleft}





280





372





700





568





457





323





323





323





\begin{flushleft}
Ammonia
\end{flushleft}





\begin{flushleft}
Ammonia
\end{flushleft}





\begin{flushleft}
Tail Gas
\end{flushleft}





\begin{flushleft}
BFW
\end{flushleft}





\begin{flushleft}
Cooling
\end{flushleft}





\begin{flushleft}
Cooling
\end{flushleft}





\begin{flushleft}
Water
\end{flushleft}





\begin{flushleft}
Water
\end{flushleft}





\begin{flushleft}
Cold
\end{flushleft}


\begin{flushleft}
Stream
\end{flushleft}





\begin{flushleft}
BFW/
\end{flushleft}


\begin{flushleft}
Steam
\end{flushleft}





\begin{flushleft}
Tail Gas
\end{flushleft}





\begin{flushleft}
Tcold, in (K)
\end{flushleft}





240





283





353





403





313





306





306





283





\begin{flushleft}
Tcold, out (K)
\end{flushleft}





283





343





653





623





353





330





330





403





\begin{flushleft}
Q (kW)
\end{flushleft}





4472





412.7





26063.9





8667.8





7282.3





13883.7





5765.7





4675.1





\begin{flushleft}
(W/m2 .K)
\end{flushleft}





150





300





150





95





150





150





150





95





970.5





16.3





428.7





790.2





274.1





1692.1





517.8





798.2





\begin{flushleft}
Ammonia
\end{flushleft}





\begin{flushleft}
LP Steam
\end{flushleft}





\begin{flushleft}
NO Gas
\end{flushleft}





\begin{flushleft}
Tail Gas
\end{flushleft}





\begin{flushleft}
BFW
\end{flushleft}





\begin{flushleft}
NO Gas
\end{flushleft}





\begin{flushleft}
NO Gas
\end{flushleft}





\begin{flushleft}
Air
\end{flushleft}





\begin{flushleft}
LT CS
\end{flushleft}





\begin{flushleft}
CS
\end{flushleft}





\begin{flushleft}
SS 321
\end{flushleft}





\begin{flushleft}
SS 321
\end{flushleft}





\begin{flushleft}
SS 304L
\end{flushleft}





\begin{flushleft}
SS 304L
\end{flushleft}





\begin{flushleft}
SS 304L
\end{flushleft}





\begin{flushleft}
2Re10
\end{flushleft}





\begin{flushleft}
2Re10
\end{flushleft}





\begin{flushleft}
Cooling
\end{flushleft}





\begin{flushleft}
Cooling
\end{flushleft}





\begin{flushleft}
Water
\end{flushleft}





\begin{flushleft}
Water
\end{flushleft}





\begin{flushleft}
SS 304L
\end{flushleft}





\begin{flushleft}
SS 304L
\end{flushleft}





\begin{flushleft}
2Re10
\end{flushleft}





\begin{flushleft}
2Re10
\end{flushleft}





5





12





\begin{flushleft}
U
\end{flushleft}





\begin{flushleft}
Area
\end{flushleft}





\begin{flushleft}
(m2 )
\end{flushleft}





\begin{flushleft}
Shell Side
\end{flushleft}


\begin{flushleft}
Fluid
\end{flushleft}


\begin{flushleft}
Shell Side
\end{flushleft}


\begin{flushleft}
MoC
\end{flushleft}


\begin{flushleft}
Tube Side
\end{flushleft}





\begin{flushleft}
Cooling
\end{flushleft}





\begin{flushleft}
Fluid
\end{flushleft}





\begin{flushleft}
Water
\end{flushleft}





\begin{flushleft}
Tube Side
\end{flushleft}





\begin{flushleft}
LT CS
\end{flushleft}





\begin{flushleft}
Ammonia
\end{flushleft}





\begin{flushleft}
BFW/
\end{flushleft}





\begin{flushleft}
NO Gas
\end{flushleft}





\begin{flushleft}
NO Gas
\end{flushleft}





\begin{flushleft}
Steam
\end{flushleft}


\begin{flushleft}
SS 321
\end{flushleft}





\begin{flushleft}
MoC
\end{flushleft}





\begin{flushleft}
SA 192/
\end{flushleft}





\begin{flushleft}
SS 304L
\end{flushleft}





\begin{flushleft}
SS 304L
\end{flushleft}





\begin{flushleft}
SA 213
\end{flushleft}





\begin{flushleft}
Pressure
\end{flushleft}





6.5





5





5





5





5





\begin{flushleft}
(bar abs)
\end{flushleft}


\begin{flushleft}
Cost
\end{flushleft}





\begin{flushleft}
SS 304L
\end{flushleft}





\begin{flushleft}
Tail Gas
\end{flushleft}





\begin{flushleft}
SS 304L
\end{flushleft}


\begin{flushleft}
Tube:12
\end{flushleft}


\begin{flushleft}
Shell:5
\end{flushleft}





54,150





5,054





197,534





191,301





60,078





704,968





346,418





177,650





83.7





7.8





305.2





295.5





92.8





1,089.10





535.2





274.4





0.65





0.061





2.35





2.28





0.72





8.39





4.12





2.11





\begin{flushleft}
Crores
\end{flushleft}





\begin{flushleft}
Crores
\end{flushleft}





\begin{flushleft}
Crores
\end{flushleft}





\begin{flushleft}
Crores
\end{flushleft}





\begin{flushleft}
Crores
\end{flushleft}





\begin{flushleft}
Crores
\end{flushleft}





\begin{flushleft}
Crores
\end{flushleft}





\begin{flushleft}
Crores
\end{flushleft}





(\$ , 2002)


\begin{flushleft}
Cost
\end{flushleft}


(1000\$, 2018)


\begin{flushleft}
Cost INR
\end{flushleft}





\begin{flushleft}
Table 9.10: Operating Parameters and Costing of Heat Exchangers
\end{flushleft}





\begin{flushleft}
** Note - SS 304L 2Re10 is a special material provided Sandvik Materials
\end{flushleft}


\begin{flushleft}
Technology. The cost factor of this material is used as 3X of SS 304L according to the data provided by Krupp Uhde. Also SA 192 and SA 213 are high
\end{flushleft}





\begin{flushleft}
\newpage
Equipment Sizing and Costing
\end{flushleft}





54





\begin{flushleft}
temperature corrosion resistant Alloys, for which cost factor used is 2X SS
\end{flushleft}


\begin{flushleft}
304L according to the data provided by Krupp Uhde [13].
\end{flushleft}


\begin{flushleft}
Hence, the total cost required for Heat Exchangers = INR 20.67 Crores
\end{flushleft}





9.6





\begin{flushleft}
Columns
\end{flushleft}





\begin{flushleft}
We have two Absorption Columns in our plant, one is a 47 m tall reactive absorption
\end{flushleft}


\begin{flushleft}
column and the second one is a 10 m tall bleaching column. Since the absorption column
\end{flushleft}


\begin{flushleft}
has a height greater than 10 m, we need to make thickness calculation based on stresses
\end{flushleft}


\begin{flushleft}
produced due to wind at heights. We can ignore this for the shorter bleaching column.
\end{flushleft}


\begin{flushleft}
We use internal pressure vessel and wind force calculations to find out the thickness
\end{flushleft}


\begin{flushleft}
required for the column body and heads.
\end{flushleft}





9.6.1





\begin{flushleft}
Shell Calculations
\end{flushleft}





\begin{flushleft}
Using Internal Pressure Vessel calculations, referring to CL 407 Handouts [16], we have,
\end{flushleft}


\begin{flushleft}
t=
\end{flushleft}





\begin{flushleft}
P Di
\end{flushleft}


\begin{flushleft}
2[Sa E $-$ 0.6P ]
\end{flushleft}





\begin{flushleft}
P = 12.1 bar, Di = 4m, Sa for MoC SS 304L = 180MPa, E = 0.9.Hence we get, t = 15
\end{flushleft}


\begin{flushleft}
mm. Now to get the trec , we have,
\end{flushleft}


\begin{flushleft}
trec =
\end{flushleft}





\begin{flushleft}
t+c
\end{flushleft}


\begin{flushleft}
m = 14mm
\end{flushleft}


1 $-$ 100





\begin{flushleft}
Using c = 1mm and m = 12.5, we get trec = 18.3 mm
\end{flushleft}


\begin{flushleft}
Both the columns have ellipsoidal heads, using details from CL 407 Handouts [16],
\end{flushleft}


\begin{flushleft}
similar calculations are done for the heads and the skirt. The details of the calculations
\end{flushleft}


\begin{flushleft}
are provided in the detailed design of the absorption column. Following the thickness
\end{flushleft}


\begin{flushleft}
calculations, weights of the columns were calculated, skirts and heads included.
\end{flushleft}


\begin{flushleft}
** Note - The bottom closure of the absorption column, A101 has a separate
\end{flushleft}


\begin{flushleft}
vessel attached inside, with its weight being 0.25 times that of the closure
\end{flushleft}


\begin{flushleft}
itself, according to the data provided by Krupp Uhde [13].
\end{flushleft}


\begin{flushleft}
** Note - The Absorption Column comes with a cooling system, inclusive of a
\end{flushleft}


\begin{flushleft}
piping network and 6 pumps. According to the data provided by Krupp Uhde
\end{flushleft}





\begin{flushleft}
\newpage
Equipment Sizing and Costing
\end{flushleft}





55





\begin{flushleft}
[13], once the entire column cost is calculated, including trays, multiplying a
\end{flushleft}


\begin{flushleft}
factor of 1.25 will give a rough estimate for the entire column cost including
\end{flushleft}


\begin{flushleft}
the cost of the cooling system.
\end{flushleft}


\begin{flushleft}
Nomograms from Peters et. al. [10] are used, where the cost from nomogram was in \$
\end{flushleft}


\begin{flushleft}
for 2002, with CE Cost Index = 390.4:
\end{flushleft}


\begin{flushleft}
A101 -
\end{flushleft}





\begin{flushleft}
B102 -
\end{flushleft}





\begin{flushleft}
Absorption Column
\end{flushleft}





\begin{flushleft}
Bleaching Column
\end{flushleft}





\begin{flushleft}
Height (m)
\end{flushleft}





44.6





9





\begin{flushleft}
Diameter (m)
\end{flushleft}





4





1.75





\begin{flushleft}
Tray Spacing
\end{flushleft}





\begin{flushleft}
1 m / 1.2 m
\end{flushleft}





\begin{flushleft}
0.6 m
\end{flushleft}





\begin{flushleft}
trec for Body (mm)
\end{flushleft}





18.3





4.55





\begin{flushleft}
trec for Heads (mm)
\end{flushleft}





18.24





4.54





\begin{flushleft}
trec for Skirt (mm)
\end{flushleft}





3.77





3.77





\begin{flushleft}
Skirt Height (m)
\end{flushleft}





3.4





1





\begin{flushleft}
NO-NOx Gases,
\end{flushleft}





\begin{flushleft}
Air, NOx Gases,
\end{flushleft}





\begin{flushleft}
Nitric Acid
\end{flushleft}





\begin{flushleft}
Nitric Acid
\end{flushleft}





11





4





10 - 50





30 - 50





12.1





4.4





80





80





\begin{flushleft}
MoC
\end{flushleft}





\begin{flushleft}
SS 304L
\end{flushleft}





\begin{flushleft}
SS 304L
\end{flushleft}





\begin{flushleft}
Weight (Tonnes)
\end{flushleft}





88.46





2.006





\begin{flushleft}
Cost (\$, 2002)
\end{flushleft}





700,000





52,000





\begin{flushleft}
Cost (\$, 2018)
\end{flushleft}





1,081,378





80,331





\begin{flushleft}
Cost INR
\end{flushleft}





\begin{flushleft}
8.33 Crores
\end{flushleft}





\begin{flushleft}
0.62 Crores
\end{flushleft}





\begin{flushleft}
Column
\end{flushleft}





\begin{flushleft}
Fluids Processed
\end{flushleft}


\begin{flushleft}
Operating Pressure
\end{flushleft}


\begin{flushleft}
(barg)
\end{flushleft}


\begin{flushleft}
Operating Temp.
\end{flushleft}


\begin{flushleft}
Range (◦ C)
\end{flushleft}


\begin{flushleft}
Design Pressure
\end{flushleft}


\begin{flushleft}
(barg)
\end{flushleft}


\begin{flushleft}
Design Temp.
\end{flushleft}


\begin{flushleft}
(◦ C)
\end{flushleft}





\begin{flushleft}
Table 9.11: Column Parameters and Costing
\end{flushleft}





\begin{flushleft}
\newpage
Equipment Sizing and Costing
\end{flushleft}





9.6.2





56





\begin{flushleft}
Tray Calculations
\end{flushleft}





\begin{flushleft}
Sieve trays are used in both the absorption columns due to their many benefits. Selection
\end{flushleft}


\begin{flushleft}
process, the tray specification and calculations for number of trays for absorption column
\end{flushleft}


\begin{flushleft}
are all specified in the detailed design of the absorption column.
\end{flushleft}


\begin{flushleft}
Nomograms from Peters et. al. [10] are used for cost estimation, where the cost from
\end{flushleft}


\begin{flushleft}
nomogram was in \$ for 2002, with CE Cost Index = 390.4:
\end{flushleft}





\begin{flushleft}
Column
\end{flushleft}





\begin{flushleft}
Diameter
\end{flushleft}





\begin{flushleft}
Number of
\end{flushleft}





\begin{flushleft}
(m)
\end{flushleft}





\begin{flushleft}
trays
\end{flushleft}





\begin{flushleft}
A101
\end{flushleft}





4





36





\begin{flushleft}
MoC
\end{flushleft}





\begin{flushleft}
SS 304L
\end{flushleft}





\begin{flushleft}
QF
\end{flushleft}





0.98





\begin{flushleft}
Per Tray
\end{flushleft}





\begin{flushleft}
Total Tray
\end{flushleft}





\begin{flushleft}
Cost
\end{flushleft}





\begin{flushleft}
Cost
\end{flushleft}





(\$, 2002)





(\$, 2018)





4,000.00





239,807.00





\begin{flushleft}
Cost INR
\end{flushleft}





1.68


\begin{flushleft}
Crores
\end{flushleft}





\begin{flushleft}
B102
\end{flushleft}





1.75





12





\begin{flushleft}
SS 304L
\end{flushleft}





1.4





1,400.00





0.30





42,823.00





\begin{flushleft}
Crores
\end{flushleft}





\begin{flushleft}
Table 9.12: Sieve Tray Costing
\end{flushleft}





\begin{flushleft}
Shell Costs
\end{flushleft}





\begin{flushleft}
Tray Costs
\end{flushleft}





\begin{flushleft}
Total Purchased Cost
\end{flushleft}





\begin{flushleft}
(INR, 2018)
\end{flushleft}





\begin{flushleft}
(INR, 2018)
\end{flushleft}





\begin{flushleft}
(INR, 2018)
\end{flushleft}





\begin{flushleft}
A101
\end{flushleft}





\begin{flushleft}
8.33 Crores
\end{flushleft}





\begin{flushleft}
1.68 Crores
\end{flushleft}





\begin{flushleft}
12.51 Crores
\end{flushleft}





\begin{flushleft}
B102
\end{flushleft}





\begin{flushleft}
0.62 Crores
\end{flushleft}





\begin{flushleft}
0.30 Crores
\end{flushleft}





\begin{flushleft}
0.92 Crores
\end{flushleft}





\begin{flushleft}
Column
\end{flushleft}





\begin{flushleft}
Table 9.13: Total Purchased Costs for Columns
\end{flushleft}





\begin{flushleft}
Hence, the total cost required for Columns, including the cooling system = INR 13.43
\end{flushleft}


\begin{flushleft}
Crores
\end{flushleft}





\begin{flushleft}
\newpage
Chapter 10
\end{flushleft}





\begin{flushleft}
Plant Economics
\end{flushleft}


\begin{flushleft}
This chapter focuses on an overview of the plant economics, by estimating the capital
\end{flushleft}


\begin{flushleft}
investments, production costs and finally the payback period for the Nitric Acid Plant.
\end{flushleft}





10.1





\begin{flushleft}
Major Equipment Costing
\end{flushleft}





\begin{flushleft}
The Dual Pressure Nitric Acid plant has many major equipments including storage
\end{flushleft}


\begin{flushleft}
tanks, compressors, pumps, heat exchangers, reactor and columns. We used costing
\end{flushleft}


\begin{flushleft}
relations provided by Peters et. al. [10] and Walas et. al. [11] to determine the total
\end{flushleft}


\begin{flushleft}
purchased cost of each of these equipments, using CE Cost index to find their costs in
\end{flushleft}


\begin{flushleft}
2018. A dollar to INR factor of 70 is used and an India location factor of 0.7 is also
\end{flushleft}


\begin{flushleft}
used. The following is the final result of our calculations in Chapter 9:
\end{flushleft}


\begin{flushleft}
Table 10.1: Total Purchased Equipment Cost
\end{flushleft}





\begin{flushleft}
Equipment
\end{flushleft}





\begin{flushleft}
Total Purchased Cost, INR (2018)
\end{flushleft}





\begin{flushleft}
Reactor
\end{flushleft}





\begin{flushleft}
1.700 Crores
\end{flushleft}





\begin{flushleft}
Storage Tanks
\end{flushleft}





\begin{flushleft}
56.56 Crores
\end{flushleft}





\begin{flushleft}
Compressors
\end{flushleft}





\begin{flushleft}
122.53 Crores
\end{flushleft}





\begin{flushleft}
Pumps
\end{flushleft}





\begin{flushleft}
62.60 Lakhs
\end{flushleft}





\begin{flushleft}
Heat Exchangers
\end{flushleft}





\begin{flushleft}
20.67 Crores
\end{flushleft}





\begin{flushleft}
Columns
\end{flushleft}





\begin{flushleft}
13.43 Crores
\end{flushleft}





\begin{flushleft}
Total Purchased Equipment Cost
\end{flushleft}





\begin{flushleft}
215.51 Crores
\end{flushleft}





57





\begin{flushleft}
\newpage
Plant Economics
\end{flushleft}





10.2





58





\begin{flushleft}
Capital Investment Estimation
\end{flushleft}





\begin{flushleft}
Total purchased equipment costs calculated in the prvious section are used to calculate
\end{flushleft}


\begin{flushleft}
the total capital investment in the plant. Working capital is calculated as 15\% of the
\end{flushleft}


\begin{flushleft}
total capital investment. The plant is considered to be a Fluid-Fluid Processing plant
\end{flushleft}


\begin{flushleft}
(F-FPP) and the percentage of delivered equipment cost is implemented for this cost
\end{flushleft}


\begin{flushleft}
estimation. The results of the estimation are provided in the table below:
\end{flushleft}





\begin{flushleft}
Direct Costs
\end{flushleft}





\begin{flushleft}
F-FPP
\end{flushleft}





\begin{flushleft}
Normalized \%
\end{flushleft}





\begin{flushleft}
Cost (Cr. INR)
\end{flushleft}





\begin{flushleft}
Purchase Equipment Delivered
\end{flushleft}





100





0.1984





215.51





\begin{flushleft}
Purchsed Equipment Installation
\end{flushleft}





47





0.0933





101.29





\begin{flushleft}
Instrumentation and controls
\end{flushleft}





36





0.0714





77.58





\begin{flushleft}
Piping
\end{flushleft}





68





0.1349





146.54





\begin{flushleft}
Electrical Systems
\end{flushleft}





11





0.0218





23.71





\begin{flushleft}
Buildings
\end{flushleft}





18





0.0357





38.79





\begin{flushleft}
Yard Improvements
\end{flushleft}





10





0.0198





21.55





\begin{flushleft}
Service Facilities
\end{flushleft}





65





0.1290





140.08





\begin{flushleft}
Land
\end{flushleft}





5





0.0099





10.78





\begin{flushleft}
Total direct plant cost
\end{flushleft}





360





0.7143





775.82





\begin{flushleft}
Engineering and Supervision
\end{flushleft}





33





0.0655





71.12





\begin{flushleft}
Construction Expenses
\end{flushleft}





41





0.0813





88.36





\begin{flushleft}
Legal Expenses
\end{flushleft}





4





0.0079





8.62





\begin{flushleft}
Contractor's Fee
\end{flushleft}





22





0.0437





47.41





\begin{flushleft}
Contingency
\end{flushleft}





44





0.0873





94.82





\begin{flushleft}
Total indirect plant cost
\end{flushleft}





144





0.2857





310.33





\begin{flushleft}
Fixed Capital Investment
\end{flushleft}





504





1.0000





1086.15





\begin{flushleft}
Indirect Costs
\end{flushleft}





\begin{flushleft}
Working Capital
\end{flushleft}





89





191.67





\begin{flushleft}
(15\% of total capital investment)
\end{flushleft}





\begin{flushleft}
Total Capital Investment
\end{flushleft}


\begin{flushleft}
Table 10.2: Total Capital Investment Estimation
\end{flushleft}





1277.82





\begin{flushleft}
\newpage
Plant Economics
\end{flushleft}





59





\begin{flushleft}
Peters et. al. [10] is used as the reference for this costing exercise, using the ratio factors
\end{flushleft}


\begin{flushleft}
provided by it for our cost estimation.
\end{flushleft}


\begin{flushleft}
Using an India Factor of 0.7, we get,
\end{flushleft}


\begin{flushleft}
Total Capital Investment = INR 894.48 Crores
\end{flushleft}





10.3





\begin{flushleft}
Raw Material Cost
\end{flushleft}





\begin{flushleft}
The estimation of the total product cost comes from raw material cost. We require
\end{flushleft}


\begin{flushleft}
Liquid Ammonia as a raw material, which is acquired via imports and from KRIBHCO,
\end{flushleft}


\begin{flushleft}
and we also require 140.55 kg of Pt-Rh catalyst gauzes every year, taken from Johnson
\end{flushleft}


\begin{flushleft}
Matthey India Pvt. Ltd. Ammonia cost is acquired from data provided by DFPCL [7].
\end{flushleft}


\begin{flushleft}
Raw material cost calculation is provided below:
\end{flushleft}


\begin{flushleft}
Raw Material
\end{flushleft}





\begin{flushleft}
Amount
\end{flushleft}





\begin{flushleft}
Cost/kg (\$/kg)
\end{flushleft}





\begin{flushleft}
Cost/year (\$/year)
\end{flushleft}





\begin{flushleft}
Ammonia
\end{flushleft}





\begin{flushleft}
267 tonnes/day
\end{flushleft}





0.429





\begin{flushleft}
37.35 Million
\end{flushleft}





\begin{flushleft}
Pt-Rh Gauze
\end{flushleft}





\begin{flushleft}
140.55 kg/year
\end{flushleft}





34,750





\begin{flushleft}
4.884 Million
\end{flushleft}





\begin{flushleft}
Total Raw Material Cost
\end{flushleft}





\begin{flushleft}
42.24 Million
\end{flushleft}





\begin{flushleft}
Table 10.3: Total Raw Material Cost Calculation
\end{flushleft}





\begin{flushleft}
Hence, the Total Raw Material Cost = \$ 42.24 Million / year
\end{flushleft}





10.4





\begin{flushleft}
Revenue Calculation
\end{flushleft}





\begin{flushleft}
The plant revenue is calculated from the untaxed prices of 62\% Weak Nitric Acid (WNA)
\end{flushleft}


\begin{flushleft}
Solution, sold at INR 20/kg, with 1\$ = INR 70, cost taken from data provided by DFPCL
\end{flushleft}


\begin{flushleft}
[7]. The results are tabulated below:
\end{flushleft}


\begin{flushleft}
Product
\end{flushleft}





\begin{flushleft}
Amount
\end{flushleft}





\begin{flushleft}
Cost/kg (\$/kg)
\end{flushleft}





\begin{flushleft}
Cost/year (\$/year)
\end{flushleft}





\begin{flushleft}
62\% WNA Solution
\end{flushleft}





\begin{flushleft}
1520 tonnes/day
\end{flushleft}





0.286





\begin{flushleft}
141.82 Million
\end{flushleft}





\begin{flushleft}
Table 10.4: Total Revenue Calculation
\end{flushleft}





\begin{flushleft}
Hence the Total Revenue = \$ 141.82 Million / year
\end{flushleft}





\begin{flushleft}
\newpage
Plant Economics
\end{flushleft}





10.5





60





\begin{flushleft}
Total Product Cost
\end{flushleft}





\begin{flushleft}
Total Product cost is calculated using the normalised percentage method from Peters
\end{flushleft}


\begin{flushleft}
et. al. [10].The result is reproduced in the following tabulated form:
\end{flushleft}


\begin{flushleft}
Category
\end{flushleft}





\begin{flushleft}
Factor
\end{flushleft}





\begin{flushleft}
Norm
\end{flushleft}





\begin{flushleft}
Cost (\$/yr)
\end{flushleft}





\begin{flushleft}
Raw material
\end{flushleft}





45





0.3460





\begin{flushleft}
42.23 Million
\end{flushleft}





\begin{flushleft}
Operating Labour
\end{flushleft}





10





0.0769





\begin{flushleft}
6.569 Million
\end{flushleft}





\begin{flushleft}
Supervising Labour
\end{flushleft}





3





0.0231





\begin{flushleft}
1.971 Million
\end{flushleft}





\begin{flushleft}
Utilities
\end{flushleft}





15





0.1153





\begin{flushleft}
14.08 Million
\end{flushleft}





\begin{flushleft}
Repairs and Maintenance
\end{flushleft}





7





0.0538





\begin{flushleft}
6.569 Million
\end{flushleft}





\begin{flushleft}
Operating supplies
\end{flushleft}





1.05





0.0081





\begin{flushleft}
0.986 Million
\end{flushleft}





\begin{flushleft}
Laboratory charges
\end{flushleft}





1.5





0.0115





\begin{flushleft}
1.408 Million
\end{flushleft}





\begin{flushleft}
Patents
\end{flushleft}





3





0.0231





\begin{flushleft}
2.815 Million
\end{flushleft}





\begin{flushleft}
Fixed charges
\end{flushleft}





15





0.1153





\begin{flushleft}
14.08 Million
\end{flushleft}





\begin{flushleft}
Plant overhead
\end{flushleft}





12





0.0923





\begin{flushleft}
11.26 Million
\end{flushleft}





\begin{flushleft}
Admin costs
\end{flushleft}





1.5





0.0115





\begin{flushleft}
0.985 Million
\end{flushleft}





\begin{flushleft}
Distribution and marketing
\end{flushleft}





11





0.0846





\begin{flushleft}
10.32 Million
\end{flushleft}





\begin{flushleft}
RnD cost
\end{flushleft}





5





0.0384





\begin{flushleft}
4.692 Million
\end{flushleft}





\begin{flushleft}
Total
\end{flushleft}





130





1.0000





\begin{flushleft}
117.965 Million
\end{flushleft}





\begin{flushleft}
Table 10.5: Total Product Cost Estimation
\end{flushleft}





\begin{flushleft}
Hence, the Total Product Cost = \$ 117.965 Million / year
\end{flushleft}


\begin{flushleft}
Please note that since the plant is setup in India, India factor of 0.7 is applied to
\end{flushleft}


\begin{flushleft}
Operating Labour, Supervising Labour and Admin costs.
\end{flushleft}





10.6





\begin{flushleft}
Payback Period
\end{flushleft}





\begin{flushleft}
The following assumptions were made while making the payback period calculations:
\end{flushleft}





\begin{flushleft}
$\bullet$ 90\% of FCI excluding the land costs is considered to be depreciable over a period
\end{flushleft}


\begin{flushleft}
of 10 years.
\end{flushleft}


\begin{flushleft}
$\bullet$ Average depreciation is calculated using linear depreciation model.
\end{flushleft}





\begin{flushleft}
\newpage
Plant Economics
\end{flushleft}





61





\begin{flushleft}
$\bullet$ Bank interest rate of 6\% over the total capital is also used.
\end{flushleft}


\begin{flushleft}
$\bullet$ Final Product and Raw Material costs are assumed to be constant for profit calculations.
\end{flushleft}


\begin{flushleft}
Total Depreciable FCI = INR 677.49 Crores
\end{flushleft}


\begin{flushleft}
Total Depreciation per year = INR 67.749 Crores / year
\end{flushleft}


\begin{flushleft}
Taxation Rate, including surcharge and education cess = 28.3\% = 0.283
\end{flushleft}


\begin{flushleft}
Total Product Cost = \$ 117.965 Million = INR 825.76 Crores / year
\end{flushleft}


\begin{flushleft}
Revenue = \$ 141.82 Million = INR 992.74 Crores / year
\end{flushleft}


\begin{flushleft}
Bank Interest Rate = 6\%
\end{flushleft}


\begin{flushleft}
Gross Profit = Revenue - Total Product Cost - Depreciation per year
\end{flushleft}


\begin{flushleft}
= INR 99.234 Crores
\end{flushleft}


\begin{flushleft}
Net Profit = Gross Profit * (1 - Taxation Rate) = INR 71.151 Crores
\end{flushleft}


\begin{flushleft}
Interest on TCI = Bank Interest Rate * TCI = INR 53.669 Crores
\end{flushleft}


\begin{flushleft}
P ayback P eriod =
\end{flushleft}





\begin{flushleft}
Depreciable F CI + interest on T CI
\end{flushleft}


\begin{flushleft}
Avg. P rof it/yr + Avg. Depreciation/yr
\end{flushleft}





\begin{flushleft}
Using 70\% production capacity for year 1, 80\% for year 2, 90\% for year 3 and 100\%
\end{flushleft}


\begin{flushleft}
thereafter, the following cash flow diagram was generated,
\end{flushleft}





\begin{flushleft}
Year
\end{flushleft}





\begin{flushleft}
Investment
\end{flushleft}





\begin{flushleft}
Cost
\end{flushleft}





\begin{flushleft}
Revenue
\end{flushleft}





\begin{flushleft}
Gross
\end{flushleft}





\begin{flushleft}
Depreciation
\end{flushleft}





\begin{flushleft}
Profit
\end{flushleft}


0





\begin{flushleft}
Net
\end{flushleft}





\begin{flushleft}
Cash
\end{flushleft}





\begin{flushleft}
Cumulative
\end{flushleft}





\begin{flushleft}
Capacity
\end{flushleft}





\begin{flushleft}
Profit
\end{flushleft}





\begin{flushleft}
Flow
\end{flushleft}





\begin{flushleft}
Position
\end{flushleft}





(\%)





-760.31





-760.31





-760.31





1





-554.10





694.92





140.82





56.46





60.49





116.94





-643.36





70





2





-633.26





794.19





160.93





56.46





74.91





131.37





-511.99





80





3





-743.18





893.47





150.28





56.46





67.27





123.73





-388.26





90





4





-825.76





992.74





166.98





56.46





79.25





135.70





-252.56





100





5





-825.76





992.74





166.98





56.46





79.25





135.70





-116.86





100





6





-825.76





992.74





166.98





56.46





79.25





135.70





18.85





100





7





-825.76





992.74





166.98





56.46





79.25





135.70





154.55





100





8





-825.76





992.74





166.98





56.46





79.25





135.70





290.26





100





9





-825.76





992.74





166.98





56.46





79.25





135.70





425.96





100





10





-825.76





992.74





166.98





56.46





79.25





135.70





561.66





100





\begin{flushleft}
Table 10.6: Cash Flow from Plant production, all values in INR Crores
\end{flushleft}





\begin{flushleft}
\newpage
Plant Economics
\end{flushleft}





62





\begin{flushleft}
Figure 10.1: Estimated Cash flow from Plant Operation
\end{flushleft}





\begin{flushleft}
Hence, Payback Period = 5.8 Years
\end{flushleft}





\begin{flushleft}
The usual payback period for a nitric acid plant is anywhere between 4-6 years [13].
\end{flushleft}


\begin{flushleft}
The following comparison with economics of a 500 TPD Dual Pressure Plant set up in
\end{flushleft}


\begin{flushleft}
Donaldsonville, Louisiana, United States by Uhde in collaboration with CF Industries in
\end{flushleft}


\begin{flushleft}
2016, shows that our plant economics are in accordance with actual industry economics.
\end{flushleft}


\begin{flushleft}
Our Plant
\end{flushleft}





\begin{flushleft}
Uhde Plant
\end{flushleft}





\begin{flushleft}
Hazira, Gujarat, India
\end{flushleft}





\begin{flushleft}
Donaldsonville, Louisiana, US
\end{flushleft}





942





500





\begin{flushleft}
Total FCI (INR)
\end{flushleft}





\begin{flushleft}
761 Crores
\end{flushleft}





\begin{flushleft}
450 Crores
\end{flushleft}





\begin{flushleft}
Payback Period (yrs)
\end{flushleft}





5.8





5





\begin{flushleft}
Location
\end{flushleft}


\begin{flushleft}
Capacity (TPD)
\end{flushleft}


\begin{flushleft}
(100\% NA)
\end{flushleft}





\begin{flushleft}
Table 10.7: Plant Economics Comparison with recently setup plant
\end{flushleft}





\begin{flushleft}
\newpage
Chapter 11
\end{flushleft}





\begin{flushleft}
Detailed Design - Absorption
\end{flushleft}


\begin{flushleft}
Column
\end{flushleft}


\begin{flushleft}
The dual pressure process, requires an absorption column to absorb NOX gases from
\end{flushleft}


\begin{flushleft}
the gaseous reactor into water and produce the desired concentration of nitric acid. The
\end{flushleft}


\begin{flushleft}
column handles three streams - (1) gas stream at 1,76,573 kg/hr, containing 13.35\% by
\end{flushleft}


\begin{flushleft}
weight of nitrogen oxides at 50◦ C and 12 bar abs, (2) deionised make up water at 11,987
\end{flushleft}


\begin{flushleft}
kg/hr, 10◦ C and 12 bar abs and (3) 37.8\% nitric acid condensate from cooler condensers,
\end{flushleft}


\begin{flushleft}
at 25770 kg/hr, 50◦ C and 12 bar abs. The column is required to produce 62,673 kg/hr
\end{flushleft}


\begin{flushleft}
of 62\% HNO3 solution, excluding the dissolved NOX gases.
\end{flushleft}


\begin{flushleft}
A sieve tray absorption column is proposed, with a column diameter of 4 m, height of
\end{flushleft}


\begin{flushleft}
44.6 m, 36 cross-flow type trays with spacings of 1 m between trays 1-13 and 1.2 m
\end{flushleft}


\begin{flushleft}
between trays 13-36. A larger tray spacing is used in the upper section of the column
\end{flushleft}


\begin{flushleft}
to ensure maximum NO to NO2 oxidation. The operating pressure is approximately 12
\end{flushleft}


\begin{flushleft}
bar abs, with an operating temperature range of 10◦ C - 50◦ C.
\end{flushleft}


\begin{flushleft}
A rigorous mathematical model was also produced for this column, to determine the
\end{flushleft}


\begin{flushleft}
exact number of trays for the entire column, using correlations and data from Ray et.
\end{flushleft}


\begin{flushleft}
al. [8] and Thiemann et. al. [6]. The model was run using MATLAB and Google
\end{flushleft}


\begin{flushleft}
Spreadsheets.
\end{flushleft}





63





\begin{flushleft}
\newpage
Detailed Design - Absorption Column
\end{flushleft}





11.1





64





\begin{flushleft}
Mathematical Model
\end{flushleft}





\begin{flushleft}
A rigorous mathematical model was used in sizing the column and determining the exact
\end{flushleft}


\begin{flushleft}
number of trays required for our column. The model is summarised below:
\end{flushleft}





\begin{flushleft}
$\bullet$ Tray by tray approach based on the given feed compositions, temperatures and
\end{flushleft}


\begin{flushleft}
pressure is implemented
\end{flushleft}


\begin{flushleft}
$\bullet$ First, NO oxidation conversion is calculated using correlations from Ray et. al. [8]
\end{flushleft}


\begin{flushleft}
and gas composition is recalculated - Step 1
\end{flushleft}


\begin{flushleft}
$\bullet$ Next, NO2 dimerisation equilibrium with N2 O4 is calculated by solving a quadratic
\end{flushleft}


\begin{flushleft}
equation provided by Ray et. al. [8], giving us the required partial pressure of
\end{flushleft}


\begin{flushleft}
NO2 , and the gas composition is calculated again - Step 2
\end{flushleft}


\begin{flushleft}
$\bullet$ Next, the amount of nitric acid formed in each plate is calculated using tray efficiency and related terms, available in Ray et. al. [8] and derived from Thiemann
\end{flushleft}


\begin{flushleft}
et. al. [6], and both liquid and gas phase compositions are recalculated - Step 3
\end{flushleft}


\begin{flushleft}
$\bullet$ NO2 - N2 O4 equilibrium is recalculated, leading to another recalculation of the gas
\end{flushleft}


\begin{flushleft}
composition - Step 4
\end{flushleft}


\begin{flushleft}
$\bullet$ Method is repeated until a tray achieves an almost zero concentration nitric acid.
\end{flushleft}


\begin{flushleft}
Tail Gas composition and the required amount of make up water are calculated at
\end{flushleft}


\begin{flushleft}
the end. Energy balances are also performed during each iteration.
\end{flushleft}





\begin{flushleft}
The model was solved using an iterative method, implemented by solving 3 functions
\end{flushleft}


\begin{flushleft}
iteratively on MATLAB R2015b, using the fsolve function. Solutions from MATLAB
\end{flushleft}


\begin{flushleft}
were iteratively fed to Google Spreadsheets for the gas composition recalculation, as
\end{flushleft}


\begin{flushleft}
MATLAB R2015b didnt allow for the functions to defined within its for loops. The
\end{flushleft}


\begin{flushleft}
code can be found in the Appendix B.
\end{flushleft}


\begin{flushleft}
Reactions occurring in the Absorption column 2N O + O2 $\rightarrow$ 2N O2 , ∆H = $-$112.7kJ/mol
\end{flushleft}


\begin{flushleft}
2N O2
\end{flushleft}





\begin{flushleft}
N2 O4 , ∆H = $-$58.1kJ/mol
\end{flushleft}





3


\begin{flushleft}
N2 O4 + H2 O
\end{flushleft}


\begin{flushleft}
2HN O3 + N O, ∆H = $-$75.0kJ/mol
\end{flushleft}


2


\begin{flushleft}
The results of the mathematical model are tabulated below:
\end{flushleft}





\begin{flushleft}
\newpage
Detailed Design - Absorption Column
\end{flushleft}





65





\begin{flushleft}
Tray
\end{flushleft}





\begin{flushleft}
Tray
\end{flushleft}





\begin{flushleft}
Nitric Acid
\end{flushleft}





\begin{flushleft}
Energy
\end{flushleft}





\begin{flushleft}
Cooling
\end{flushleft}





\begin{flushleft}
Number
\end{flushleft}





\begin{flushleft}
Spacing(m)
\end{flushleft}





\begin{flushleft}
Conc.(wt\%)
\end{flushleft}





\begin{flushleft}
(kJ/s)
\end{flushleft}





\begin{flushleft}
Water(kg/s)
\end{flushleft}





1





1.5





0.6200





0.8786





0.0145





-2369.3334





28.3413





2





1





0.5939





0.8077





0.0126





-2184.3813





26.1290





3





1





0.5691





0.7909





0.0114





-2062.5688





24.6719





4





1





0.5448





0.7730





0.0103





-1901.1506





22.7410





5





1





0.5210





0.7547





0.0092





-1713.0681





20.4912





6





1





0.4982





0.7363





0.0082





-1537.1034





18.3864





7





1





0.4764





0.7183





0.0072





-1373.5775





16.4304





8





1





0.4561





0.7099





0.0065





-1236.6114





14.7920





9





1





0.4368





0.6839





0.0058





-1101.8346





13.1798





10





1





0.4186





0.6678





0.0052





-997.0431





11.9264





11





1





0.4015





0.6524





0.0046





-890.6925





10.6542





12





1





0.3857





0.6379





0.0041





-799.0663





9.5582





13





1.2





0.3710





0.6727





0.0038





-1546.1511





14.2266





14





1.2





0.3224





0.8300





0.0028





-729.7235





6.7144





15





1.2





0.2927





0.8227





0.0026





-481.0421





4.4262





16





1.2





0.2632





0.8164





0.0023





-412.6295





3.7967





17





1.2





0.2353





0.8107





0.0021





-369.9972





3.4045





18





1.2





0.2083





0.8053





0.0019





-336.2951





3.0944





19





1.2





0.1826





0.8002





0.0017





-304.1869





2.7989





20





1.2





0.1583





0.7955





0.0016





-279.8806





2.5753





21





1.2





0.1343





0.7910





0.0015





-262.3342





2.4138





22





1.2





0.1108





0.7867





0.0014





-246.1110





2.2645





23





1.2





0.0878





0.7825





0.0009





-191.3266





1.7605





24





1.2





0.0732





0.7794





0.0007





-144.0108





1.3251





25





1.2





0.0605





0.7770





0.0007





-124.7951





1.1483





26





1.2





0.0481





0.7748





0.0006





-143.4206





1.3197





27





1.2





0.0366





0.7729





0.0006





-99.8074





0.9184





28





1.2





0.0256





0.7843





0.0004





-89.8746





0.8270





29





1.2





0.0176





0.7828





0.0002





-54.5939





0.5023





30





1.2





0.0138





0.7819





0.0001





-30.9432





0.2847





31





1.2





0.0115





0.7815





0.0001





-11.4580





0.1054





32





1.2





0.0094





0.7811





0.0001





-8.7970





0.0809





33





1.2





0.0073





0.7808





0.0001





-8.2174





0.0756





34





1.2





0.0052





0.7804





0.0001





-8.2106





0.0755





35





1.2





0.0032





0.7800





0.0001





-7.5166





0.0692





36





1.5





0.0013





0.8212





0.0001





-8.1250





0.0744





\begin{flushleft}
X
\end{flushleft}





\begin{flushleft}
Fa
\end{flushleft}





\begin{flushleft}
Table 11.1: Mathematical Model Results
\end{flushleft}





\begin{flushleft}
\newpage
Detailed Design - Absorption Column
\end{flushleft}





66





\begin{flushleft}
X - NO Conversion between trays; Fa - Nitric Acid formed on each tray (kmol/hr);
\end{flushleft}


\begin{flushleft}
Energy - Energy to be removed from each tray (kJ/s); Water Flowrate - Amount of
\end{flushleft}


\begin{flushleft}
Cooling/Chilled Water Required for each tray (kg/s)
\end{flushleft}


\begin{flushleft}
Temp. for section below Tray 13 - 50◦ C - Cooled using cooling water
\end{flushleft}


\begin{flushleft}
Temp. for section above Tray 13 - 10◦ C - Cooled using chilled water
\end{flushleft}


\begin{flushleft}
Equations solved in each step:
\end{flushleft}


\begin{flushleft}
Step 1 k1 tP 2 = (b $-$ a)$-$2 [X(b $-$ a)/(1 $-$ X)b] $-$ ln[(1 $-$ X) $-$ (aX/k1 )]
\end{flushleft}


\begin{flushleft}
X - NO conversion
\end{flushleft}


\begin{flushleft}
k1 - NO Oxidation reaction rate constant
\end{flushleft}


\begin{flushleft}
t - Residence time of gas between trays
\end{flushleft}


\begin{flushleft}
P - Operating Pressure
\end{flushleft}


\begin{flushleft}
b - Mole fraction of oxygen in gas
\end{flushleft}


\begin{flushleft}
2a - Mole fraction of NO in gas
\end{flushleft}


\begin{flushleft}
Step 2 p2N O2 [(2 $-$ $\mu$)2 + AB$\mu$] + pN O2 [A(2 $-$ $\mu$) $-$ ABP $\mu$] + AP $\mu$[2BP $\mu$ $-$ (2 $-$ $\mu$)] = 0
\end{flushleft}


\begin{flushleft}
pNO2 - NO2 Partial Pressure
\end{flushleft}


\begin{flushleft}
$\mu$ - Ratio of moles of nitrogen peroxide to total moles of gas
\end{flushleft}


\begin{flushleft}
P - Operating Pressure
\end{flushleft}


\begin{flushleft}
A, B - Reaction constants for NO2 -N2 O4 equilibrium, given by k2 = A(1 - B*pNOX )
\end{flushleft}


\begin{flushleft}
Step 3 Fa = [PN O + PN OX $-$ (k3 E 3 + 2E 2 )/(k2 + E)](GN O + GN OX )C(PN O + PN OX )$-$1
\end{flushleft}


\begin{flushleft}
Fa - Moles of nitric acid formed
\end{flushleft}


\begin{flushleft}
PNO - Partial Pressure of NO
\end{flushleft}


\begin{flushleft}
PNOX - Partial Pressure of NOX
\end{flushleft}


\begin{flushleft}
E - Equilibrium partial pressure of NO2
\end{flushleft}


\begin{flushleft}
k4 - Equilibrium relation between NO and N2 O4 - Thiemann et. al. [6]
\end{flushleft}


\begin{flushleft}
k2 - Equilibrium relation between NO2 and N2 O4
\end{flushleft}


\begin{flushleft}
k3 = k4 /k2 ; GNO - Flow rate of NO ; GNOX - Flow rate of NOX ; C - Tray Efficiency
\end{flushleft}





\begin{flushleft}
\newpage
Detailed Design - Absorption Column
\end{flushleft}





11.2





67





\begin{flushleft}
Tray Selection and Specifications
\end{flushleft}





\begin{flushleft}
The following parameters were considered while selecting the type of tray for the column:
\end{flushleft}


\begin{flushleft}
Tray Type
\end{flushleft}





\begin{flushleft}
Bubble cap
\end{flushleft}





\begin{flushleft}
Dual Flow
\end{flushleft}





\begin{flushleft}
Sieve
\end{flushleft}





\begin{flushleft}
Valve
\end{flushleft}





\begin{flushleft}
Capacity
\end{flushleft}





\begin{flushleft}
Moderate
\end{flushleft}





\begin{flushleft}
Very High
\end{flushleft}





\begin{flushleft}
High
\end{flushleft}





\begin{flushleft}
High
\end{flushleft}





\begin{flushleft}
Low to
\end{flushleft}





\begin{flushleft}
Low to
\end{flushleft}





\begin{flushleft}
Moderate
\end{flushleft}





\begin{flushleft}
Moderate
\end{flushleft}





\begin{flushleft}
Pressure
\end{flushleft}





\begin{flushleft}
High
\end{flushleft}





\begin{flushleft}
Drop
\end{flushleft}


\begin{flushleft}
Turndown
\end{flushleft}


\begin{flushleft}
Efficiency
\end{flushleft}


\begin{flushleft}
Maintenance
\end{flushleft}





\begin{flushleft}
Very High
\end{flushleft}





\begin{flushleft}
Approx
\end{flushleft}





\begin{flushleft}
Approx
\end{flushleft}





2:1





3-5:1





0.5-0.7





0.7-0.9





0.7-0.9





\begin{flushleft}
Low
\end{flushleft}





\begin{flushleft}
Low
\end{flushleft}





\begin{flushleft}
Low
\end{flushleft}





0.6-0.8


\begin{flushleft}
Relatively
\end{flushleft}


\begin{flushleft}
High
\end{flushleft}





\begin{flushleft}
Fouling
\end{flushleft}





\begin{flushleft}
High
\end{flushleft}





\begin{flushleft}
Low to
\end{flushleft}


\begin{flushleft}
Moderate
\end{flushleft}





\begin{flushleft}
Extremely low
\end{flushleft}





\begin{flushleft}
Low
\end{flushleft}





\begin{flushleft}
Tendency
\end{flushleft}


\begin{flushleft}
Cost
\end{flushleft}





\begin{flushleft}
Moderate
\end{flushleft}





\begin{flushleft}
Low to
\end{flushleft}


\begin{flushleft}
Moderate
\end{flushleft}





\begin{flushleft}
High
\end{flushleft}





\begin{flushleft}
Low
\end{flushleft}





\begin{flushleft}
Low
\end{flushleft}





\begin{flushleft}
Marginally higher
\end{flushleft}


\begin{flushleft}
than sieve trays
\end{flushleft}





\begin{flushleft}
Table 11.2: Tray Selection Parameters
\end{flushleft}





\begin{flushleft}
Sieve Trays were finally selected because of their versatile properties, making them
\end{flushleft}


\begin{flushleft}
feasible for almost every condition.
\end{flushleft}


\begin{flushleft}
Plate Design Procedure provided in Sinnott et. al. [18] is used for our design. Sieve
\end{flushleft}


\begin{flushleft}
Trays of the following specifications are used in the absorption column:
\end{flushleft}


\begin{flushleft}
Column Diameter, Dc = 4m
\end{flushleft}


\begin{flushleft}
Area, Ac = 12.57 m2
\end{flushleft}


\begin{flushleft}
Downcomer Area, Ad = 0.12 Ac = 1.51 m2
\end{flushleft}


\begin{flushleft}
Net Area, An = Ac - Ad = 11.06 m2
\end{flushleft}


\begin{flushleft}
Active Area, Aa = Ac - 2Ad = 9.55 m2
\end{flushleft}


\begin{flushleft}
Hole Area, Ah = 0.1Aa = 0.955 m2
\end{flushleft}


\begin{flushleft}
Referring to Sinnott et. al. [18], we get lw /Dc = 0.77 for given hole area, hence
\end{flushleft}


\begin{flushleft}
Weir Length, lw = 0.77Dc = 3.08 m
\end{flushleft}


\begin{flushleft}
Weir height = 50 mm
\end{flushleft}





\begin{flushleft}
\newpage
Detailed Design - Absorption Column
\end{flushleft}





68





\begin{flushleft}
Hole Diameter = 5mm
\end{flushleft}


\begin{flushleft}
Plate Thickness = 5mm
\end{flushleft}





11.2.1





\begin{flushleft}
Weeping Check
\end{flushleft}





\begin{flushleft}
Maximum liquid flow rate = 17.83 kg/s
\end{flushleft}


\begin{flushleft}
Minimum liquid flow rate, at 70\% turn down = 12.48 kg/s
\end{flushleft}


\begin{flushleft}
hw = 50 mm liquid
\end{flushleft}


\begin{flushleft}
Maximum how = 750(17.83/(1325*3.08))2/3 = 20.05 mm liquid
\end{flushleft}


\begin{flushleft}
Minimum how = 750(12.48/(1325*3.08))2/3 = 15.80 mm liquid
\end{flushleft}


\begin{flushleft}
Hence, at minimum rate , hw + how = 65.80 mm liquid
\end{flushleft}


\begin{flushleft}
Referring to Sinnott et. al. [18], we get K2 = 30.2 for given hw + how , hence
\end{flushleft}


\begin{flushleft}
Uh,min = (30.5 - 0.9(25.4 - 5))/(13.27)0.5 = 3.25 m/s
\end{flushleft}


\begin{flushleft}
Actual minimum vapour velocity = 3.333 m/s
\end{flushleft}


\begin{flushleft}
So, the minimum operating rate will be above the weep point.
\end{flushleft}





11.2.2





\begin{flushleft}
Plate Pressure Drop
\end{flushleft}





\begin{flushleft}
Maximum vapour velocity, Uh,max = 3.87 m/s
\end{flushleft}


\begin{flushleft}
Referring to Sinnott et. al. [18], we get Co = 0.84 for given hole area and plate thickness,
\end{flushleft}


\begin{flushleft}
hence
\end{flushleft}


\begin{flushleft}
Dry plate drop, hd = 51*(3.87/0.84)2 *(13.27/1325) = 10.84 mm liquid
\end{flushleft}


\begin{flushleft}
Residual head, hr = 12500/1325 = 9.43 mm liquid
\end{flushleft}





\begin{flushleft}
So, total plate pressure drop, ht = 10.84 + 9.43 + 50 + 20.05 = 90.32 mm liquid =
\end{flushleft}


\begin{flushleft}
1.174 kPa
\end{flushleft}





11.2.3





\begin{flushleft}
Perforated Area and Hole Pitch
\end{flushleft}





\begin{flushleft}
Referring to Sinnott et. al. [18], we get $\theta$c = 102◦ for given hole area and plate thickness,
\end{flushleft}


\begin{flushleft}
hence,
\end{flushleft}


\begin{flushleft}
Angle subtended at plate edge by unperforated strip = 180 - 102 = 78◦
\end{flushleft}





\begin{flushleft}
\newpage
Detailed Design - Absorption Column
\end{flushleft}





69





\begin{flushleft}
Mean length, unperforated edge strips = (4 - 0.05)*$\pi$*78/180 = 5.38 m
\end{flushleft}


\begin{flushleft}
Area of unperforated edge strips = 0.05*5.38 = 0.269 m2
\end{flushleft}


\begin{flushleft}
Mean length of calming zone = (4 - 0.05)*sin(102/2) = 3.07 m
\end{flushleft}


\begin{flushleft}
Area of calming zone = 2(3.07*0.05) = 0.307 m2
\end{flushleft}


\begin{flushleft}
Total area of perforations, Ap = 9.55 - 0.269 - 0.307 = 8.974 m2
\end{flushleft}


\begin{flushleft}
Ah /Ap = 0.106, hence, referring to Sinnott et. al. [18], we get,
\end{flushleft}


\begin{flushleft}
lp /dh = 3 for given hole area and perforated area, hence,
\end{flushleft}


\begin{flushleft}
Hole pitch, lp = 15 mm, equilateral triangular.
\end{flushleft}





11.2.4





\begin{flushleft}
Number of Holes
\end{flushleft}





\begin{flushleft}
Area of one hole = 1.964 * 10-5 m2
\end{flushleft}


\begin{flushleft}
Hence, number of holes = 0.955/(1.964 * 10-5 ) = 48,650
\end{flushleft}





\begin{flushleft}
Type of Tray
\end{flushleft}





\begin{flushleft}
Sieve Tray
\end{flushleft}





\begin{flushleft}
Flow type
\end{flushleft}





\begin{flushleft}
Cross Flow
\end{flushleft}





\begin{flushleft}
Diameter (Dc )
\end{flushleft}





\begin{flushleft}
4m
\end{flushleft}





\begin{flushleft}
Area (Ac )
\end{flushleft}





\begin{flushleft}
12.57 m2
\end{flushleft}





\begin{flushleft}
Downcomer Area (Ad )
\end{flushleft}





\begin{flushleft}
1.51 m2
\end{flushleft}





\begin{flushleft}
Net Area (An )
\end{flushleft}





\begin{flushleft}
11.06 m2
\end{flushleft}





\begin{flushleft}
Active Area (Aa )
\end{flushleft}





\begin{flushleft}
9.55 m2
\end{flushleft}





\begin{flushleft}
Hole Area (Ah )
\end{flushleft}





\begin{flushleft}
0.955 m2
\end{flushleft}





\begin{flushleft}
Hole diameter
\end{flushleft}





\begin{flushleft}
5 mm
\end{flushleft}





\begin{flushleft}
Active Holes
\end{flushleft}





48600





\begin{flushleft}
Weir Height
\end{flushleft}





\begin{flushleft}
50 mm
\end{flushleft}





\begin{flushleft}
Weir Length ( = 0.77Dc )
\end{flushleft}





\begin{flushleft}
3.08 m
\end{flushleft}





\begin{flushleft}
MoC
\end{flushleft}





\begin{flushleft}
SS 304L (Nitric Acid Grade)
\end{flushleft}





\begin{flushleft}
Plate thickness
\end{flushleft}





\begin{flushleft}
5 mm
\end{flushleft}





\begin{flushleft}
Plate Pressure Drop
\end{flushleft}





\begin{flushleft}
1.175 kPa
\end{flushleft}





\begin{flushleft}
Hole pitch
\end{flushleft}





\begin{flushleft}
Equilateral Triangular - lp = 15 mm
\end{flushleft}





\begin{flushleft}
Turn Down
\end{flushleft}





\begin{flushleft}
70\% of Max Liquid Flow Rate
\end{flushleft}





\begin{flushleft}
Liquid Hold up on tray
\end{flushleft}





\begin{flushleft}
50 mm
\end{flushleft}





\begin{flushleft}
Table 11.3: Final Tray Specifications
\end{flushleft}





\begin{flushleft}
\newpage
Detailed Design - Absorption Column
\end{flushleft}





70





\begin{flushleft}
Figure 11.1: Sieve Tray Perforated Area with Downcomer
\end{flushleft}





\begin{flushleft}
Figure 11.2: Sieve Tray Cooling Coils, Image provided by Krupp Uhde [13]
\end{flushleft}





\begin{flushleft}
Pipe Diameter - 2 - 4 inches
\end{flushleft}


\begin{flushleft}
Pipe Length on each tray - 10m
\end{flushleft}


\begin{flushleft}
Process fluid - Cooling and Chilled water
\end{flushleft}


\begin{flushleft}
Instead of the staggered arrangement, a packed single layer arrangement is used in the
\end{flushleft}


\begin{flushleft}
Dual Pressure Nitric Acid Absorption Column, with each layer of cooling coils, instead
\end{flushleft}


\begin{flushleft}
of being stacked over each other are packed in one layer, one after the other horizontally.
\end{flushleft}





\begin{flushleft}
\newpage
Detailed Design - Absorption Column
\end{flushleft}





11.3





11.4





71





\begin{flushleft}
Head Selection
\end{flushleft}


\begin{flushleft}
Head type
\end{flushleft}





\begin{flushleft}
Pressure range
\end{flushleft}





\begin{flushleft}
Torispherical
\end{flushleft}





\begin{flushleft}
Upto 150 psig
\end{flushleft}





\begin{flushleft}
Ellipsoidal
\end{flushleft}





\begin{flushleft}
150 - 500 psig
\end{flushleft}





\begin{flushleft}
Hemispherical
\end{flushleft}





\begin{flushleft}
Greater than 500 psig
\end{flushleft}





\begin{flushleft}
Tower Thickness
\end{flushleft}





\begin{flushleft}
The column is an internal pressure vessel, with cylindrical body and ellipsoidal heads.
\end{flushleft}


\begin{flushleft}
Design Pressure, P = 12.1 bar
\end{flushleft}


\begin{flushleft}
Internal Diameter, Di = 4 m
\end{flushleft}


\begin{flushleft}
Sa , for SS304L = 180 MPa
\end{flushleft}


\begin{flushleft}
Weld joint efficiency, E = 0.9
\end{flushleft}


\begin{flushleft}
Corrosion allowance, c = 1mm
\end{flushleft}


\begin{flushleft}
Milling tolerance, m =12.5\%
\end{flushleft}


\begin{flushleft}
Cylindrical Body t=
\end{flushleft}





\begin{flushleft}
P Di
\end{flushleft}


\begin{flushleft}
2[Sa E $-$ 0.6P ]
\end{flushleft}





\begin{flushleft}
Hence, we get, t = 15 mm and subsequently, trec = 18.29 mm
\end{flushleft}


\begin{flushleft}
Volume of MoC used = $\pi$*Di *height*trec = 10.013 m3
\end{flushleft}


\begin{flushleft}
Density = 8000 kg/m3
\end{flushleft}


\begin{flushleft}
Hence, Weight of Shell = 80,103.2 kg
\end{flushleft}


\begin{flushleft}
Ellipsoidal Heads t=
\end{flushleft}





\begin{flushleft}
P Di
\end{flushleft}


\begin{flushleft}
2[Sa E $-$ 0.1P ]
\end{flushleft}





\begin{flushleft}
Hence, we get t = 14.95 mm and subsequently, trec = 18.23 mm
\end{flushleft}


\begin{flushleft}
Volume of MoC used = Ellipsoid Surface Area*trec = 0.393 m3
\end{flushleft}


\begin{flushleft}
Weight of Head = 3141.04 kg
\end{flushleft}


\begin{flushleft}
Skirt and Wind Considerations - MoC - CS
\end{flushleft}


\begin{flushleft}
Column height = 44.6 m
\end{flushleft}


\begin{flushleft}
Skirt height = 3.4 m
\end{flushleft}





\begin{flushleft}
\newpage
Detailed Design - Absorption Column
\end{flushleft}





72





\begin{flushleft}
Insulation diameter = 5 m
\end{flushleft}


\begin{flushleft}
Area = 240 m2
\end{flushleft}


\begin{flushleft}
Wind velocity = 70.4 mph1
\end{flushleft}


\begin{flushleft}
Pw = 0.002*70.42 = 9.905 psf = 474.25 Pa
\end{flushleft}


\begin{flushleft}
Force = Pw * G * Area = 474.25 * 1.9 * 240 = 216.3 kN
\end{flushleft}





\begin{flushleft}
M = 216.3 * 1000 * (48/2) = 5190.2 kNm
\end{flushleft}


\begin{flushleft}
Thickness, t = 2.296 mm and subsequently, trec = 3.767 mm
\end{flushleft}


\begin{flushleft}
Column Height
\end{flushleft}





\begin{flushleft}
44.6 m
\end{flushleft}





\begin{flushleft}
Column Diameter
\end{flushleft}





\begin{flushleft}
4m
\end{flushleft}





\begin{flushleft}
Tray Type
\end{flushleft}





\begin{flushleft}
Sieve Tray
\end{flushleft}





\begin{flushleft}
Number of trays
\end{flushleft}





36





\begin{flushleft}
Tray Spacing
\end{flushleft}





\begin{flushleft}
Trays 0 - 12: 1m, Trays 13 - 35: 1.2 m
\end{flushleft}


\begin{flushleft}
Tray 0 - NO Gas
\end{flushleft}


\begin{flushleft}
Tray 35 - Process Water
\end{flushleft}





\begin{flushleft}
Inlets
\end{flushleft}





\begin{flushleft}
Tray 12 - Acid Condensate
\end{flushleft}


\begin{flushleft}
Trays 0-12 - Cooling Water
\end{flushleft}


\begin{flushleft}
Trays 13-35 - Chilled Water
\end{flushleft}





\begin{flushleft}
Body Thickness
\end{flushleft}





\begin{flushleft}
18.29 mm
\end{flushleft}





\begin{flushleft}
Head Type
\end{flushleft}





\begin{flushleft}
Ellipsoidal - Axis Ratio = 1.56:2
\end{flushleft}





\begin{flushleft}
Head Thickness
\end{flushleft}





\begin{flushleft}
18.23 mm
\end{flushleft}





\begin{flushleft}
MoC
\end{flushleft}





\begin{flushleft}
SS 304L (Nitric Acid Grade)
\end{flushleft}





\begin{flushleft}
Operating Pressure
\end{flushleft}





\begin{flushleft}
11 barg
\end{flushleft}





\begin{flushleft}
Operating Temperature
\end{flushleft}





\begin{flushleft}
10-50◦ C
\end{flushleft}





\begin{flushleft}
Design Pressure
\end{flushleft}





\begin{flushleft}
12.1 barg
\end{flushleft}





\begin{flushleft}
Design Temperature
\end{flushleft}





\begin{flushleft}
65◦ C
\end{flushleft}





\begin{flushleft}
Bottom and Top Disengaging spaces
\end{flushleft}





\begin{flushleft}
1.5 m each
\end{flushleft}





\begin{flushleft}
Table 11.4: Final Column Specifications
\end{flushleft}





\begin{flushleft}
A Detailed Mechanical Drawing of the Absorption Column, with all design specifications
\end{flushleft}


\begin{flushleft}
is available in Appendix C
\end{flushleft}


1


\begin{flushleft}
https://www.weatheronline.in/weather/maps/city?WMO=42840\&CONT=inin\&LAND=IGJ\&ART=WST\&
\end{flushleft}


\begin{flushleft}
LEVEL=162\&MOD=tab
\end{flushleft}





\begin{flushleft}
\newpage
Chapter 12
\end{flushleft}





\begin{flushleft}
Environmental Impact
\end{flushleft}


\begin{flushleft}
Only one waste stream exits our plant - Tail Gas containing pollutants NO and NO2 ,
\end{flushleft}


\begin{flushleft}
widely known to be precursors to acid rain and smog, and greenhouse gas N2 O, having
\end{flushleft}


\begin{flushleft}
a global warming potential 280 times that of CO2 . Though there is no statutory limit
\end{flushleft}


\begin{flushleft}
on N2 O emissions in India, United Nation's Clean Development Mechanism provides
\end{flushleft}


\begin{flushleft}
monetary benefits for reducing N2 O emissions. NOX have a strict limit to be followed
\end{flushleft}


\begin{flushleft}
for Nitric Acid Plants in India, and we require an abatement unit to take care of reducing
\end{flushleft}


\begin{flushleft}
these emissions.
\end{flushleft}





12.1





\begin{flushleft}
Tail Gas Composition and Emission Limits
\end{flushleft}





\begin{flushleft}
The table below shows the composition of the Tail Gas leaving the Absorption Column:
\end{flushleft}


\begin{flushleft}
Component
\end{flushleft}





\begin{flushleft}
Flow rate (kg/hr)
\end{flushleft}





\begin{flushleft}
Concentration
\end{flushleft}





\begin{flushleft}
O2
\end{flushleft}





3307





\begin{flushleft}
2.21 wt\%
\end{flushleft}





\begin{flushleft}
N2
\end{flushleft}





145812





\begin{flushleft}
97.71 wt\%
\end{flushleft}





\begin{flushleft}
N2 O
\end{flushleft}





142





\begin{flushleft}
1194.8 ppmv
\end{flushleft}





\begin{flushleft}
NO2
\end{flushleft}





18





\begin{flushleft}
NO
\end{flushleft}





58





\begin{flushleft}
Flow Rate
\end{flushleft}





\begin{flushleft}
149337 kg/hr
\end{flushleft}





\begin{flushleft}
637.8 ppmv
\end{flushleft}


\begin{flushleft}
118846 Nm3 /hr
\end{flushleft}





\begin{flushleft}
Table 12.1: Composition of Tail Gas leaving Absorption Column
\end{flushleft}





73





\begin{flushleft}
\newpage
Environmental Impact
\end{flushleft}





74





\begin{flushleft}
According to the data from the Ministry of Environment, Forest and Climate Change of
\end{flushleft}


\begin{flushleft}
the Central Government of India [19], the permissible emission levels of NOX for Nitric
\end{flushleft}


\begin{flushleft}
Acid Plants is 400 mg/m3 or 400ppmv. Since our NOX emissions are over the limit, we
\end{flushleft}


\begin{flushleft}
will require an abatement unit to reduce the NOX levels.
\end{flushleft}





12.2





\begin{flushleft}
Abatement Technology
\end{flushleft}





\begin{flushleft}
EnviNOX R by Uhde GmbH is world's best abatement technology for Nitric Acid Plants.
\end{flushleft}


\begin{flushleft}
The EnviNOX R Unit provides $>$97\% NOX and $>$98\% N2 O removal, ensuring minimal
\end{flushleft}


\begin{flushleft}
pressure drop ($<$40 mbar), for an average period of more than 10 years.
\end{flushleft}


\begin{flushleft}
Referring Groves et. al. [20], many process variants of this technology are available, but
\end{flushleft}


\begin{flushleft}
we will be using the EnviNOX R process variant-2 for our nitric acid plant. The reasons for choosing this is the lower reducing agent costs and lower reaction temperature
\end{flushleft}


\begin{flushleft}
requirements for achieving very high emission reductions. Catalysts used in this technology, EnviCat R -NOX and EnviCat R -N2 O, are both Iron Zeolites, used individually
\end{flushleft}


\begin{flushleft}
in the DeNOX and DeN2 O R stages.
\end{flushleft}


\begin{flushleft}
Being a propreitary technology, Uhde has given exclusive rights to Clariant and Süd
\end{flushleft}


\begin{flushleft}
Chemie for production of these catalysts for the vast nitric acid process market.
\end{flushleft}


\begin{flushleft}
EnviNOX R process variant-2 involves preheating the tail gas using the secondary air and
\end{flushleft}


\begin{flushleft}
the reactor outlet gas stream to a temperature of 350◦ C at 12 bar abs. Stoichiometric
\end{flushleft}


\begin{flushleft}
amounts of gaseous ammonia are mixed with the tail gas before entering the reactor
\end{flushleft}


\begin{flushleft}
unit containing the catalyst pellets in an annular porous cylindrical setup. Ammonia
\end{flushleft}


\begin{flushleft}
reduces NOX in the DeNOX stage of the reactor unit, ensuring that the DeN2 O R process
\end{flushleft}


\begin{flushleft}
downstream doesn't get hampered by the presence of NOX in the tail gas. Propane is
\end{flushleft}


\begin{flushleft}
added to this mixture, which acts as a reducing agent for N2 O in the DeN2 O R stage,
\end{flushleft}


\begin{flushleft}
removing $>$98\% N2 O from the tail gas, giving us vastly reduced emissions for our plant.
\end{flushleft}


\begin{flushleft}
The tail gas can is then expanded in the tail gas turbine. The tail gas turbine provides
\end{flushleft}


\begin{flushleft}
around 65\% of the power required to run the compressors.
\end{flushleft}


\begin{flushleft}
DeNOX Reactions:
\end{flushleft}


\begin{flushleft}
6N O2 + 8N H3 $\rightarrow$ 7N2 + 12H2 O
\end{flushleft}


\begin{flushleft}
4N O + 4N H3 + O2 $\rightarrow$ 4N2 + 6H2 O
\end{flushleft}





\begin{flushleft}
\newpage
Environmental Impact
\end{flushleft}





75





\begin{flushleft}
DeN2 O R Reactions:
\end{flushleft}


\begin{flushleft}
10N2 O + C3 H8 $\rightarrow$ 10N2 + 3CO2 + 4H2 O
\end{flushleft}





\begin{flushleft}
Figure 12.1: EnviNOX R Process Variant - 2, recreated from [20]
\end{flushleft}





12.3





\begin{flushleft}
Tail Gas Post Abatement
\end{flushleft}





\begin{flushleft}
A catalyst volume of 7m3 is used, as suggested Krupp Uhde [13]. This volume of catalyst
\end{flushleft}


\begin{flushleft}
gives us a N2 O removal of 98.5\% and NOX removal of 97\%. The final tail gas composition
\end{flushleft}


\begin{flushleft}
is shown in the table below:
\end{flushleft}


\begin{flushleft}
Component
\end{flushleft}





\begin{flushleft}
Flow rate (kg/hr)
\end{flushleft}





\begin{flushleft}
Concentration
\end{flushleft}





\begin{flushleft}
O2
\end{flushleft}





3292





\begin{flushleft}
2.2 wt\%
\end{flushleft}





\begin{flushleft}
N2
\end{flushleft}





145966





\begin{flushleft}
97.7 wt\%
\end{flushleft}





\begin{flushleft}
N2 O
\end{flushleft}





2.13





\begin{flushleft}
18 ppm
\end{flushleft}





\begin{flushleft}
NO2
\end{flushleft}





0.54





\begin{flushleft}
5 ppmv
\end{flushleft}





\begin{flushleft}
NO
\end{flushleft}





1.74





\begin{flushleft}
15 ppmv
\end{flushleft}





\begin{flushleft}
CO2
\end{flushleft}





42





\begin{flushleft}
339 ppmv
\end{flushleft}





\begin{flushleft}
H2 O
\end{flushleft}





87





\begin{flushleft}
730 ppmv
\end{flushleft}





\begin{flushleft}
Total
\end{flushleft}





\begin{flushleft}
149392 kg/hr
\end{flushleft}





\begin{flushleft}
119144 Nm3 /hr
\end{flushleft}





\begin{flushleft}
\newpage
Environmental Impact
\end{flushleft}





12.4





76





\begin{flushleft}
Revised Plant Economics
\end{flushleft}





\begin{flushleft}
Addition of the EnviNOX R Unit in the plant will incur increased fixed and working
\end{flushleft}


\begin{flushleft}
capital investments, mainly involving the catalyst, additional ammonia and propane
\end{flushleft}


\begin{flushleft}
costs.
\end{flushleft}





12.4.1





\begin{flushleft}
Reactor Unit Cost
\end{flushleft}





\begin{flushleft}
According to the data provided by Krupp Uhde [13],
\end{flushleft}


\begin{flushleft}
Catalyst cost = A
\end{flushleft}


\begin{flushleft}
C100,000 per m3 of catalyst
\end{flushleft}


\begin{flushleft}
Since the catalyst is has an average life of 10 years, only one batch of catalyst is capitalised and the amount catalyst used is 7 m3 , hence, using A
\end{flushleft}


\begin{flushleft}
C1 = INR 78,
\end{flushleft}


\begin{flushleft}
Total Catalyst Cost = A
\end{flushleft}


\begin{flushleft}
C100,000 * 7 = A
\end{flushleft}


\begin{flushleft}
C700,000 = INR 5.46 Crores
\end{flushleft}


\begin{flushleft}
The reactor is an annular cylindrical fixed bed reactor, with two catalyst bed of around
\end{flushleft}


\begin{flushleft}
3.5 m3 each, filled inside a perforated cylinder, to allow gas flow from outside the cylinder
\end{flushleft}


\begin{flushleft}
to the inner annular region. The beds are housed inside a cylindrical shell of 5 m internal
\end{flushleft}


\begin{flushleft}
diameter, with ellipsoidal heads, operating at 12 bar abs. The MoC used is SS 321.
\end{flushleft}


\begin{flushleft}
According to the data provided by Krupp Uhde [13], the reactor setup is designed by
\end{flushleft}


\begin{flushleft}
Uhde GmbH and will cost around INR 1 Crore for the given catalyst volume. Hence,
\end{flushleft}


\begin{flushleft}
Additional Fixed Capital Investment = INR 6.46 Crore
\end{flushleft}





12.4.2





\begin{flushleft}
Additional Raw Material Costs
\end{flushleft}





\begin{flushleft}
Additional Ammonia costs were included in the Raw Material Cost Estimation done in
\end{flushleft}


\begin{flushleft}
Chapter 10
\end{flushleft}


\begin{flushleft}
Additional Ammonia Required = 40 kg/hr
\end{flushleft}


\begin{flushleft}
Propane Required = 15 kg/hr = 8.09 gallon/hr = 64,073 gallon/year Propane Cost =
\end{flushleft}


\begin{flushleft}
INR 50/gal
\end{flushleft}


\begin{flushleft}
Total Propane costs per year = INR 32,03,690 / year = \$ 45,767 / year
\end{flushleft}


\begin{flushleft}
Using the method outlined in Chapter 10, calculation were again done, giving us,
\end{flushleft}





\begin{flushleft}
\newpage
Environmental Impact
\end{flushleft}





77





\begin{flushleft}
New Total Capital Investment = INR 921.29 Crores
\end{flushleft}


\begin{flushleft}
New Raw Material Cost = \$ 42.28 Million / year
\end{flushleft}


\begin{flushleft}
New Total Product Cost = \$ 118.09 Million / year = INR 826.65 Crores / year
\end{flushleft}





12.4.3





\begin{flushleft}
Additional Revenue Calculation
\end{flushleft}





\begin{flushleft}
Addition of a N2 O abatement technology can be registered under the United Nation's
\end{flushleft}


\begin{flushleft}
Clean Development Mechanism as a Large Scale Emission Reducing Project. Under the
\end{flushleft}


\begin{flushleft}
CDM, in developing countries, efforts taken to reduce potential global warming causing
\end{flushleft}


\begin{flushleft}
emissions are rewarded with Certified Emission Reduction units, or Carbon Credits,
\end{flushleft}


\begin{flushleft}
that can be purchased by Annex I countries1 to comply with their emission limitation
\end{flushleft}


\begin{flushleft}
targets as set by the Kyoto Protocol2 .
\end{flushleft}


\begin{flushleft}
Since the prices of CERs fluctuate from year to year, two calculations were done for
\end{flushleft}


\begin{flushleft}
calculating the payback period, (1) without the CER Revenue, and (2) with the CER
\end{flushleft}


\begin{flushleft}
Revenue.
\end{flushleft}





12.4.3.1





\begin{flushleft}
No CER Revenue
\end{flushleft}





\begin{flushleft}
Total Revenue remains the same, hence we have,
\end{flushleft}


\begin{flushleft}
Total Capital Investment = INR 921.29 Crores
\end{flushleft}


\begin{flushleft}
Total Depreciable FCI = INR 697.80 Crores
\end{flushleft}


\begin{flushleft}
Total Depreciation per year = INR 69.780 Crores / year
\end{flushleft}


\begin{flushleft}
Taxation Rate, including surcharge and education cess = 28.3\% = 0.283
\end{flushleft}


\begin{flushleft}
Total Product Cost = \$ 118.09 Million = INR 826.65 Crores / year
\end{flushleft}


\begin{flushleft}
Revenue = \$ 141.82 Million = INR 992.74 Crores / year
\end{flushleft}


\begin{flushleft}
Bank Interest Rate = 6\%
\end{flushleft}


\begin{flushleft}
Gross Profit = Revenue - Total Product Cost - Depreciation per year
\end{flushleft}


\begin{flushleft}
= INR 96.308 Crores
\end{flushleft}


1





\begin{flushleft}
Annex I countries - https://en.wikipedia.org/wiki/United\_Nations\_Framework\_Convention\_
\end{flushleft}


\begin{flushleft}
on\_Climate\_Change\#Annex\_I\_countries
\end{flushleft}


2


\begin{flushleft}
Kyoto Protocol - https://en.wikipedia.org/wiki/Kyoto\_Protocol
\end{flushleft}





\begin{flushleft}
\newpage
Environmental Impact
\end{flushleft}





78





\begin{flushleft}
Net Profit = Gross Profit * (1 - Taxation Rate) = INR 69.053 Crores
\end{flushleft}


\begin{flushleft}
Interest on TCI = Bank Interest Rate * TCI = INR 55.278 Crores
\end{flushleft}


\begin{flushleft}
Hence, Payback Period = 6 Years
\end{flushleft}





12.4.3.2





\begin{flushleft}
With CER Revenue
\end{flushleft}





\begin{flushleft}
According to the UN's Clean Development Mechanism, project registered under it will
\end{flushleft}


\begin{flushleft}
be rewarded with 1 CER / tonne CO2 eq. reduced below the baseline for the entirety of
\end{flushleft}


\begin{flushleft}
the crediting period. The baseline model was selected to be the emission from the plant,
\end{flushleft}


\begin{flushleft}
as India doesn't have any limits on N2 O emissions. The baseline model was selected
\end{flushleft}


\begin{flushleft}
after referring to Methodology AM0028 [21], accessible from the CDM portal of UN.
\end{flushleft}


\begin{flushleft}
Crediting Period = 10 years
\end{flushleft}


\begin{flushleft}
Global Warming Potential(GWP) of N2 O = 280 * GWP of CO2
\end{flushleft}


\begin{flushleft}
Emission Reduction = (142 - 2.13)*8000 = 1119 tonnes N2 O = 3,12,973 tonnes CO2 eq.
\end{flushleft}


\begin{flushleft}
Trading value of CER = A
\end{flushleft}


\begin{flushleft}
C0.26 / CER3 (Assuming trading value to remain constant)
\end{flushleft}


\begin{flushleft}
Revenue Earned per year = A
\end{flushleft}


\begin{flushleft}
C81,373 = INR 63.48 Lakhs / year
\end{flushleft}


\begin{flushleft}
Hence, Revenue earned during Crediting Period = INR 6.348 Crores
\end{flushleft}


\begin{flushleft}
Hence, the revenue earned from CER trading, will help us recover the fixed capital
\end{flushleft}


\begin{flushleft}
investment required for setting up the abatement unit.
\end{flushleft}


\begin{flushleft}
Payback Period = 6 years
\end{flushleft}





3





\begin{flushleft}
https://www.moneycontrol.com/commodity/cer-price.html
\end{flushleft}





\begin{flushleft}
\newpage
Appendix A
\end{flushleft}





\begin{flushleft}
Final Process Flow Diagram
\end{flushleft}


\begin{flushleft}
The Final Process Flow Diagram with the Stream Table is shown below. The collective
\end{flushleft}


\begin{flushleft}
image was taken using Google Spreadsheets. The PFD was made using draw.io.
\end{flushleft}





79





\begin{flushleft}
\newpage
Line No
\end{flushleft}


\begin{flushleft}
Stream Comp.
\end{flushleft}


\begin{flushleft}
Phase
\end{flushleft}


\begin{flushleft}
NH3
\end{flushleft}


\begin{flushleft}
O2
\end{flushleft}


\begin{flushleft}
N2
\end{flushleft}


\begin{flushleft}
H2O
\end{flushleft}


\begin{flushleft}
N2O
\end{flushleft}


\begin{flushleft}
NO2
\end{flushleft}


\begin{flushleft}
NO
\end{flushleft}


\begin{flushleft}
HNO3
\end{flushleft}


\begin{flushleft}
CO2
\end{flushleft}


\begin{flushleft}
C3H8
\end{flushleft}


\begin{flushleft}
Total
\end{flushleft}


\begin{flushleft}
Pressure (bar)
\end{flushleft}


\begin{flushleft}
Temperature (C)
\end{flushleft}





1


\begin{flushleft}
Air feed
\end{flushleft}


\begin{flushleft}
g
\end{flushleft}


0.00


43518.85


145541.03


0.00


0.00


0.00


0.00


0.00


0.00


0.00


189059.88


1.00


35.00





\begin{flushleft}
1C
\end{flushleft}


\begin{flushleft}
Primary Air
\end{flushleft}


\begin{flushleft}
g
\end{flushleft}


0.00


39004.55


128386.69


0.00


0.00


0.00


0.00


0.00


0.00


0.00


167391.24


5.00


240.00





\begin{flushleft}
1D
\end{flushleft}


\begin{flushleft}
Secondary Air
\end{flushleft}


\begin{flushleft}
g
\end{flushleft}


0.00


4514.30


17154.34


0.00


0.00


0.00


0.00


0.00


0.00


0.00


21668.64


5.00


240.00





2


\begin{flushleft}
Ammonia Feed
\end{flushleft}


\begin{flushleft}
l
\end{flushleft}


11003.30


0.00


0.00


0.00


0.00


0.00


0.00


0.00


0.00


0.00


11003.30


1.00


-33.00





\begin{flushleft}
2A
\end{flushleft}


\begin{flushleft}
Evaporated NH3
\end{flushleft}


\begin{flushleft}
g
\end{flushleft}


10963.30


0.00


0.00


0.00


0.00


0.00


0.00


0.00


0.00


0.00


10963.30


5.00


70.00





3


\begin{flushleft}
Reactor Inlet
\end{flushleft}


\begin{flushleft}
g
\end{flushleft}


10963.30


39004.55


128386.69


0.00


0.00


0.00


0.00


0.00


0.00


0.00


178354.54


5.00


230.00





4


\begin{flushleft}
Reactor Outlet
\end{flushleft}


\begin{flushleft}
g
\end{flushleft}


0.00


13569.84


128657.55


17412.30


141.88


0.00


18573.12


0.00


0.00


0.00


178354.69


5.00


427.00





5


\begin{flushleft}
HE102 Outlet
\end{flushleft}


\begin{flushleft}
g
\end{flushleft}


0.00


12125.47


128657.84


17412.30


141.88


4153.16


15864.54


0.00


0.00


0.00


178355.18


5.00


300.00





6


\begin{flushleft}
HE103 Outlet
\end{flushleft}


\begin{flushleft}
g
\end{flushleft}


0.00


9855.66


128657.84


17412.30


141.88


10679.54


11608.20


0.00


0.00


0.00


178355.42


5.00


185.00





7


\begin{flushleft}
Acid Condensate
\end{flushleft}


\begin{flushleft}
l
\end{flushleft}


0.00


0.00


0.00


16019.32


0.00


0.00


0.00


9750.89


0.00


0.00


25770.20


12.00


50.00





8


\begin{flushleft}
NOx Comp. inlet
\end{flushleft}


\begin{flushleft}
g
\end{flushleft}


0.00


8590.66


145812.18


0.00


141.88


18932.46


3095.52


0.00


0.00


0.00


176572.69


5.00


50.00





\begin{flushleft}
8A
\end{flushleft}


\begin{flushleft}
NOx Comp. Outlet
\end{flushleft}


\begin{flushleft}
g
\end{flushleft}


0.00


8590.66


145812.18


0.00


141.88


18932.46


3095.52


0.00


0.00


0.00


176572.69


12.00


260.00





9


\begin{flushleft}
Absorption Inlet
\end{flushleft}


\begin{flushleft}
g
\end{flushleft}


0.00


7042.89


145812.18


0.00


141.88


23382.28


193.47


0.00


0.00


0.00


176572.69


12.00


50.00





10


\begin{flushleft}
Tail Gas
\end{flushleft}


\begin{flushleft}
g
\end{flushleft}


0.00


3306.59


145812.18


0.00


141.88


17.75


58.05


0.00


0.00


0.00


149336.44


12.00


130.00





11


\begin{flushleft}
Absorption Outlet
\end{flushleft}


\begin{flushleft}
l
\end{flushleft}


0.00


0.00


0.00


23815.84


0.00


2319.83


0.00


38857.50


0.00


0.00


64993.17


12.00


50.00





12


\begin{flushleft}
Process Water
\end{flushleft}


\begin{flushleft}
l
\end{flushleft}


0.00


0.00


0.00


11986.72


0.00


0.00


0.00


0.00


0.00


0.00


11986.72


12.00


10.00





13


\begin{flushleft}
Bleaching Air
\end{flushleft}


\begin{flushleft}
g
\end{flushleft}


0.00


4514.30


17154.34


0.00


0.00


2319.83


0.00


0.00


0.00


0.00


23988.47


5.00


80.00





14


\begin{flushleft}
Nitric Acid Pdt
\end{flushleft}


\begin{flushleft}
l
\end{flushleft}


0.00


0.00


0.00


23815.84


0.00


0.00


0.00


38857.50


0.00


0.00


62673.34


5.00


50.00





15


\begin{flushleft}
Mixed Tail Gas
\end{flushleft}


\begin{flushleft}
g
\end{flushleft}


40.00


3306.59


145812.18


0.00


141.88


17.75


58.05


0.00


0.00


0.00


149376.44


12.00


350.00





16


\begin{flushleft}
Reduced Tail Gas
\end{flushleft}


\begin{flushleft}
g
\end{flushleft}


0.00


3291.59


145966.09


87.18


2.13


0.54


1.74


0.00


41.96


0.00


149391.23


12.00


350.00





17


\begin{flushleft}
Tail Gas Out
\end{flushleft}


\begin{flushleft}
g
\end{flushleft}


0.00


3291.59


145966.09


87.18


2.13


0.54


1.74


0.00


41.96


0.00


149391.23


1.00


35.00





18


\begin{flushleft}
Reducing Ammonia
\end{flushleft}


\begin{flushleft}
g
\end{flushleft}


40.48


0.00


0.00


0.00


0.00


0.00


0.00


0.00


0.00


0.00


40.48


12.00


350.00





19


\begin{flushleft}
Propane
\end{flushleft}


\begin{flushleft}
g
\end{flushleft}


0.00


0.00


0.00


0.00


0.00


0.00


0.00


0.00


0.00


15.00


15.00


12.00


350.00





\begin{flushleft}
\newpage
Appendix B
\end{flushleft}





\begin{flushleft}
MATLAB Code for the
\end{flushleft}


\begin{flushleft}
Mathematical Model for NOX
\end{flushleft}


\begin{flushleft}
Absorption
\end{flushleft}


\begin{flushleft}
AbsorptionColumn.m File
\end{flushleft}


\begin{flushleft}
1 clear all ;
\end{flushleft}


\begin{flushleft}
2 clc ;
\end{flushleft}


\begin{flushleft}
3 X1 = fsolve ( @myfunc , 0.5) ;
\end{flushleft}


\begin{flushleft}
4 Gno2 = fsolve ( @myfunc3 , 0.5) *1 . 4 8 7 3 9 1 84 1 / 1 2 ;
\end{flushleft}


\begin{flushleft}
5 F = fsolve ( @myfunc2 , 0.5) ;
\end{flushleft}


\begin{flushleft}
6 \% Gno2 = fsolve ( @myfunc3 , 0.5) * 1 . 5 8 7 7 0 2 8 8 8 / 1 2 ;
\end{flushleft}


\begin{flushleft}
7 \% [ p1 , p2 , p3 ] = myfunc4 () ;
\end{flushleft}


\begin{flushleft}
8 \% p = [ p1 , p2 , p3 ];
\end{flushleft}


\begin{flushleft}
9 \% roots ( p ) * 1 . 5 5 2 5 7 3 2 0 2 / 1 2
\end{flushleft}





\begin{flushleft}
Solving Step - 1
\end{flushleft}


\begin{flushleft}
1 function [ y ] = myfunc ( x )
\end{flushleft}


2


3





\begin{flushleft}
Vw = 1765 76.4/360 0;
\end{flushleft}





4





\begin{flushleft}
G = 1.614361111;
\end{flushleft}





5





\begin{flushleft}
b = 0 .061583 33333/ G ; \% O2 molfrac
\end{flushleft}





6





\begin{flushleft}
a = 0 . 0 0 1 8 0 5 5 5 5 5 5 6 / ( 2 * G ) ; \% NO molfrac
\end{flushleft}





7


8





\begin{flushleft}
T = 283;
\end{flushleft}





9





\begin{flushleft}
P = 12;
\end{flushleft}





10





\begin{flushleft}
rhov = P *100*( Vw / G ) /(8.314* T ) ;
\end{flushleft}





11





\begin{flushleft}
k1 = (10\^{}((635/ T ) - 1.0261) ) /(8.206* T *10\^{} -3) ;
\end{flushleft}





81





\begin{flushleft}
\newpage
Environmental Impact
\end{flushleft}


12 \%


13





\begin{flushleft}
k1 = exp ((530/ T ) + 7.78)
\end{flushleft}


\begin{flushleft}
ar = pi *(4\^{}2) /4;
\end{flushleft}





14





\begin{flushleft}
vel = Vw /( rhov * ar ) ;
\end{flushleft}





15





\begin{flushleft}
t = 1.5/ vel ;
\end{flushleft}





16


17





\begin{flushleft}
y = (( x *( b - a ) /( b *(1 - x ) ) ) - log (1 - x -( a * x / k1 ) ) ) /(( b - a ) \^{}2) - k1 * t * P * P ;
\end{flushleft}





18


\begin{flushleft}
19 end
\end{flushleft}





\begin{flushleft}
Solving Step - 3
\end{flushleft}


\begin{flushleft}
1 function [ y ] = myfunc2 ( f )
\end{flushleft}


2


3





\begin{flushleft}
X = 0.7804;
\end{flushleft}





4





\begin{flushleft}
P = 12;
\end{flushleft}





5





\begin{flushleft}
T = 307.68;
\end{flushleft}





6


7





\begin{flushleft}
V = 152108.312/3600;
\end{flushleft}





8





\begin{flushleft}
Gno = 0 . 0 0 0 0 1 4 1 8 7 0 0 9 6 8 ;
\end{flushleft}





9





\begin{flushleft}
Gno2 = 0.0041;
\end{flushleft}





10





\begin{flushleft}
Gn2o4 = 0 . 00 65 10 5 81 49 5;
\end{flushleft}





11





\begin{flushleft}
Gnox = Gno2 + Gn2o4 ;
\end{flushleft}





12





\begin{flushleft}
G = 1.487318477;
\end{flushleft}





13





\begin{flushleft}
Ghno3 = 0.000 2849531 8;
\end{flushleft}





14





\begin{flushleft}
Gh2o = 0.1785554731;
\end{flushleft}





15


16





\begin{flushleft}
Pno = P * Gno / G ;
\end{flushleft}





17





\begin{flushleft}
Pno2 = P * Gno2 / G ;
\end{flushleft}





18





\begin{flushleft}
Pn2o4 = P * Gn2o4 / G ;
\end{flushleft}





19





\begin{flushleft}
Pnox = Pno2 + Pn2o4 ;
\end{flushleft}





20


21





\begin{flushleft}
A = 10\^{}(8.756 - (2838/ T ) ) ;
\end{flushleft}





22





\begin{flushleft}
B = ( T /210.3) + (790/ T ) - (3.8794) ;
\end{flushleft}





23





\begin{flushleft}
k2 = A *(1 - B * Pnox ) ;
\end{flushleft}





24





\begin{flushleft}
w = ( Ghno3 *63) /( Ghno3 *63 + Gh2o *18) ;
\end{flushleft}





25





\begin{flushleft}
rhol = ( Ghno3 *63 + Gh2o *18) /(( Ghno3 *63/1513) + ( Gh2o *18/1000) ) ;
\end{flushleft}





26





\begin{flushleft}
nHNO3 = ( Ghno3 *63) /(( Ghno3 *63 + Gh2o *18) *10/ rhol ) ;
\end{flushleft}





27





\begin{flushleft}
k4 = 10\^{}(7.412 - 20.28921* w + 32.47322* w * w - 30.87* w * w * w ) ;
\end{flushleft}





28





\begin{flushleft}
k3 = k4 / k2 ;
\end{flushleft}





29





\begin{flushleft}
function [ f ] = myfunc5 ( x )
\end{flushleft}





30





\begin{flushleft}
f = 3* k3 * x * x * x + (2* x * x / k2 ) + x - 3* Pno - Pnox ;
\end{flushleft}





31





\begin{flushleft}
end
\end{flushleft}





32





\begin{flushleft}
E = fsolve ( @myfunc5 , 0.5) ;
\end{flushleft}





33





\begin{flushleft}
rhov = P *100*( V / G ) /(8.314* T ) ;
\end{flushleft}





34





\begin{flushleft}
H = 0.04;
\end{flushleft}





35





\begin{flushleft}
nNONOx = ( Gno *30 + Gno2 *46 + Gn2o4 *92) /(10* V /( rhov ) ) ;
\end{flushleft}





36





\begin{flushleft}
R = ( Pno + Pnox ) / P ;
\end{flushleft}





37





\begin{flushleft}
if R $<$ 0.01
\end{flushleft}





82





\begin{flushleft}
\newpage
Environmental Impact
\end{flushleft}


38





83





\begin{flushleft}
A = 10.86 - 1.65*( R \^{}970000) + 37.59* exp ( -28.8* R ) ;
\end{flushleft}





39





\begin{flushleft}
else
\end{flushleft}





40





\begin{flushleft}
A = 8.73;
\end{flushleft}





41





\begin{flushleft}
end
\end{flushleft}





42





\begin{flushleft}
ar = pi *(4\^{}2) /4;
\end{flushleft}





43





\begin{flushleft}
vel = V /( rhov * ar ) ;
\end{flushleft}





44





\begin{flushleft}
Dh = 0.003;
\end{flushleft}





45





\begin{flushleft}
s = 0.9;
\end{flushleft}





46





\begin{flushleft}
C = ( A *( P \^{}0.15) *( nNONOx \^{}0.1) *( X \^{}0.4) *( H \^{}0.15) *(( nHNO3 ) \^{}0.1) ) /(( vel \^{}0.26) *( Dh
\end{flushleft}


\begin{flushleft}
\^{}0.15) *( T \^{}0.87) *( s \^{}0.13) ) ;
\end{flushleft}





47





\begin{flushleft}
y = (( Pno + Pnox - (( k3 * E * E * E + 2* E * E ) /( k2 + E ) ) ) *( Gno + Gnox ) * C /( Pno + Pnox )
\end{flushleft}


\begin{flushleft}
) - f;
\end{flushleft}





48


\begin{flushleft}
49 end
\end{flushleft}





\begin{flushleft}
Solving Step - 2 and 4
\end{flushleft}


\begin{flushleft}
1 function [ y ] = myfunc3 ( x )
\end{flushleft}


2


3





\begin{flushleft}
T = 280;
\end{flushleft}





4





\begin{flushleft}
A = 10\^{}(8.756 - (2838/ T ) ) ;
\end{flushleft}





5





\begin{flushleft}
B = ( T /210.3) + (790/ T ) - (3.8794) ;
\end{flushleft}





6


7





\begin{flushleft}
Gno2 = 0. 00 4 15 07 91 5 82 ;
\end{flushleft}





8





\begin{flushleft}
Gn2o4 = 0 . 00 65 50 1 49 65 3;
\end{flushleft}





9





\begin{flushleft}
G = 1.487441445;
\end{flushleft}





10


11





\begin{flushleft}
u = ( Gno2 + Gn2o4 ) / G ;
\end{flushleft}





12





\begin{flushleft}
P = 12;
\end{flushleft}





13


14





\begin{flushleft}
\% y = (((( Gno2 -2* x ) \^{}2) * P ) /(( Gn2o4 + x ) * (G - x ) ) ) - A *(1 - B *(( Gno2 + Gn2o4 - x
\end{flushleft}


\begin{flushleft}
) * P /( G - x ) ) ) ;
\end{flushleft}





15





\begin{flushleft}
y = x * x *(((2 - u ) \^{}2) + A * B * u ) + x *( A *(2 - u ) - A * B * P * u *(2 - u ) ) + ( A * P * u *(2* B * P * u (2 - u ) ) ) ;
\end{flushleft}





16


\begin{flushleft}
17 end
\end{flushleft}





\begin{flushleft}
\newpage
Appendix C
\end{flushleft}





\begin{flushleft}
Detailed Drawing of Absorption
\end{flushleft}


\begin{flushleft}
Column
\end{flushleft}


\begin{flushleft}
The detailed drawing for the Absorption Column was made using Adobe Illustrator.
\end{flushleft}


\begin{flushleft}
The detailed calculations to arrive at the dimensions have been shown in Chapter 11.
\end{flushleft}





84





\begin{flushleft}
\newpage
4m
\end{flushleft}


\begin{flushleft}
Tail Gas to HE106
\end{flushleft}





\begin{flushleft}
Tail Gas to HE106
\end{flushleft}


\begin{flushleft}
N105
\end{flushleft}





\begin{flushleft}
1.56 m
\end{flushleft}





\begin{flushleft}
Material of Construction
\end{flushleft}





\begin{flushleft}
Demister
\end{flushleft}





\begin{flushleft}
For the column : SS304L (Nitric Acid Grade)
\end{flushleft}


\begin{flushleft}
For the skirt : CS
\end{flushleft}





\begin{flushleft}
Process Water Inlet
\end{flushleft}





\begin{flushleft}
1.5 m
\end{flushleft}





\begin{flushleft}
Cooling Coils
\end{flushleft}


\begin{flushleft}
N069/N070
\end{flushleft}





\begin{flushleft}
N104
\end{flushleft}





\begin{flushleft}
PLATE 35
\end{flushleft}


\begin{flushleft}
1.2 m Manhole 7
\end{flushleft}


\begin{flushleft}
N067/N068
\end{flushleft}





\begin{flushleft}
PLATE 34
\end{flushleft}





\begin{flushleft}
Operating Parameters
\end{flushleft}





\begin{flushleft}
N065/N066
\end{flushleft}





\begin{flushleft}
PLATE 33
\end{flushleft}





\begin{flushleft}
Operating Pressure - 11 barg
\end{flushleft}


\begin{flushleft}
Operating Temperature - 283-323 K
\end{flushleft}


\begin{flushleft}
Weak Acid Feed (Plate 12) - 25770 kg/hr
\end{flushleft}


\begin{flushleft}
Process Water Feed (Plate 35) - 11580 kg/hr
\end{flushleft}


\begin{flushleft}
NO Gas feed (Column Bottom) - 176577 kg/hr
\end{flushleft}





\begin{flushleft}
N063/N064
\end{flushleft}





\begin{flushleft}
PLATE 32
\end{flushleft}


\begin{flushleft}
N061/N062
\end{flushleft}





\begin{flushleft}
PLATE 31
\end{flushleft}





\begin{flushleft}
N059/N060
\end{flushleft}





\begin{flushleft}
PLATE 30
\end{flushleft}





\begin{flushleft}
Design Parameters
\end{flushleft}





\begin{flushleft}
N057/N058
\end{flushleft}





\begin{flushleft}
PLATE 29
\end{flushleft}


\begin{flushleft}
Manhole 6
\end{flushleft}





\begin{flushleft}
N051/N052
\end{flushleft}





\begin{flushleft}
Design Pressure - 12.1 barg
\end{flushleft}


\begin{flushleft}
Design Temperature - 338 K
\end{flushleft}


\begin{flushleft}
Weld Joint Efficiency - 90\%
\end{flushleft}


\begin{flushleft}
Shell Thickness - 18.3 mm
\end{flushleft}


\begin{flushleft}
Head Thickness - 18.3 mm
\end{flushleft}


\begin{flushleft}
Skirt Thickness - 3.8 mm
\end{flushleft}





\begin{flushleft}
N047/N048
\end{flushleft}





\begin{flushleft}
Nozzle Details
\end{flushleft}





\begin{flushleft}
N055/N056
\end{flushleft}





\begin{flushleft}
PLATE 28
\end{flushleft}





\begin{flushleft}
N053/N054
\end{flushleft}





\begin{flushleft}
PLATE 27
\end{flushleft}





\begin{flushleft}
PLATE 26
\end{flushleft}





\begin{flushleft}
N049/N050
\end{flushleft}





\begin{flushleft}
PLATE 25
\end{flushleft}





\begin{flushleft}
PLATE 24
\end{flushleft}





\begin{flushleft}
Cooling Coils
\end{flushleft}


\begin{flushleft}
1) N001-N024, N071-N072 - Cooling Water Cooled coils
\end{flushleft}


\begin{flushleft}
- Diameter : 2-4 inches
\end{flushleft}


\begin{flushleft}
2) N025-N070 - Chilled Water Cooled coils
\end{flushleft}


\begin{flushleft}
- Diameter : 2 inches
\end{flushleft}





\begin{flushleft}
N045/N046
\end{flushleft}





\begin{flushleft}
PLATE 23
\end{flushleft}


\begin{flushleft}
Manhole 5
\end{flushleft}


\begin{flushleft}
N043/N044
\end{flushleft}





\begin{flushleft}
PLATE 22
\end{flushleft}





\begin{flushleft}
N041/N042
\end{flushleft}





\begin{flushleft}
Main Process Streams
\end{flushleft}


\begin{flushleft}
3) N101 - NO Gas Inlet
\end{flushleft}


\begin{flushleft}
- Diameter : 8 inches
\end{flushleft}


\begin{flushleft}
4) N102 - Product acid oulet
\end{flushleft}


\begin{flushleft}
- Diameter : 6 inches
\end{flushleft}


\begin{flushleft}
5) N103 - Weak Acid Inlet
\end{flushleft}


\begin{flushleft}
- Diameter : 4 inches
\end{flushleft}


\begin{flushleft}
6) N104 - Process Water Inlet
\end{flushleft}


\begin{flushleft}
- Diameter : 4 inches
\end{flushleft}


\begin{flushleft}
7) N105 - Tail Gas Outlet
\end{flushleft}


\begin{flushleft}
- Diameter : 8 inches
\end{flushleft}





\begin{flushleft}
PLATE 21
\end{flushleft}





\begin{flushleft}
N039/N040
\end{flushleft}





\begin{flushleft}
PLATE 20
\end{flushleft}





\begin{flushleft}
N037/N038
\end{flushleft}





\begin{flushleft}
PLATE 19
\end{flushleft}


\begin{flushleft}
44.16 m
\end{flushleft}


\begin{flushleft}
N035/N036
\end{flushleft}





\begin{flushleft}
PLATE 18
\end{flushleft}





\begin{flushleft}
N033/N034
\end{flushleft}





\begin{flushleft}
PLATE 17
\end{flushleft}


\begin{flushleft}
Manhole 4
\end{flushleft}





\begin{flushleft}
N031/N032
\end{flushleft}





\begin{flushleft}
Manholes
\end{flushleft}


\begin{flushleft}
8) Manhole 1-Manhole 7 - Diameter: 0.6 m
\end{flushleft}





\begin{flushleft}
PLATE 16
\end{flushleft}





\begin{flushleft}
N029/N030
\end{flushleft}





\begin{flushleft}
PLATE 15
\end{flushleft}





\begin{flushleft}
N027/N028
\end{flushleft}





\begin{flushleft}
PLATE 14
\end{flushleft}





\begin{flushleft}
Design Notes
\end{flushleft}





\begin{flushleft}
N025/N026
\end{flushleft}





\begin{flushleft}
PLATE 13
\end{flushleft}


\begin{flushleft}
1.2 m
\end{flushleft}





\begin{flushleft}
N103
\end{flushleft}


\begin{flushleft}
Weak Acid Inlet
\end{flushleft}





\begin{flushleft}
1) Distance between the plates : Plate 0 to Plate 12 - 1 m
\end{flushleft}


\begin{flushleft}
Plate 12 to Plate 35 - 1.2 m
\end{flushleft}


\begin{flushleft}
2) Fillet weld joining column cylinder to ellipsoidal heads
\end{flushleft}


\begin{flushleft}
3) Stainless Steel plate is used to join the skirt and the column
\end{flushleft}


\begin{flushleft}
4) Exact representation of manhole and pipe positions is as in figure 2
\end{flushleft}


\begin{flushleft}
5) For figure 1, cooling coils are not diametrically opposite for consecutive
\end{flushleft}


\begin{flushleft}
trays, they are represented as such for ease of viewing.
\end{flushleft}





\begin{flushleft}
N023/N024
\end{flushleft}





\begin{flushleft}
PLATE 12
\end{flushleft}


\begin{flushleft}
1m
\end{flushleft}


\begin{flushleft}
N021/N022
\end{flushleft}





\begin{flushleft}
PLATE 11
\end{flushleft}


\begin{flushleft}
Manhole 3
\end{flushleft}





\begin{flushleft}
N019/N020
\end{flushleft}





\begin{flushleft}
PLATE 10
\end{flushleft}


\begin{flushleft}
N017/N018
\end{flushleft}





\begin{flushleft}
PLATE 9
\end{flushleft}


\begin{flushleft}
N015/N016
\end{flushleft}





\begin{flushleft}
PLATE 8
\end{flushleft}


\begin{flushleft}
N013/N014
\end{flushleft}





\begin{flushleft}
PLATE 7
\end{flushleft}


\begin{flushleft}
N011/N012
\end{flushleft}





\begin{flushleft}
PLATE 6
\end{flushleft}


\begin{flushleft}
N009/N010
\end{flushleft}





\begin{flushleft}
PLATE 5
\end{flushleft}


\begin{flushleft}
Manhole 2
\end{flushleft}





\begin{flushleft}
N007/N008
\end{flushleft}





\begin{flushleft}
PLATE 4
\end{flushleft}


\begin{flushleft}
N005/N006
\end{flushleft}





\begin{flushleft}
PLATE 3
\end{flushleft}


\begin{flushleft}
N003/N004
\end{flushleft}





\begin{flushleft}
PLATE 2
\end{flushleft}


\begin{flushleft}
N001/N002
\end{flushleft}





\begin{flushleft}
PLATE 1
\end{flushleft}


\begin{flushleft}
1m
\end{flushleft}


\begin{flushleft}
N071/N072
\end{flushleft}





\begin{flushleft}
PLATE 0
\end{flushleft}


\begin{flushleft}
1.5 m
\end{flushleft}





\begin{flushleft}
NOX
\end{flushleft}


\begin{flushleft}
Gas
\end{flushleft}


\begin{flushleft}
Inlet
\end{flushleft}





\begin{flushleft}
N101
\end{flushleft}


\begin{flushleft}
NOX Gas Inlet
\end{flushleft}





\begin{flushleft}
0.44 m
\end{flushleft}





\begin{flushleft}
1.56 m
\end{flushleft}





\begin{flushleft}
Process Water Inlet
\end{flushleft}


\begin{flushleft}
4.5 m
\end{flushleft}





\begin{flushleft}
Weak Acid Inlet
\end{flushleft}





\begin{flushleft}
N102
\end{flushleft}


\begin{flushleft}
Product Acid Outlet
\end{flushleft}





\begin{flushleft}
1.8 m
\end{flushleft}


\begin{flushleft}
0.6 m
\end{flushleft}





\begin{flushleft}
Manhole 1
\end{flushleft}





\begin{flushleft}
Cooling/Chilled Water Inlet
\end{flushleft}





\begin{flushleft}
Product Acid Outlet
\end{flushleft}





\begin{flushleft}
\newpage
Appendix D
\end{flushleft}





\begin{flushleft}
DWSIM Flowsheet
\end{flushleft}


\begin{flushleft}
The Process Flowsheet was developed using DWSIM simulations software. It is an Free
\end{flushleft}


\begin{flushleft}
and Open Source software. More info on DWSIM can be found here
\end{flushleft}





86





\begin{flushleft}
\newpage
\newpage
References
\end{flushleft}


\begin{flushleft}
[1] Thyssenkrupp Industrial Solutions Nitric Acid Process Handbook https://d13qmi8c46i38w.cloudfront.net/media/UCPthyssenkruppBAIS/
\end{flushleft}


\begin{flushleft}
assets.files/products\_\_\_services/fertilizer\_plants/nitrate\_plants/
\end{flushleft}


\begin{flushleft}
brochure-nitric-acid\_scr.pdf
\end{flushleft}


\begin{flushleft}
[2] Global Nitric Acid Market By Plant Type, By Sales Channel, By Application, By
\end{flushleft}


\begin{flushleft}
Region, Competition Forecast and Opportunities, 2011-2025, Research and Markets
\end{flushleft}


\begin{flushleft}
Nitric Acid Sample Market Report https://www.researchandmarkets.com/research/qsn6h4/global\_nitric
\end{flushleft}


\begin{flushleft}
[3] Nitric Acid Market Size, Share \& Trends Analysis Report By Application (Fertilizers,
\end{flushleft}


\begin{flushleft}
Adipic Acid, Nitrobenzene, Toulene di-isocynate, Nitrochlorobenzene), By Region,
\end{flushleft}


\begin{flushleft}
Vendor Landscape, And Segment Forecasts, 2012-2022, Grand View Research Nitric
\end{flushleft}


\begin{flushleft}
Acid Sample Market Report www.grandviewresearch.com/industry-analysis/nitric-acid-market
\end{flushleft}


\begin{flushleft}
[4] Nitric Acid IHS Chemical Economics Handbook www.ihsmarkit.com/products/nitric-acid-chemical-economics-handbook.
\end{flushleft}


\begin{flushleft}
html
\end{flushleft}


\begin{flushleft}
[5] Jacob A. Moulijn, Michiel Makkee, Annelies E. Van Diepen, Chemical Process
\end{flushleft}


\begin{flushleft}
Technology, p.260-267, 2nd ed., 2013, John Wiley \& Sons Ltd., Great Britain
\end{flushleft}


\begin{flushleft}
[6] Thiemann M, Scheibler E, Wiegand KW, Nitric acid, nitrous acid, and nitrogen
\end{flushleft}


\begin{flushleft}
oxides, Ullmann's Encyclopedia of Industrial Chemistry, Vol. 23, Wiley--VCH
\end{flushleft}


\begin{flushleft}
Verlag GmbH \& Co. KGaA, 2003. doi:10.1002/14356007.a17 293
\end{flushleft}


\begin{flushleft}
[7] Deepak Fertilisers and Petrochemicals Corporation Ltd., MIDC Road, Taloja, Navi
\end{flushleft}


\begin{flushleft}
Mumbai, Maharashtra - 410208, India
\end{flushleft}


88





\begin{flushleft}
\newpage
References
\end{flushleft}





89





\begin{flushleft}
[8] Martyn S. Ray (Curtin University of Technology, Western Australia), David W.
\end{flushleft}


\begin{flushleft}
Johnston (Shell Refining(Australia) Pvt. Ltd.), Chemical Engineering Design Project:
\end{flushleft}


\begin{flushleft}
A Case Study Approach, Topics in Chemical Engineering, ISSN: 0277-5883, Vol.
\end{flushleft}


\begin{flushleft}
6, Gordon and Breach Science Publishers, 1989, United Kingdom
\end{flushleft}


\begin{flushleft}
http://inspect.nigc.ir/Portal/Images/Images\_Traning/files/files/
\end{flushleft}


\begin{flushleft}
chemist\%20book/Chemical\%20Engineering\%20Design\%20Project\%20-\%20A\%
\end{flushleft}


\begin{flushleft}
20Case\%20Study\%20Approach.PDF
\end{flushleft}


\begin{flushleft}
[9] Hirokazu Tsukahara,Takanobu Ishida, Mitsufumi Mayumi, Gas-Phase Oxidation of
\end{flushleft}


\begin{flushleft}
Nitric Oxide: Chemical Kinetics and Rate Constant, Nitric Oxide, Vol. 3 - Issue 3,
\end{flushleft}


\begin{flushleft}
Elsevier, 1999, Japan. doi:10.1006/niox.1999.0232
\end{flushleft}


\begin{flushleft}
[10] Max S. Peters, Klaus D. Timmerhaus, Plant Design And Economics For
\end{flushleft}


\begin{flushleft}
Chemical Engineers, 4th ed., ISBN - 0-07-100871-3, McGraw Hill Inc., International Edition, 1991, United States
\end{flushleft}


\begin{flushleft}
[11] Stanley M. Walas, Chemical Process Equipment Selection and Design, Butterworth - Heinemann Series in Chemical Engineering, ISBN - 0-7506-9385-1, Butterworth - Heinemann, Reed Publishing, 1990, United States
\end{flushleft}


\begin{flushleft}
[12] The Chemical Engineering Cost Index, Chemical Engineering Online https://www.chemengonline.com/pci
\end{flushleft}


\begin{flushleft}
[13] Mr. Amrish Dholakia, Engineering Manager \& General Manager-Process, Uhde
\end{flushleft}


\begin{flushleft}
House, Thyssenkrupp Industrial Solutions (formerly Uhde India), Vikhroli, Mumbai
\end{flushleft}


- 400083


\begin{flushleft}
[14] Gold Price India Website - https://www.goldpriceindia.com
\end{flushleft}


\begin{flushleft}
[15] Infomine - http://www.infomine.com/investment/metal-prices/rhodium/
\end{flushleft}


\begin{flushleft}
[16] CL 407 - Process Equipment Design, Handouts and Reading Material by Prof. Arun
\end{flushleft}


\begin{flushleft}
S. Moharir, Prof. Yogendra Shastri, Indian Institute of Technology Bombay, Mumbai
\end{flushleft}


- 400076


\begin{flushleft}
[17] United Nations Industrial Development Organisation (UNIDO), Internation Fertiliser Development Centre (IFDC), Fertilizer Manual, ISBN - 0-7923-5032-4,
\end{flushleft}


\begin{flushleft}
Kluwer Academic Publishers, 1998, Netherlands
\end{flushleft}





\begin{flushleft}
\newpage
References
\end{flushleft}





90





\begin{flushleft}
[18] R.K. Sinnott, J.M. Coulson, J.F. Richardson, Chemical Engineering Design,
\end{flushleft}


\begin{flushleft}
Chemical Engineering, p.564-586, Vol. 6 3rd ed., ISBN - 0-7506-4142-8, Butterworth - Heinemann, 1999, Great Britain
\end{flushleft}


\begin{flushleft}
[19] G.S.R. 1607(E), Ministry of Environment, Forest and Climate Change, Central
\end{flushleft}


\begin{flushleft}
Government of India, 29th December 2017
\end{flushleft}


\begin{flushleft}
https://drive.google.com/open?id=1xkQjup6FQcv8fdppsAhLQLrq0oE6upVp
\end{flushleft}


\begin{flushleft}
[20] Groves, M.C.E., Sasonow, A., Hydrogen and Nitrates Division, Uhde GmbH, Uhde
\end{flushleft}


\begin{flushleft}
EnviNOX R technology for NOX and N2 O abatement: a contribution to reducing emissions from nitric acid plants, Journal of Integrative Environmental Sciences,
\end{flushleft}


\begin{flushleft}
7(sup1), p.211222, 2010, doi:10.1080/19438151003621334
\end{flushleft}


\begin{flushleft}
[21] AM0028: N2 O destruction in the tail gas of Caprolactam production plants - Ver.
\end{flushleft}


\begin{flushleft}
06.0, United Nations Framework Convention on Climate Change
\end{flushleft}


\begin{flushleft}
https:
\end{flushleft}


\begin{flushleft}
//cdm.unfccc.int/filestorage/b/g/IV326LBA5XCTF04RUQ7MWDKG8SPNZ1.pdf/
\end{flushleft}


\begin{flushleft}
EB73\_repan05\_AM0028\_ver06.0.pdf?t=a2d8cHBxcm1ifDCgYNMsIPmIXChVn94xNxCI
\end{flushleft}





\newpage



\end{document}
